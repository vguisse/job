\textbf{Commun à tous les candidats}

\medskip

\begin{enumerate}
\item \textbf{Restitution organisée de connaissances}

\medskip
 
Démontrer à l'aide de la définition et des deux propriétés ci-dessous que si 
$\left(u_{n}\right)$ et $\left(v_{n}\right)$ sont deux suites adjacentes, alors elles sont convergentes et elles ont la même limite.

\setlength\parindent{5mm}
\begin{itemize}
\item[] Définition : deux suites sont adjacentes lorsque l'une est croissante, l'autre est décroissante et la différence des deux converge vers $0$. 
\item[] Propriété 1 : si deux suites $\left(u_{n}\right)$ et $\left(v_{n}\right)$ sont adjacentes avec $\left(u_{n}\right)$ croissante et $\left(v_{n}\right)$ décroissante alors pour tout entier naturel $n,~ v_{n} \geqslant u_{n}$. 
\item[] Propriété 2 : toute suite croissante et majorée converge ; toute suite décroissante et minorée converge.
\end{itemize}
\setlength\parindent{0mm}
 
\emph{Dans la suite de cet exercice, toute trace de recherche, même incomplète, ou d'initiative même non fructueuse, sera prise en compte dans l'évaluation.}
 
\item Dans les cas suivants, les suites $\left(u_{n}\right)$ et $\left(v_{n}\right)$ ont-elles la même limite ? Sont-elles adjacentes ? 

Justifier les réponses. 
	\begin{enumerate}
		\item $u_{n} = 1 - 10^{-n}$ et $v_{n} = 1 + 10^{-n}$ ;
		\item $u_{n} = \sqrt{n + 1}$ et $v_{n} = \sqrt{n + 1} + \dfrac{1}{n}$ ; 
		\item $u_{n} = 1 -  \dfrac{1}{n}$ et  $v_{n} = 1 +  \dfrac{(-1)^n}{n}$. 
	\end{enumerate}
\item On considère un nombre réel $a$ positif et les suites $\left(u_{n}\right)$ et $\left(v_{n}\right)$ 
définies pour tout nombre entier naturel $n$ non nul par : $u_{n} = 1 - \dfrac{1}{n}$ et $v_{n} =  \e^{  a + \dfrac{1}{n}}$.

Existe-t-il une valeur de $a$ telle que les suites soient adjacentes ? 
\end{enumerate}