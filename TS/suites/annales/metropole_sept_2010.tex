\textbf{ \hfill 5 points}
 
\textbf{Commun à tous les candidats}

\medskip
 
Soit $\left(u_{n}\right)$ la suite définie par $u_{0} = 5$ et pour tout nombre entier naturel $n$, par  $u_{n+1} = \dfrac{4u_{n} - 1}{u_{n} +2}$.

\medskip
 
Si $f$ est la fonction définie sur l'intervalle $]- 2~;~+ \infty[$ par $f(x) = \dfrac{4x - 1}{x + 2}$, alors on a, pour tout nombre entier naturel $n,~ u_{n+1} = f\left(u_{n}\right)$.
 
On donne en annexe 2 (à rendre avec la copie) une partie de la courbe représentative $\mathcal{C}$ de la fonction $f$ ainsi que la droite $\Delta$ d'équation $y = x$.
 
\begin{enumerate}
\item 
	\begin{enumerate}
		\item Sur l'axe des abscisses, placer $u_{0}$ puis construire $u_{1},~ u_{2}$ et $u_{3}$ en laissant apparents les traits de construction. 
		\item Quelles conjectures peut-on émettre sur le sens de variation et sur la convergence de la suite $\left(u_{n}\right)$ ?
	\end{enumerate} 
\item
	\begin{enumerate}
		\item Démontrer par récurrence que, pour tout nombre entier naturel $n$, on a $u_{n} - 1 > 0.$
		\item \emph{Dans cette question, toute trace de recherche, même incomplète, ou d'initiative même non fructueuse, sera prise en compte dans l'évaluation.}
		 
Valider par une démonstration les conjectures émises à la question 1. b.
	\end{enumerate} 
\item Dans cette question, on se propose d'étudier la suite $\left(u_{n}\right)$ par une autre méthode, en déterminant une expression de $u_{n}$ en fonction de $n$.
 
Pour tout nombre entier naturel $n$, on pose $v_{n} = \dfrac{1}{u_{n} - 1}$. 
	\begin{enumerate}
		\item Démontrer que la suite $\left(v_{n}\right)$ est une suite arithmétique de raison $\dfrac{1}{3}$. 
		\item  Pour tout nombre entier naturel $n$, exprimer $v_{n}$ puis $u_{n}$ en fonction de $n$. 
		\item  En déduire la limite de la suite $\left(u_{n}\right)$. 
	\end{enumerate}
\end{enumerate}

\vspace{1cm} 
\begin{center}

(à rendre avec la copie) 

\vspace{2cm}

\psset{unit=1.5cm}
\begin{pspicture}(-2,-2.5)(7,4.25)
\psaxes[linewidth=1.5pt]{->}(0,0)(-2,-2.5)(7,4.25)
\psplot[linewidth=1.5pt]{-2}{4.25}{x}
\psplot[linewidth=1.5pt,linecolor=blue,plotpoints=8000]{-0.62}{6.25}{4 x mul 1 sub x  2 add div}
\uput[ul](0,0){O} \uput[ul](2,2){$\Delta$} \uput[u](3,2.2){$\mathcal{C}$}
\end{pspicture}
\end{center}