%\textbf{\textsc{Exercice 1} \hfill 4 points}
 
\textbf{Commun  à tous les candidats}
 
Certains\baremeexo{13} résultats de la PARTIE A pourront être utilisés dans la PARTIE B, mais les deux parties peuvent être traitées indépendamment l'une de l'autre. 

\medskip

 \textbf{PARTIE A :}
  
On définit :
\begin{itemize}
\item la suite $\left(u_{n}\right)$ par : $u_{0} = 13$ et, pour tout entier naturel $n,~ u_{n+1}  = \dfrac{1}{5}u_{n} +\dfrac{4}{5}$. 
\item la suite $\left(S_{n}\right)$ par : pour tout entier naturel $n,~ S_{n} = \displaystyle\sum_{k=0}^n u_{k} =  u_{0} +u_{1} +u_{2} + \cdots + u_{n}$.
\end{itemize}

\begin{enumerate}
\item  Montrer\baremeque{3} par récurrence que, pour tout entier naturel $n, u_{n} =  1 + \dfrac{12}{5^n}$.  

En déduire la limite de la suite $\left(u_{n}\right)$. 
\item  
	\begin{enumerate}
		\item  Déterminer\baremeque{1} le sens de variation de la suite $\left(S_{n}\right)$. 
		\item Calculer\baremeque{3} $S_{n}$ en fonction de $n$. 
		\item Déterminer\baremeque{2} la limite de la suite $\left(S_{n}\right)$. 
	\end{enumerate}
\end{enumerate}

\medskip

\textbf{PARTIE B : }
 
Etant donnée une suite $\left(x_{n}\right)$, de nombres réels, définie pour tout entier naturel $n$, on considère la suite $\left(S_{n}\right)$ 
définie par $S_{n} =  \displaystyle\sum_{k=0}^n x_{k}$.
 
Indiquer pour chaque proposition suivante si elle est vraie ou fausse.

 Justifier dans chaque cas. 
 
\begin{itemize}
\item[] Proposition 1: si\baremeque{2} la suite $\left(x_{n}\right)$ est convergente, alors la suite $\left(S_{n}\right)$ l'est aussi. 
\item[] Proposition 2 : les\baremeque{2} suites $\left(x_{n}\right)$ et $\left(S_{n}\right)$ ont le même sens de variation. 
\end{itemize}