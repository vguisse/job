On considère la suite\index{suite} $(u_n)$ définie par $u_0=\dfrac{1}{2}$ et telle que pour tout entier naturel~$n$, 

\[u_{n+1} = \dfrac{3u_n}{1+2u_n}\]

\begin{enumerate}
\item
  	\begin{enumerate}
  		\item Calculer $u_1$ et $u_2$.
  		\item Démontrer, par récurrence, que pour tout entier naturel $n$, $0 < u_n$.
  	\end{enumerate}
\item On admet que pour tout entier naturel $n$, $u_n<1$.
  	\begin{enumerate}
  		\item Démontrer que la suite $\left(u_n\right)$ est croissante.
  		\item Démontrer que la suite $\left(u_n\right)$ converge.
  	\end{enumerate}
\item Soit $\left(v_n\right)$ la suite définie, pour tout entier naturel $n$, par $v_n = \dfrac{u_n}{1 - u_n}$.
  	\begin{enumerate}
  		\item Montrer que la suite $(v_n)$ est une suite géométrique de raison 3.
  		\item Exprimer pour tout entier naturel $n$, $v_n$ en fonction de $n$.
  		\item En déduire que, pour tout entier naturel $n$, $u_n = \dfrac{3^n}{3^n+1}$.
  		\item Déterminer la limite de la suite $(u_n)$.
  	\end{enumerate}
\end{enumerate}