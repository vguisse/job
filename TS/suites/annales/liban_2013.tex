On considère la suite\index{suite} numérique $\left(v_{n}\right)$ définie pour tout entier naturel $n$ par 

$\left\{\begin{array}{l c l}
v_{0} &=& 1\\ 	 
v_{n + 1}&=& \dfrac{9}{6 - v_{n}}
\end{array}\right.$ 

\bigskip

\textbf{Partie A}

\medskip
 
\begin{enumerate}
\item On souhaite écrire un algorithme \index{algorithme} affichant, pour un entier naturel $n$ donné, tous les termes de la suite, du rang $0$ au rang $n$.
 
Parmi les trois algorithmes suivants, un seul convient. Préciser lequel en justifiant la réponse.

\medskip

\hspace{-1cm} 
\begin{small}
\begin{tabularx}{1.1\linewidth}{|X|c|X|c|X|}\cline{1-1}\cline{3-3}\cline{5-5}
\multicolumn{1}{|c|}{\textbf{Algorithme \No 1}}&&\multicolumn{1}{|c|}{\textbf{Algorithme \No 2}}&&\multicolumn{1}{|c|}{\textbf{Algorithme \No 3}}\\ \cline{1-1}\cline{3-3}\cline{5-5}
\textbf{Variables :}&&\textbf{Variables :}&&\textbf{Variables :}\\
$v$ est un réel&&$v$ est un réel&&$v$ est un réel\\ 
$i$ et $n$ sont des entiers naturels&&$i$ et $n$ sont des entiers naturels&&$i$ et $n$ sont des entiers naturels\\
~&&&&\\ 
\textbf{Début de l'algorithme :}&&\textbf{Début de l'algorithme :}&& \textbf{Début de l'algorithme :}\\ 
Lire $n$&&Lire $n$&&Lire $n$\\ 
$v$ prend la valeur $1$&&Pour $i$ variant de $1$ à $n$ faire&&$v$ prend la valeur $1$\\ 
Pour $i$ variant de $1$ à $n$ faire&&$v$ prend la valeur $1$&& Pour $i$ variant de $1$ à $n$ faire\\ 
\hspace{0.2cm}$v$ prend la valeur $\dfrac{9}{6 - v}$&&\hspace{0.2cm}Afficher $v$&& \hspace{0.2cm}Afficher $v$\\  
Fin pour&&$v$ prend la valeur $\dfrac{9}{6 - v}$&&$v$ prend la valeur $\dfrac{9}{6 - v}$\\
Afficher $v$&&Fin pour&&Fin pour\\
&&&&Afficher $v$\\
\textbf{Fin algorithme}&&\textbf{Fin algorithme}&&\textbf{Fin algorithme}\\  \cline{1-1}\cline{3-3}\cline{5-5}
\end{tabularx}
\end{small}

\item Pour $n = 10$ on obtient l'affichage suivant :

\medskip
\begin{tabularx}{\linewidth}{|*{10}{>{\centering \arraybackslash}X|}}\hline
1&1,800&2,143&2,333&2,455&2,538&2,600&2,647&2,684&2,714\\ \hline
\end{tabularx}
\medskip

Pour $n = 100$, les derniers termes affichés sont :

\medskip
\begin{tabularx}{\linewidth}{|*{10}{>{\centering \arraybackslash}X|}}\hline
2,967&2,968&2,968&2,968&2,969&2,969&2,969&2,970&2,970&2,970\\ \hline
\end{tabularx}
\medskip
 
Quelles conjectures peut-on émettre concernant la suite $\left(v_{n}\right)$ ? 
\item 
	\begin{enumerate}
		\item Démontrer par récurrence que, pour tout entier naturel $n,\: 0 < v_{n} < 3$.  
		\item Démontrer que, pour tout entier naturel $n,\: v_{n+1} - v_{n} = \dfrac{\left(3 - v_{n} \right)^2}{6 - v_{n}}$. 
 
La suite $\left(v_{n}\right)$ est-elle monotone ? 
		\item Démontrer que la suite $\left(v_{n}\right)$ est convergente.
	\end{enumerate}
\end{enumerate}
 
\bigskip

\textbf{Partie B Recherche de la limite de la suite } \boldmath $\left(v_{n}\right)$ \unboldmath

\medskip

On considère la suite\index{suite} $\left(w_{n}\right)$ définie pour tout $n$ entier naturel par 

\[w_{n} = \dfrac{1}{v_{n} - 3}.\]
 
\begin{enumerate}
\item Démontrer que $\left(w_{n}\right)$ est une suite arithmétique de raison $- \dfrac{1}{3}$ 
\item En déduire l'expression de $\left(w_{n}\right)$, puis celle de $\left(v_{n}\right)$ en fonction de $n$. 
\item Déterminer la limite de la suite $\left(v_{n}\right)$. 
\end{enumerate}
