\textbf{ \hfill 4 points}

\textbf{Commun à tous les candidats}
 
\medskip

On considère la suite $\left(u_{n}\right)_{n\in \N}$ définie par : 

$u_{0} = 1$ et pour tout $n \in  \N,~ u_{n+1} = \dfrac{1}{3}u_{n} + n - 2$. 

\begin{enumerate}
\item Calculer $u_{1},~u_{2}$ et $u_{3}$.
\item 
	\begin{enumerate}
		\item Démontrer que pour tout entier naturel $n \geqslant  4,~  u_{n} \geqslant 0$. 
		\item En déduire que pour tout entier naturel $n \geqslant 5,~ u_{n} \geqslant n - 3$. 
		\item En déduire la limite de la suite $\left(u_{n}\right)_{n\in \N}$. 
	\end{enumerate}
\item On définit la suite $\left(v_{n}\right)_{n\in \N}$ par : pour tout $n \in \N,~ v_{n} = -2u_{n} + 3n - \dfrac{21}{2}$. 
	\begin{enumerate}
		\item Démontrer que la suite $\left(v_{n}\right)_{n\in \N}$ est une suite géométrique dont on donnera la raison et le 
premier terme. 
		\item  	En d\'eduire que : pour tout $n \in \N,~ u_{n} =  \dfrac{25}{4}\left(\dfrac{1}{3} \right)^n + \dfrac{3}{2}n -\dfrac{21}{4}$. 
		\item  Soit la somme $S_{n}$ définie pour tout entier naturel $n$ par : $S_{n} = \displaystyle\sum_{k=0}^n u_{k}$. 

Déterminer l'expression de $S_{n}$ en fonction de $n$.
	\end{enumerate} 
\end{enumerate}