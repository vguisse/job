%!TEX encoding = UTF-8 Unicode
\documentclass[10pt]{article}
\usepackage[T1]{fontenc}
\usepackage[utf8]{inputenc}
\usepackage{fourier}
\usepackage[scaled=0.875]{helvet}
\renewcommand{\ttdefault}{lmtt}
\usepackage{makeidx}
\usepackage{amsmath,amssymb}
\usepackage{fancybox}
\usepackage[normalem]{ulem}
\usepackage{pifont}
\usepackage{lscape}
\usepackage{multicol}
\usepackage{mathrsfs}
\usepackage{tabularx}
\usepackage{colortbl}
\usepackage{multirow}
\usepackage{textcomp} 
\newcommand{\euro}{\eurologo{}}
%Tapuscrit : Denis Vergès 
\usepackage{pst-plot,pst-tree,pstricks,pst-node,pst-text}
\usepackage{pst-eucl}
\usepackage{pstricks-add}
\newcommand{\R}{\mathbb{R}}
\newcommand{\N}{\mathbb{N}}
\newcommand{\D}{\mathbb{D}}
\newcommand{\Z}{\mathbb{Z}}
\newcommand{\Q}{\mathbb{Q}}
\newcommand{\C}{\mathbb{C}}
\setlength{\textheight}{23.5cm}
\setlength{\voffset}{-1.5cm}
\def\e{\text{e}}
\def\i{\text{i}}
\newcommand{\vect}[1]{\mathchoice%
{\overrightarrow{\displaystyle\mathstrut#1\,\,}}%
{\overrightarrow{\textstyle\mathstrut#1\,\,}}%
{\overrightarrow{\scriptstyle\mathstrut#1\,\,}}%
{\overrightarrow{\scriptscriptstyle\mathstrut#1\,\,}}}
\renewcommand{\theenumi}{\textbf{\arabic{enumi}}}
\renewcommand{\labelenumi}{\textbf{\theenumi.}}
\renewcommand{\theenumii}{\textbf{\alph{enumii}}}
\renewcommand{\labelenumii}{\textbf{\theenumii.}}
\def\Oij{$\left(\text{O},~\vect{\imath},~\vect{\jmath}\right)$}
\def\Oijk{$\left(\text{O},~\vect{\imath},~\vect{\jmath},~\vect{k}\right)$}
\def\Ouv{$\left(\text{O},~\vect{u},~\vect{v}\right)$}
\makeindex
\usepackage{fancyhdr}
\usepackage[colorlinks=true,pdfstartview=FitV,linkcolor=blue,citecolor=blue,urlcolor=blue]{hyperref}
\usepackage[frenchb]{babel}
\usepackage[np]{numprint}
\begin{document}
\setlength\parindent{0mm}
\rhead{A. P. M. E. P.}
\lhead{\small{Baccalauréat S : l'intégrale 2013}}
\renewcommand \footrulewidth{.2pt}
\pagestyle{fancy}
\thispagestyle{empty} 
\begin{center}
{\huge\textbf{\decofourleft~Baccalauréat S  
2013~\decofourright\\ \vspace{1cm} L'intégrale de mars 2013 à  mars 2014}}

\vspace{1cm}

Pour un accès direct cliquez sur les liens {\Large 
\textcolor{blue}{bleus}}
\end{center}

\vspace{1cm}
 
{\Large 
%\hyperlink{Caledoniemars}{Nouvelle-Calédonie   mars 2012} 
%\dotfill 3  \medskip
   
\hyperlink{Pondichery}{Pondichéry  16 avril 2013} \dotfill 3  \medskip

\hyperlink{AmeriqueNord}{Amérique du Nord 30  mai 2012} \dotfill 9  \medskip

\hyperlink{Liban}{Liban  28  mai 2013} \dotfill 14  \medskip

\hyperlink{Polynesie}{Polynésie 7  juin 2013} \dotfill 21  \medskip

\hyperlink{Antilles}{Antilles-Guyane 18 juin 2013} \dotfill 28  \medskip

\hyperlink{Asie}{Asie 19  juin 2013} \dotfill 35  \medskip

\hyperlink{Centres etrangers}{Centres étrangers 12  juin 2013} \dotfill 42  \medskip

\hyperlink{Metropole}{Métropole  20 juin 2013} \dotfill 48  \medskip

%\hyperlink{Polynesiesep}{Polynésie 7  juin 2013} \dotfill 21  \medskip

\hyperlink{Antillessep}{Antilles-Guyane  11 septembre 2013} \dotfill 54  \medskip

\hyperlink{Metropolesep}{Métropole  12 septembre 20133} \dotfill 60  \medskip

\hyperlink{Caledonienov}{Nouvelle-Calédonie   14 novembre 2013} \dotfill 66  \medskip

\hyperlink{AmeriSud}{Amérique du Sud  21 novembre 2013} \dotfill 71  \medskip


\hyperlink{Caledoniemars}{Nouvelle-Calédonie   7 mars 2014} \dotfill 76  \medskip}

\vspace{1cm}À la fin index des notions abordées
\newpage ~
\newpage
%%%%%%%%%%%%%%%%%%%%%%%%%%%%%%%%%%    
\hypertarget{Pondichery}{}

\rhead{\textbf{A. P. M. E. P.}}
\lhead{\small Baccalauréat S}
\lfoot{\small{Pondichéry}}
\rfoot{\small{16 avril 2013}}
\renewcommand \footrulewidth{.2pt}
\pagestyle{fancy}
\thispagestyle{empty}

\begin{center}{\Large\textbf{\decofourleft~Baccalauréat S Pondichéry  16 avril 2013~\decofourright}}
\end{center}

\vspace{0,25cm}

\textbf{\textsc{Exercice 1} \hfill 5 points}
 
\textbf{Commun  à tous les candidats}

\medskip

\textbf{Partie 1}

\medskip
 
On s'intéresse à l'évolution de la hauteur d'un plant de maïs en fonction du temps. Le graphique en annexe 1 représente cette évolution. La hauteur est en mètres et le temps en jours.
 
On décide de modéliser cette croissance par une fonction logistique du type : 

\[h(t) = 	\dfrac{a}{1 + b\text{e}^{- 0,04t}}\]

où $a$ et $b$ sont des constantes réelles positives, $t$ est la variable temps exprimée en jours et $h(t)$ désigne la hauteur du plant, exprimée en mètres.
 
On sait qu'initialement, pour $t = 0$, le plant mesure $0,1$~m et que sa hauteur tend vers une hauteur limite de $2$~m.
 
Déterminer les constantes $a$ et $b$ afin que la fonction $h$ corresponde à la croissance du plant de maïs étudié.

\bigskip
 
\textbf{Partie 2}

\medskip

On considère désormais que la croissance du plant de maïs est donnée par la fonction $f$ définie sur [0~;~250] par 

\[f(t) = \dfrac{2}{1 + 19\text{e}^{- 0,04t}}\] 

\begin{enumerate}
\item Déterminer $f'(t)$ en fonction de $t$ ($f'$ désignant la fonction dérivée de la fonction $f$). 
En déduire les variations de la fonction $f$ sur l'intervalle [0~;~250]. 
\item Calculer le temps nécessaire pour que le plant de maïs atteigne une hauteur supérieure à $1,5$~m. 
\item   
	\begin{enumerate}
		\item Vérifier que pour tout réel $t$ appartenant à l'intervalle [0~;~250] on a 
		
$f(t) = \dfrac{2\text{e}^{0,04t}}{\text{e}^{0,04t} + 19}$. 
 
Montrer que la fonction $F$ définie sur l'intervalle [0~;~250] par 

$F(t) = 50\ln \left(\text{e}^{0,04t} + 19\right)$ est une primitive de la fonction $f$. 
		\item Déterminer la valeur moyenne\index{valeur moyenne} de $f$ sur l'intervalle [50~;~100]. 

En donner une valeur approchée à $10^{-2}$ près et interpréter ce résultat.
	\end{enumerate}  
\item On s'intéresse à la vitesse de croissance du plant de maïs  ; elle est donnée par la fonction dérivée de la fonction $f$.
 
La vitesse de croissance est maximale pour une valeur de $t$. 

En utilisant le graphique donné en annexe, déterminer une valeur approchée de celle-ci. Estimer alors la hauteur du plant.
\end{enumerate}

\vspace{0,5cm}

\textbf{\textsc{Exercice 2} \hfill 4 points}
 
\textbf{Commun  à tous les candidats}

\medskip

\emph{Pour chacune des questions, quatre propositions de réponse sont données dont une seule est exacte. Pour chacune des questions indiquer, sans justification, la bonne réponse sur la copie. Une réponse exacte rapporte $1$ point. Une réponse fausse ou l'absence de réponse ne rapporte ni n'enlève aucun point. Il en est de même dans le cas où plusieurs réponses sont données pour une même question.}

\medskip
 
L'espace est rapporté à un repère orthonormal. $t$ et $t'$ désignent des paramètres réels.

Le plan (P) a pour équation $x - 2y + 3z + 5 = 0$. 

Le plan (S) a pour représentation paramétrique $\left\{\begin{array}{l c l}
x&=&- 2 + t + 2t'\\
y&=&- t - 2t'\\
z&=&- 1 - t + 3t'
\end{array}\right.$ 

La droite (D) a pour représentation paramétrique $\left\{\begin{array}{l c l}
x&=&- 2 + t\\
y&=&- t \\
z&=&- 1 - t
\end{array}\right.$
 
On donne les points de l'espace M$(-1~;~2~;~3)$ et N$(1~;~-2~;~9)$.

\medskip
 
\begin{enumerate}
\item Une représentation paramétrique du plan (P) est : 

\medskip
\scriptsize{\begin{tabularx}{\linewidth}{*{4}{X}}
\textbf{a.~}$\left\{\begin{array}{l c l}
x&=& t\\y&=& 1- 2t\\ z&=& -1 + 3t \end{array}\right.$&
\textbf{b.~}$\left\{\begin{array}{l c l}
x&=& t + 2t'\\y&=& 1- t + t'\\z&=& - 1 - t\end{array}\right.$&
\textbf{c.~}$\left\{\begin{array}{l c l} 	 
x&=&t + t'\\ y&=& 1 - t- 2t'\\z&=& 1 - t - 3t'\end{array}\right.$& 
\textbf{d.~}$\left\{\begin{array}{l c l}
x&=& 1 + 2t + t'\\y&=& 1 - 2t + 2t'\\z&=& - 1 - t'\end{array}\right.$
\end{tabularx}} \normalsize

\medskip 	 	 

\item 
	\begin{enumerate}
		\item La droite (D) et le plan (P) sont sécants au point A$(- 8~;~3~;~2)$. 
		\item La droite (D) et le plan (P) sont perpendiculaires. 
		\item La droite (D) est une droite du plan (P). 
		\item La droite (D) et le plan (P) sont strictement parallèles.
	\end{enumerate} 
\item 
	\begin{enumerate}
		\item La droite (MN) et la droite (D) sont orthogonales. 
		\item La droite (MN) et la droite (D) sont parallèles. 
		\item La droite (MN) et la droite (D) sont sécantes. 
		\item La droite (MN) et la droite (D) sont confondues.
	\end{enumerate}		 
\item 
	\begin{enumerate}
		\item Les plans (P) et (S) sont parallèles. 
		\item La droite $(\Delta)$ de représentation paramétrique $\left\{\begin{array}{l c l}x&=&t\\y&=&- 2 - t\\z &=& -3-t \end{array}\right.$ est la droite d'intersection des plans  (P) et (S). 
		\item Le point M appartient à l'intersection des plans (P) et (S). 
		\item Les plans (P) et (S) sont perpendiculaires.
	\end{enumerate} 
\end{enumerate}
 
\vspace{0,5cm}

\textbf{\textsc{Exercice 3} \hfill 5 points}
 
\textbf{Candidats n'ayant pas suivi l'enseignement de spécialité}

\medskip
 
Le plan complexe est muni d'un repère orthonormé direct \Ouv.
  
On note i le nombre complexe tel que $\text{i}^2 = - 1$.\index{complexes}
 
On considère le point A d'affixe $z_{\text{A}} = 1$ et le point B d'affixe $z_{\text{B}} = \text{i}$.
 
À tout point $M$ d'affixe $z_{M} = x + \text{i}y$, avec $x$ et $y$ deux réels tels que $y \neq 0$, on associe le point $M'$ d'affixe $z_{M'} = - \text{i}z_{M}$.
 
On désigne par $I$ le milieu du segment [A$M$].
 
Le but de l'exercice est de montrer que pour tout point $M$ n'appartenant pas à (OA), la médiane (O$I$) du triangle OA$M$ est aussi une hauteur du triangle OB$M'$ (propriété 1) et que B$M' = 2 \text{O}I$ (propriété 2). 

\medskip
 
\begin{enumerate}
\item Dans cette question et uniquement dans cette question, on prend 

$z_{M} = 2\text{e}^{- \text{i}\frac{\pi}{3}}$. 
	\begin{enumerate}
		\item Déterminer la forme algébrique de $z_{M}$. 
		\item Montrer que $z_{M'} = - \sqrt{3} - \text{i}$.
		
Déterminer le module et un argument de $z_{M'}$. 
		\item Placer les points A, B, $M, M'$ et $I$ dans le repère \Ouv{ }en prenant 2~cm pour unité graphique.
		 
Tracer la droite (O$I$) et vérifier rapidement les propriétés 1 et 2 à l'aide du graphique.
	\end{enumerate} 
\item On revient au cas général en prenant $z_{M} = x + \text{i}y$ avec $y \neq 0$. 
	\begin{enumerate}
		\item Déterminer l'affixe du point $I$ en fonction de $x$ et $y$. 
		\item Déterminer l'affixe du point $M'$ en fonction de $x$ et $y$. 
		\item Écrire les coordonnées des points $I$, B et $M'$. 
		\item Montrer que la droite (O$I$) est une hauteur du triangle OB$M'$. 
		\item Montrer que B$M' = 2 \text{O}I$.
	\end{enumerate} 
\end{enumerate}

\vspace{0,5cm}

\textbf{\textsc{Exercice 3} \hfill 5 points}
 
\textbf{Candidats ayant  suivi l'enseignement de spécialité}

\medskip
 
On étudie l'évolution dans le temps du nombre de jeunes et d'adultes dans une population d'animaux. 

Pour tout entier naturel $n$, on note $j_{n}$ le nombre d'animaux jeunes après $n$ années d'observation et $a_{n}$ le nombre d'animaux adultes après $n$ années d'observation. 

Il y a au début de la première année de l'étude, 200 animaux jeunes et 500 animaux adultes. 

Ainsi $j_{0} = 200$ et $a_{0} = 500$. 

On admet que pour tout entier naturel $n$ on a :
 
\[\left\{\begin{array}{l c l}
j_{n+ 1}& =&0,125j_{n} + 0,525a_{n}\\
a_{n+1} &=& 0,625j_{n} + 0,625a_{n}
\end{array}\right.\] 

On introduit les matrices\index{matrice} suivantes : 

$A = \begin{pmatrix} 0,125 &0,525\\ 0,625& 0,625\\
\end{pmatrix}$ et, pour tout entier naturel $n, U_{n} = \begin{pmatrix}j_{n}\\a_{n}\end{pmatrix}$.

\medskip
 
\begin{enumerate}
\item 
	\begin{enumerate}
		\item Montrer que pour tout entier naturel $n, U_{n+ 1} = A \times U_{n}$. 
		\item Calculer le nombre d'animaux jeunes et d'animaux adultes après un an d'observation puis après deux ans d'observation (résultats arrondis à l'unité près par défaut). 
		\item Pour tout entier naturel $n$ non nul, exprimer $U_{n}$ en fonction de $A^n$ et de $U_{0}$. 
	\end{enumerate}
\item On introduit les matrices suivantes $Q = \begin{pmatrix}7&3\\-5& 5\\
\end{pmatrix}$ et $D = 	\begin{pmatrix}- 0,25&0\\0& 	1\end{pmatrix}$. 
	\begin{enumerate}
		\item On admet que la matrice\index{matrice} $Q$ est inversible et que $Q^{- 1} = \begin{pmatrix} 0,1&-0,06\\0,1& 0,14\end{pmatrix}$. 

Montrer que $Q \times D \times Q^{- 1} = A$. 
		\item Montrer par récurrence sur $n$ que pour tout entier naturel $n$ non nul : $A^n = Q \times D^n \times Q^{- 1}$. 
		\item Pour tout entier naturel $n$ non nul, déterminer $D^n$ en fonction de $n$.
	\end{enumerate} 
\item On admet que pour tout entier naturel $n$ non nul, 

\[A^n = \begin{pmatrix}0,3 + 0,7 \times (- 0,25)^n&0,42 - 0,42 \times (- 0,25)^n \\
0,5 - 0,5 \times (- 0,25)^n& 0,7\phantom{0} + 0,3\phantom{0} \times (- 0,25)^n\\
\end{pmatrix}\]
 
	\begin{enumerate}
		\item En déduire les expressions de $j_{n}$ et $a_{n}$ en fonction de $n$ et déterminer les limites de ces deux suites. 
		\item Que peut-on en conclure pour la population d'animaux étudiée ? 
	\end{enumerate}
\end{enumerate}

\vspace{0,5cm}
 
\textbf{\textsc{Exercice 4} \hfill 6 points}
 
\textbf{Commun  à tous les candidats}

\medskip

Dans une entreprise, on s'intéresse à la probabilité\index{probabilité} qu'un salarié soit absent durant une période d'épidémie de grippe.

\medskip
\setlength\parindent{8mm}
\begin{itemize}
\item[$\bullet~~$] Un salarié malade est absent 
\item[$\bullet~~$] La première semaine de travail, le salarié n'est pas malade. 
\item[$\bullet~~$] Si la semaine $n$ le salarié n'est pas malade, il tombe malade la semaine $n + 1$ avec une probabilité égale à $0,04$. 
\item[$\bullet~~$] Si la semaine $n$ le salarié est malade, il reste malade la semaine $n + 1$ avec une probabilité égale à $0,24$.
\end{itemize}
\setlength\parindent{0mm}
 
On désigne, pour tout entier naturel $n$ supérieur ou égal à 1, par $E_{n}$ l'évènement \og le salarié est absent pour cause de maladie la $n$-ième semaine \fg. On note $p_{n}$ la probabilité de l'évènement $E_{n}$.
 
On a ainsi : $p_{1} = 0$ et, pour tout entier naturel $n$ supérieur ou égal à 1 : $0 \leqslant  p_{n} < 1$.

\medskip
 
\begin{enumerate}
\item 
	\begin{enumerate}
		\item Déterminer la valeur de $p_{3}$ à l'aide d'un arbre\index{arbre de probabilités} de probabilité. 
		\item Sachant que le salarié a été absent pour cause de maladie la troisième semaine, déterminer la probabilité qu'il ait été aussi absent pour cause de maladie la deuxième semaine.
	\end{enumerate} 
\item 
	\begin{enumerate}
		\item Recopier sur la copie et compléter l'arbre\index{arbre de probabilités} de probabilité donné ci-dessous
		
\begin{center}
\pstree[linecolor=blue,treemode=R]{\TR{}}
{
	\pstree{\TR{$E_{n}$}\taput{$p_{n}$}}
	  { 
		  \TR{$E_{n+1}$}\taput{\ldots}
		  \TR{$\overline{E_{n+1}}$}\tbput{\ldots}	   
	  }
	\pstree{\TR{$\overline{E_{n}}$}\tbput{\ldots}}
	  {
		  \TR{$E_{n+1}$}\taput{\ldots}
		  \TR{$\overline{E_{n+1}}$}\tbput{\ldots}		  
	  }
}
\end{center}

		\item Montrer que, pour tout entier naturel $n$ supérieur ou égal à 1,
		
		 $p_{n+ 1} = 0,2p_{n} + 0,04$. 
		\item Montrer que la suite\index{suite} $\left(u_{n}\right)$ définie pour tout entier naturel $n$ supérieur ou égal à 1 par $u_{n} = p_{n} - 0,05$ est une suite géométrique dont on donnera le premier terme et la raison $r$.
		
En déduire l'expression de $u_{n}$ puis de $p_{n}$ en fonction de $n$ et $r$. 
		\item En déduire la limite de la suite $\left(p_{n}\right)$. 
		\item On admet dans cette question que la suite $\left(p_{n}\right)$ est croissante. On considère l'algorithme\index{algorithme}  suivant : 

\begin{center}
\begin{tabularx}{0.9\linewidth}{|l X|}\hline
Variables		& K et J sont des entiers naturels, P est un nombre réel\\ 
Initialisation 	&P prend la valeur $0$\\ 
				&J prend la valeur $1$\\ 
Entrée			& Saisir la valeur de K\\ 
Traitement		&Tant que P $< 0,05 - 10^{- \text{K}}$\\ 
				&\quad P prend la valeur $0,2 \times \text{P} + 0,04$\\
				&\quad  J prend la valeur J $+ 1$\\ 
				&Fin tant que \\
Sortie			&Afficher J \\ \hline
\end{tabularx}
\end{center}

À quoi correspond l'affichage final J ?
 
Pourquoi est-on sûr que cet algorithme s'arrête ?
	\end{enumerate} 
\item Cette entreprise emploie 220 salariés. Pour la suite on admet que la probabilité\index{probabilité} pour qu'un salarié soit malade une semaine donnée durant cette période d'épidémie est égale à $p = 0,05$.
 
On suppose que l'état de santé d'un salarié ne dépend pas de l'état de santé de ses collègues.
 
On désigne par $X$ la variable aléatoire qui donne le nombre de salariés malades une semaine donnée. 
	\begin{enumerate}
		\item Justifier que la variable aléatoire $X$ suit une loi binomiale\index{loi binomiale} dont on donnera les paramètres.
		 
Calculer l'espérance mathématique $\mu$ et l'écart type $\sigma$ de la variable aléatoire $X$. 
		\item On admet que l'on peut approcher la loi de la variable aléatoire $\dfrac{X - \mu}{\sigma}$ 
		
par la loi normale\index{loi normale}  centrée réduite c'est-à-dire de paramètres $0$ et $1$.
		 
On note $Z$ une variable aléatoire suivant la loi normale centrée réduite. 

Le tableau suivant donne les probabilités de l'évènement $Z < x$ pour quelques valeurs du nombre réel $x$.

\medskip
\begin{tabularx}{\linewidth}{|c|*{11}{>{\scriptsize \centering \arraybackslash}X|}}\hline
$x$& $-1,55$ &$-1,24$ &$-0,93$ &$- 0,62$ &$- 0,31$ &0,00 &0,31 &0,62 &0,93 &1,24 &1,55\\ \hline 
$P(Z < x)$& 0,061 &0,108 &0,177 &0,268 &0,379 &0,500 &0,621 &0,732 &0,823 &0,892 &0,939\\ \hline
\end{tabularx}

\medskip
 
Calculer, au moyen de l'approximation proposée en question b., une valeur approchée à $10^{-2}$ près de la probabilité de l'évènement : \og le nombre de salariés absents dans l'entreprise au cours d'une semaine donnée est supérieur ou égal  à 7 et inférieur ou égal à 15 \fg. 
	\end{enumerate}
\end{enumerate}

\newpage
\begin{center}
\textbf{Annexe (Exercice 1)}

\bigskip

\psset{xunit=0.05cm,yunit=4cm,comma=true}
\begin{pspicture}(-5,-0.2)(230,2.2)
\multido{\n=0+20}{12}{\psline[linestyle=dashed,linewidth=0.2pt,linecolor=orange](\n,0)(\n,2.2)}
\multido{\n=0.0+0.2}{12}{\psline[linestyle=dashed,linewidth=0.2pt,linecolor=orange](0,\n)(230,\n)}
\psaxes[linewidth=1.5pt,Dx=20,Dy=0.2]{->}(0,0)(-5,-0.18)(230,2.2)
\psplot[plotpoints=8000,linewidth=1.25pt,linecolor=blue]{0}{230}{2 19 2.71828 0.04 x mul exp div 1 add div}
\psline(0,2)(230,2)\uput[d](10,2){$y = 2$}
\uput[u](210,0){temps $t$ (en jours)}
\uput[r](0,2.15){hauteur (en mètres)}
\uput[dl](0,0){O}
\end{pspicture}
\end{center}
%%%%%%%%%%%%%%%%  fin Pondichéry avril 2013
\newpage
%%%%%%%%%%%%%%%%  Amérique du Nord mai 203
\hypertarget{AmeriqueNord}{}

\lfoot{\small{Amérique du Nord}}
\rfoot{\small 30  mai 2013}
\pagestyle{fancy}
\thispagestyle{empty}
\begin{center}\textbf{Durée : 4 heures }

\vspace{0,5cm}

{\Large\textbf{\decofourleft~Baccalauréat S Amérique du Nord
~\decofourright\\30  mai 2013}}
\end{center}

\vspace{0,5cm}

\textbf{Exercice 1 \hfill  5 points}

\textbf{Commun à tous les candidats}

\medskip

On se place dans l'espace muni d'un repère orthonormé.
 
On considère les points A(0~;~4~;~1), B (1~;~3~;~0), C$(2~;~-1~;~- 2)$ et D $(7~;~- 1~;~4)$.

\medskip
 
\begin{enumerate}
\item Démontrer que les points A, B et C ne sont pas alignés. 
\item Soit $\Delta$ la droite passant par le point D et de vecteur directeur  
$\vect{u}(2~;~- 1~;~3)$. 
	\begin{enumerate}
		\item Démontrer que la droite $\Delta$ est orthogonale au plan (ABC). 
		\item En déduire une équation cartésienne du plan (ABC). 
		\item Déterminer une représentation paramétrique de la droite $\Delta$. 
		\item Déterminer les coordonnées du point H, intersection de la droite $\Delta$ et du plan (ABC).
	\end{enumerate} 
\item Soit $\mathcal{P}_{1}$ le plan d'équation $x + y + z = 0$ et $\mathcal{P}_{2}$ le plan d'équation $x + 4y + 2 = 0$. 
	\begin{enumerate}
		\item Démontrer que les plans $\mathcal{P}_{1}$ et $\mathcal{P}_{2}$ sont sécants. 
		\item Vérifier que la droite $d$, intersection des plans $\mathcal{P}_{1}$ et $\mathcal{P}_{2}$, a pour représentation paramétrique 
$\left\{\begin{array}{l c l}
x&=&-4t-2\\ 
y &=&t\\ 
z &=& 3t + 2
\end{array}\right., \:\: t \in \R$. 
		\item La droite $d$ et le plan (ABC) sont-ils sécants ou parallèles ? 
	\end{enumerate}
\end{enumerate}

\vspace{0,5cm}

\textbf{Exercice 2 \hfill  5 points}

\textbf{Candidats N'AYANT PAS SUIVI l'enseignement de spécialité mathématiques}

\medskip
  
On considère la suite\index{suite} $\left(u_{n}\right)$ définie par $u_{0} = 1$ et, pour tout entier naturel $n$,

\[ u_{n+1} = \sqrt{2u_{n}}.\]

\begin{enumerate}
\item On considère l'algorithme\index{algorithme} suivant :

\begin{center}
\begin{tabular}{|ll|}\hline
Variables :&$n$ est un entier naturel\\ 
&$u$ est un réel positif\\
Initialisation :& Demander la valeur de $n$\\
 	&Affecter à $u$ la valeur 1\\
Traitement :&Pour $i$ variant de 1 à $n$ :\\
	&\hspace{0.3cm}| Affecter à $u$ la valeur $\sqrt{2u}$\\
	&Fin de Pour\\ 
Sortie :& Afficher $u$\\ \hline
\end{tabular}
\end{center} 
 
	\begin{enumerate}
		\item Donner une valeur approchée à $10^{-4}$ près du résultat qu'affiche cet algorithme lorsque l'on choisit $n = 3$.
		\item Que permet de calculer cet algorithme ? 
		\item Le tableau ci-dessous donne des valeurs approchées obtenues à l'aide de cet algorithme pour certaines valeurs de $n$.
		
\begin{center}
\begin{tabularx}{0.8\linewidth}{|c|*{5}{>{\centering \arraybackslash}X|}}\hline
$n$				& 1 		&5 			&10 		&15 		&20\\ \hline 
Valeur affichée	&\np{1,4142} &\np{1,9571} &\np{1,9986} &\np{1,9999} &\np{1,9999}\\ \hline
\end{tabularx}
\end{center}

Quelles conjectures peut-on émettre concernant la suite $\left(u_{n}\right)$ ?
	\end{enumerate} 
\item
	\begin{enumerate}
		\item Démontrer que, pour tout entier naturel $n,\: 0 < u_{n} \leqslant 2$. 
		\item Déterminer le sens de variation de la suite $\left(u_{n}\right)$. 
		\item Démontrer que la suite $\left(u_{n}\right)$ est convergente. On ne demande pas la valeur de sa limite.
	\end{enumerate} 
\item On considère la suite\index{suite} $\left(v_{n}\right)$ définie, pour tout entier naturel $n$, par $v_{n} = \ln u_{n} - \ln 2$. 
	\begin{enumerate}
		\item Démontrer que la suite $\left(v_{n}\right)$ est la suite géométrique de raison $\dfrac{1}{2}$ et de premier terme  
$v_{0} = - \ln 2$. 
		\item Déterminer, pour tout entier naturel $n$, l'expression de $v_{n}$ en fonction de $n$, puis de $u_{n}$ en fonction de $n$. 
		\item Déterminer la limite de la suite $\left(u_{n}\right)$. 
		\item Recopier l'algorithme\index{algorithme} ci-dessous et le compléter par les instructions du traitement et de la sortie, de façon à afficher en sortie la plus petite valeur de $n$ telle que $u_{n} > 1,999$.
		
\begin{center}
\begin{tabular}{|l l|}\hline		 
Variables :		&$n$ est un entier naturel\\
				& $u$ est un réel\\
Initialisation :&Affecter à $n$ la valeur $0$\\
				&Affecter à $u$ la valeur 1\\ 
Traitement :	&\\
				&\\ 
Sortie :		&\\ \hline
\end{tabular}
\end{center}
	\end{enumerate} 
\end{enumerate}

\vspace{0,5cm}

\textbf{Exercice 2 \hfill  5 points}

\textbf{Candidats AYANT  SUIVI l'enseignement de spécialité mathématiques}

\medskip

\textbf{Partie A}

\medskip
 
On considère l'algorithme\index{algorithme} suivant :

\begin{center}
\begin{tabular}{|l l|}\hline
Variables :& 	$a$ est un entier naturel\\
	& $b$ est un entier naturel\\ 
	&$c$ est un entier naturel\\ 
Initialisation :& 	Affecter à $c$ la valeur $0$\\
			& Demander la valeur de $a$\\
			& Demander la valeur de $b$\\ 
Traitement :& 	Tant que $a > b$\\ 
			&\hspace{0.3cm}\begin{tabular}{|l}
			Affecter à $c$ la valeur $c + 1$\\ 
			Affecter à $a$ la valeur $a - b$
			\end{tabular}\\
			& Fin de tant que \\
Sortie : 	&Afficher $c$\\
			& Afficher $a$\\ \hline
			\end{tabular}
\end{center} 

\begin{enumerate}
\item Faire fonctionner cet algorithme avec $a =	13$ et $b =	4$ en indiquant les valeurs des variables à chaque étape. 
\item Que permet de calculer cet algorithme ?
\end{enumerate}
 
\bigskip

\textbf{Partie B}

\medskip

À chaque lettre de l'alphabet, on associe, grâce au tableau ci-dessous, un nombre entier compris entre $0$ et $25$.

\medskip

\begin{tabularx}{\linewidth}{|*{13}{>{\centering \arraybackslash}X|}}\hline
A	&B&C &D&E &F&G  &H&I&J& K &L &M \\ \hline
0	&1 	&2	&3 	&4 	&5 	&6 	&7	&8 	&9 	&10 	&11 	&12\\ \hline \hline 
N	&O	&P	&Q	&R &S	& T	&U 	&V 	&W 	&X 		&Y		&Z\\ \hline 
13	&14	&15	&16	&17 &18 &19 &20	&21	&22	&23 	&24		&25\\ \hline
\end{tabularx}

\medskip 

On définit un procédé de codage de la façon suivante :
 
\hspace{0,5cm}\begin{tabular}{l p{9.5cm}}
\emph{Étape} 1 :& 	À la lettre que l'on veut coder, on associe le nombre $m$ correspondant dans le tableau.\\ 
\emph{Étape} 2 :& 	On calcule le reste de la division euclidienne\index{division euclidienne} de $9m + 5$ par $26$ et on le note $p$.\\
\emph{Étape} 3 :& 	Au nombre $p$, on associe la lettre correspondante dans le tableau.\\
\end{tabular}

\medskip
 
\begin{enumerate}
\item Coder la lettre U. 
\item Modifier l'algorithme de la partie A pour qu'à une valeur de $m$ entrée par l'utilisateur, il affiche la valeur de $p$, calculée à l'aide du procédé de codage précédent.
\end{enumerate}
 
\bigskip

\textbf{Partie C}

\medskip
 
\begin{enumerate}
\item Trouver un nombre entier $x$ tel que $9x \equiv 1\quad [26]$. 
\item Démontrer alors l'équivalence : 

\[9m + 5 \equiv  p\quad [26] \iff m \equiv  3p -  15\quad  [26].\]
 
\item Décoder alors la lettre B. 
\end{enumerate}

\vspace{0,5cm}

\textbf{Exercice 3 \hfill  5 points}

\textbf{Commun à tous les candidats}

\begin{center}
\emph{Les parties A B et C peuvent être traitées indépendamment les unes des autres}\end{center}
 
Une boulangerie industrielle utilise une machine pour fabriquer des pains de campagne pesant en moyenne 400~grammes. Pour être vendus aux clients, ces pains doivent peser au moins 385~grammes. Un pain dont la masse est strictement inférieure à 385~grammes est un pain non-commercialisable, un pain dont la masse est supérieure ou égale à 385~grammes est commercialisable.
 
La masse d'un pain fabriqué par la machine peut être modélisée par une variable aléatoire $X$ suivant la loi normale\index{loi normale} d'espérance $\mu = 400$ et d'écart-type $\sigma = 11$.

\medskip
 
\emph{Les probabilités seront arrondies au millième le plus proche}

\medskip
 
\textbf{Partie A}

\medskip
 
\emph{On pourra utiliser le tableau suivant dans lequel les valeurs sont arrondies au millième le plus proche.}

\medskip 

\begin{tabularx}{\linewidth}{|c|*{9}{>{\centering \arraybackslash}X|}}\hline
$x$&380&385&390&395&400&405&410&415&420\\ \hline
$P(X \leqslant x)$&0,035&0,086&0,182&0,325&0,5&0,675&0,818&0,914&0,965\\ \hline
\end{tabularx}

\medskip

\begin{enumerate}
\item Calculer $P(390 \leqslant X \leqslant 410)$. 
\item Calculer la probabilité\index{probabilité} $p$ qu'un pain choisi au hasard dans la production soit commercialisable. 
\item Le fabricant trouve cette probabilité $p$ trop faible. Il décide de modifier ses méthodes de production afin de faire varier la valeur de $\sigma$ sans modifier celle de $\mu$.
 
Pour quelle valeur de $\sigma$ la probabilité qu'un pain soit commercialisable est-elle égale à 96\,\% ? On arrondira le résultat au dixième.
 
On pourra utiliser le résultat suivant : lorsque $Z$ est une variable aléatoire qui suit la loi normale\index{loi normale} d'espérance $0$ et d'écart-type 1, on a $P(Z \leqslant  -1,751) \approx 0,040$.
\end{enumerate}

\bigskip
 
\textbf{Partie B}

\medskip
 
Les méthodes de production ont été modifiées dans le but d'obtenir 96\,\% de pains commercialisables. 

Afin d'évaluer l'efficacité de ces modifications, on effectue un contrôle qualité sur un échantillon de 300~pains fabriqués.

\medskip
 
\begin{enumerate}
\item Déterminer l'intervalle de fluctuation asymptotique au seuil de 95\,\% de la proportion de pains commercialisables dans un échantillon de taille $300$. 
\item Parmi les $300$~pains de l'échantillon, $283$ sont commercialisables. 

Au regard de l'intervalle de fluctuation\index{intervalle de fluctuation} obtenu à la question 1, peut-on décider que l'objectif a été atteint ? 
\end{enumerate}

\bigskip
 
\textbf{Partie C}

\medskip 

Le boulanger utilise une balance électronique. Le temps de fonctionnement sans dérèglement, en jours, de cette balance électronique est une variable aléatoire $T$ qui suit une loi exponentielle\index{loi exponentielle} de paramètre $\lambda$.

\medskip
 
\begin{enumerate}
\item On sait que la probabilité que la balance électronique ne se dérègle pas avant 30~jours est de $0,913$. En déduire la valeur de $\lambda$ arrondie au millième.

\medskip
 
Dans toute la suite on prendra $\lambda = 0,003$.

\medskip
 
\item Quelle est la probabilité que la balance électronique fonctionne encore sans dérèglement après 90~jours, sachant qu'elle a fonctionné sans dérèglement 60~jours ? 
\item Le vendeur de cette balance électronique a assuré au boulanger qu'il y avait une chance sur deux pour que la balance ne se dérègle pas avant un an. A-t-il raison ? Si non, pour combien de jours est-ce vrai ? 
\end{enumerate}

\vspace{0,5cm}

\textbf{Exercice 4 \hfill  5 points}

\textbf{Commun à tous les candidats}

\medskip

Soit $f$ la fonction définie sur l'intervalle $]0~;~+ \infty[$ par 

\[f(x) = \dfrac{1 + \ln (x)}{x^2}\]

et soit $\mathcal{C}$ la courbe représentative de la fonction $f$ dans un repère du plan. La courbe $\mathcal{C}$ est donnée ci-dessous : 

\begin{center}
\psset{xunit=3.5cm,yunit=2.5cm}
\begin{pspicture*}(-0.2,-1.5)(3.5,1.4)
\psaxes[linewidth=1.5pt]{->}(0,0)(-0.2,-1.5)(3.5,1.4)
\psgrid[gridlabels=0pt,subgriddiv=1,gridcolor=orange](-1,-1)(4,2)
\psplot[plotpoints=8000,linewidth=1.25pt,linecolor=blue]{0.25}{3.5}{x ln 1 add x dup mul div}
\uput[u](1.2,0.85){$\mathcal{C}$}\uput[dl](0,0){O}
\end{pspicture*}
\end{center}
 
\begin{enumerate}
\item 
	\begin{enumerate}
		\item Étudier la limite de $f$ en $0$. 
		\item Que vaut $\displaystyle\lim_{x \to + \infty} \dfrac{\ln (x)}{x}$ ? En déduire la limite de la fonction $f$ en $+ \infty$. 
		\item En déduire les asymptotes éventuelles à la courbe $\mathcal{C}$.
	\end{enumerate} 
\item 
	\begin{enumerate}
		\item On note $f'$ la fonction dérivée de la fonction $f$ sur l'intervalle $]0~;~+ \infty[$.  

Démontrer que, pour tout réel $x$ appartenant à l'intervalle $]0~;~+ \infty[$, 

\[f'(x) = \dfrac{- 1 - 2\ln (x)}{x^3}.\]
 
		\item Résoudre sur l'intervalle $]0~;~+ \infty[$ l'inéquation $-1 - 2\ln (x) > 0$.
		 
En déduire le signe de $f'(x)$ sur l'intervalle $]0~;~+ \infty[$. 
		\item Dresser le tableau des variations de la fonction $f$.
	\end{enumerate} 
\item 
	\begin{enumerate}
		\item Démontrer que la courbe $\mathcal{C}$ a un unique point d'intersection avec l'axe des abscisses, dont on précisera les coordonnées. 
		\item En déduire le signe de $f(x)$ sur l'intervalle $]0~;~+ \infty[$.
	\end{enumerate} 
\item Pour tout entier $n \geqslant 1$, on note $I_{n}$ l'aire, exprimée en unités d'aires, du domaine délimité par l'axe des abscisses, la courbe $\mathcal{C}$ et les droites d'équations respectives $x = \dfrac{1}{\text{e}}$ et $x = n$. 
	\begin{enumerate}
		\item Démontrer que $0 \leqslant  I_{2} \leqslant \text{e} - \dfrac{1}{2}$.
		 
On admet que la fonction $F$, définie sur l'intervalle $]0~;~+ \infty[$ par $F(x) = \dfrac{- 2 - \ln (x)}{x}$,est une primitive de la fonction $f$ sur l'intervalle $]0~;~+ \infty[$. 
		\item Calculer $I_{n}$ en fonction de $n$. 
		\item Étudier la limite de $I_{n}$ en $+ \infty$. Interpréter graphiquement le résultat obtenu.
	\end{enumerate} 
\end{enumerate}
%%%%%%%%%%%%   fin Amérique du Nord mai 2013
\newpage
%%%%%%%%%%%%   Liban mai 2013
\hypertarget{Liban}{}

\lfoot{\small{Liban}}
\rfoot{\small{28 mai 2013}}
\renewcommand \footrulewidth{.2pt}
\pagestyle{fancy}
\thispagestyle{empty}

\begin{center}{\Large\textbf{\decofourleft~Baccalauréat S  Liban 28 mai 2013 \decofourright}}
\end{center}

\vspace{0,5cm}

\textbf{\textsc{Exercice 1} \hfill 4 points}

\textbf{Commun à tous les candidats}
 
\medskip

\emph{Cet exercice est un questionnaire à choix multiples. Aucune justification n'est demandée. Pour chacune des questions, une seule des propositions est correcte.\\ 
Chaque réponse correcte rapporte un point. Une réponse erronée ou une absence de réponse n'ôte pas de point. On notera sur la copie le numéro de la question, suivi de la lettre correspondant à la proposition choisie.}

\medskip
 
L'espace est rapporté à un repère orthonormé \Oijk. 

Les points A, B, C et D ont pour coordonnées respectives A$(1~;~-1~;~2)$, B$(3~;~3~;~8)$, C$(-3~;~5~;~4)$ et D(1~;~2~;~3). 
 
On note $\mathcal{D}$ la droite ayant pour représentation paramétrique 

$\left\{\begin{array}{l c l}
x&=&t + 1\\
y &=& 2t - 1\\   
z &=& 3t+2
\end{array}\right., t \in \R$ 
 
et $\mathcal{D}'$ la droite ayant pour représentation paramétrique $\left\{\begin{array}{l c l}
x&=& k + 1\\
y &=& k + 3\\
z &=&-k + 4
\end{array}\right.,  k \in \R$.
  
On note $\mathcal{P}$ le plan d'équation $x + y - z + 2 = 0$. 

\medskip

\textbf{Question 1 :}
 
Proposition \textbf{a.~} Les droites $\mathcal{D}$ et $\mathcal{D}'$ sont parallèles. 

Proposition \textbf{b.~} Les droites $\mathcal{D}$ et $\mathcal{D}'$ sont coplanaires.
 
Proposition \textbf{c.~} Le point C appartient à la droite $\mathcal{D}$.

Proposition \textbf{d.~} Les droites $\mathcal{D}$ et $\mathcal{D}'$ sont orthogonales. 

\medskip

\textbf{Question 2 :}
 
Proposition \textbf{a.~} Le plan $\mathcal{P}$ contient la droite $\mathcal{D}$ et est parallèle à la droite $\mathcal{D}'$.
 
Proposition \textbf{b.~} Le plan $\mathcal{P}$ contient la droite $\mathcal{D}'$ et est parallèle à la droite $\mathcal{D}$.
 
Proposition \textbf{c.~} Le plan $\mathcal{P}$ contient la droite $\mathcal{D}$ et est orthogonal à la droite $\mathcal{D}'$.
 
Proposition \textbf{d.~} Le plan $\mathcal{P}$ contient les droites $\mathcal{D}$ et $\mathcal{D}'$.

\medskip

\textbf{Question 3 :}
 
Proposition \textbf{a.~} Les points A, D et C sont alignés.
 
Proposition \textbf{b.~} Le triangle ABC est rectangle en A.
 
Proposition \textbf{c.~} Le triangle ABC est équilatéral.
 
Proposition \textbf{d.~} Le point D est le milieu du segment [AB]. 
 

\medskip

\textbf{Question 4 :}
 
On note $\mathcal{P}'$ le plan contenant la droite $\mathcal{D}'$ et le point A. Un vecteur normal à ce plan est :
 
Proposition \textbf{a.~} $\vect{n}(-1~;~5~;~4)$
 
Proposition \textbf{b.~} $\vect{n}(3~;~-1~;~2)$
 
Proposition \textbf{c.~} $\vect{n}(1~;~2~;~3)$
 
Proposition \textbf{d.~} $\vect{n}(1~;~1~;~-1)$


\vspace{0,5cm}

\textbf{\textsc{Exercice 2} \hfill 5 points}

\textbf{Commun à tous les candidats}
 
\medskip

L'entreprise \emph{Fructidoux} fabrique des compotes qu'elle conditionne en petits pots de 50~grammes. Elle souhaite leur attribuer la dénomination \og compote allégée \fg. 

La législation impose alors que la teneur en sucre, c'est-à-dire la proportion de sucre dans la compote, soit comprise entre 0,16 et 0,18. On dit dans ce cas que le petit pot de compote est conforme.
 
L'entreprise possède deux chaînes de fabrication F$_{1}$ et F$_{2}$.

\medskip
 
\emph{Les parties A et B peuvent être traitées indépendamment}

\medskip
 
\textbf{Partie A}

\medskip
 
La chaîne de production F$_{2}$ semble plus fiable que la chaîne de production F$_{1}$. Elle est cependant moins rapide.
 
Ainsi, dans la production totale, 70\,\% des petits pots proviennent de la chaîne F$_{1}$ et 30\,\% de la chaîne F$_{2}$.
 
La chaîne F$_{1}$ produit 5\,\% de compotes non conformes et la chaîne F$_{2}$ en produit 1\,\%.
 
On prélève au hasard un petit pot dans la production totale. On considère les évènements :
 
$E$ : \og Le petit pot provient de la chaîne F$_{2}$ \fg 

$C$ : \og Le petit pot est conforme. \fg

\medskip
 
\begin{enumerate}
\item Construire un arbre\index{arbre de probabilités} pondéré sur lequel on indiquera les données qui précèdent. 
\item Calculer la probabilité de l'évènement : \og Le petit pot est conforme et provient de la chaîne de production F$_{1}$. \fg 
\item Déterminer la probabilité de l'évènement $C$. 
\item Déterminer, à $10^{-3}$ près, la probabilité de l'évènement $E$ sachant que l'évènement $C$ est réalisé.
\end{enumerate}

\bigskip
 
\textbf{Partie B}

\medskip
 
\begin{enumerate}
\item 

On note $X$ la variable aléatoire qui, à un petit pot pris au hasard dans la production de la chaîne F$_{1}$, associe sa teneur en sucre.
 
On suppose que $X$ suit la loi normale\index{loi normale} d'espérance $m_{1} = 0,17$ et d'écart-type $\sigma_{1} = 0,006$. 

Dans la suite, on pourra utiliser le tableau ci-dessous.
\begin{center}
\begin{tabularx}{0.7\linewidth}{|*{3}{>{\centering \arraybackslash}X|}}\hline 
$\alpha$& $\beta$&$P(\alpha \leqslant X \leqslant \beta)$\\ \hline  
0,13 &0,15 &\np{0,0004}\\ \hline 
0,14 &0,16 &\np{0,0478}\\ \hline 
0,15 &0,17 &\np{0,4996} \\ \hline
0,16 &0,18 &\np{0,9044}\\ \hline 
0,17 &0,19 &\np{0,4996}\\ \hline 
0,18 &0,20 &\np{0,0478}\\ \hline 
0,19 &0,21 &\np{0,0004} \\ \hline
\end{tabularx}
\end{center}
 
Donner une valeur approchée à $10^{-4}$ près de la probabilité qu'un petit pot prélevé au hasard dans la production de la chaîne F$_{1}$ soit conforme. 
\item On note $Y$ la variable aléatoire qui, à un petit pot pris au hasard dans la production de la chaîne F$_{2}$, associe sa teneur en sucre.
 
On suppose que $Y$ suit la loi normale\index{loi normale} d'espérance $m_{2} = 0,17$ et d'écart-type $\sigma_{2}$.
 
On suppose de plus que la probabilité qu'un petit pot prélevé au hasard dans la production de la chaîne F$_{2}$ soit conforme est égale à $0,99$.
 
Soit Z la variable aléatoire définie par $Z = \dfrac{Y - m_{2}}{\sigma_{2}}$. 
	\begin{enumerate}
		\item Quelle loi la variable aléatoire $Z$ suit-elle ? 
		\item Déterminer, en fonction de $\sigma_{2}$ l'intervalle auquel appartient $Z$ lorsque $Y$ appartient à l'intervalle [0,16~;~0,18]. 
		\item En déduire une valeur approchée à $10^{-3}$ près de $\sigma_{2}$.
		 
On pourra utiliser le tableau donné ci-dessous, dans lequel la variable aléatoire $Z$ suit la loi normale\index{loi normale} d'espérance $0$ et d'écart-type $1$.
	\end{enumerate}
\begin{center}
\begin{tabularx}{0.5\linewidth}{|*{2}{>{\centering \arraybackslash}X|}}\hline 
$\beta$&$P(- \beta  \leqslant Z \leqslant \beta)$\\ \hline 
\np{2,4324} &0,985\\ \hline 
\np{2,4573} &0,986\\ \hline 
\np{2,4838} &0,987\\ \hline 
\np{2,5121} &0,988\\ \hline 
\np{2,5427} &0,989\\ \hline 
\np{2,5758} &0,990\\ \hline 
\np{2,6121} &0,991\\ \hline 
\np{2,6521} &0,992\\ \hline
\np{2,6968} &0,993\\ \hline
\end{tabularx}
\end{center} 

\end{enumerate} 

\vspace{0,5cm}

\textbf{\textsc{Exercice 3} \hfill 6 points}

\textbf{Commun à tous les candidats}
 
\medskip
 
Étant donné un nombre réel $k$, on considère la fonction $f_{k}$ définie sur $\R$ par 

\[f_{k}(x) = \dfrac{1}{1 + \text{e}^{- kx}}.\]
 
Le plan est muni d'un repère orthonormé \Oij.

\bigskip
 
\textbf{Partie A}

\medskip

Dans cette partie on choisit $k = 1$. On a donc, pour tout réel $x,f_{1}(x) = \dfrac{1}{1 + \text{e}^{- x}}$.
 
La représentation graphique $\mathcal{C}_{1}$ de la fonction $f_{1}$ dans le repère \Oij{} est donnée en ANNEXE, à rendre avec la copie.

\medskip
 
\begin{enumerate}
\item Déterminer les limites de $f_{1}(x)$ en $+ \infty$ et en $- \infty$ et interpréter graphiquement les résultats obtenus. 
\item Démontrer que, pour tout réel $x, f_{1}(x) = \dfrac{\text{e}^{x}}{1 +  \text{e}^{x}}$. 
\item On appelle $f'_{1}$ la fonction dérivée de $f_{1}$ sur $\R$. Calculer, pour tout réel $x,\: f'_{1}(x)$. 

En déduire les variations de la fonction $f_{1}$ sur $\R$. 
\item On définit le nombre $I = \displaystyle\int_{0}^1  f_{1}(x)\:\text{d}x$. 

Montrer que $I = \ln \left(\dfrac{1 + \text{e}}{2}\right)$. Donner une interprétation graphique de $I$.
\end{enumerate}
 
\bigskip
 
\textbf{Partie B}

\medskip
 
Dans cette partie, on choisit $k = - 1$ et on souhaite tracer la courbe $\mathcal{C}_{- 1}$ représentant la fonction $f_{- 1}$. 

Pour tout réel $x$, on appelle $P$ le point de $\mathcal{C}_{1}$ d'abscisse $x$ et $M $ le point de $\mathcal{C}_{- 1}$ d'abscisse $x$.
 
On note $K$ le milieu du segment $[MP]$.

\medskip
 
\begin{enumerate}
\item Montrer que, pour tout réel $x,\: f_{1}(x) + f_{- 1}(x) = 1$. 
\item En déduire que le point $K$ appartient à la droite d'équation $y = \dfrac{1}{2}$. 
\item Tracer la courbe $\mathcal{C}_{- 1}$ sur l'ANNEXE, à rendre avec la copie. 
\item En déduire l'aire, en unités d'aire, du domaine délimité par les courbes $\mathcal{C}_{1}$, $\mathcal{C}_{- 1}$ l'axe des ordonnées et la droite d'équation $x = 1$. 
\end{enumerate}

\bigskip
 
\textbf{Partie C}

\medskip
 
Dans cette partie, on ne privilégie pas de valeur particulière du paramètre $k$.
 
Pour chacune des affirmations suivantes, dire si elle est vraie ou fausse et justifier la réponse.

\medskip
 
\begin{enumerate}
\item Quelle que soit la valeur du nombre réel $k$, la représentation graphique de la fonction $f_{k}$ est strictement comprise entre les droites d'équations $y = 0$ et $y = 1$. 
\item Quelle que soit la valeur du réel $k$, la fonction $f_{k}$ est strictement croissante. 
\item Pour tout réel $k \geqslant 10,\: f_{k}\left(\dfrac{1}{2}\right) \geqslant  0,99$. 
\end{enumerate} 

\vspace{0,5cm}

\textbf{\textsc{Exercice 4} \hfill 5 points}

\textbf{Candidats N'AYANT PAS SUIVI l'enseignement de spécialité}
 
\medskip


On considère la suite\index{suite} numérique $\left(v_{n}\right)$ définie pour tout entier naturel $n$ par 

$\left\{\begin{array}{l c l}
v_{0} &=& 1\\ 	 
v_{n + 1}&=& \dfrac{9}{6 - v_{n}}
\end{array}\right.$ 

\bigskip

\textbf{Partie A}

\medskip
 
\begin{enumerate}
\item On souhaite écrire un algorithme \index{algorithme} affichant, pour un entier naturel $n$ donné, tous les termes de la suite, du rang $0$ au rang $n$.
 
Parmi les trois algorithmes suivants, un seul convient. Préciser lequel en justifiant la réponse.

\medskip

\hspace{-1cm} 
\begin{small}
\begin{tabularx}{1.1\linewidth}{|X|c|X|c|X|}\cline{1-1}\cline{3-3}\cline{5-5}
\multicolumn{1}{|c|}{\textbf{Algorithme \No 1}}&&\multicolumn{1}{|c|}{\textbf{Algorithme \No 2}}&&\multicolumn{1}{|c|}{\textbf{Algorithme \No 3}}\\ \cline{1-1}\cline{3-3}\cline{5-5}
\textbf{Variables :}&&\textbf{Variables :}&&\textbf{Variables :}\\
$v$ est un réel&&$v$ est un réel&&$v$ est un réel\\ 
$i$ et $n$ sont des entiers naturels&&$i$ et $n$ sont des entiers naturels&&$i$ et $n$ sont des entiers naturels\\
~&&&&\\ 
\textbf{Début de l'algorithme :}&&\textbf{Début de l'algorithme :}&& \textbf{Début de l'algorithme :}\\ 
Lire $n$&&Lire $n$&&Lire $n$\\ 
$v$ prend la valeur $1$&&Pour $i$ variant de $1$ à $n$ faire&&$v$ prend la valeur $1$\\ 
Pour $i$ variant de $1$ à $n$ faire&&$v$ prend la valeur $1$&& Pour $i$ variant de $1$ à $n$ faire\\ 
\hspace{0.2cm}$v$ prend la valeur $\dfrac{9}{6 - v}$&&\hspace{0.2cm}Afficher $v$&& \hspace{0.2cm}Afficher $v$\\  
Fin pour&&$v$ prend la valeur $\dfrac{9}{6 - v}$&&$v$ prend la valeur $\dfrac{9}{6 - v}$\\
Afficher $v$&&Fin pour&&Fin pour\\
&&&&Afficher $v$\\
\textbf{Fin algorithme}&&\textbf{Fin algorithme}&&\textbf{Fin algorithme}\\  \cline{1-1}\cline{3-3}\cline{5-5}
\end{tabularx}
\end{small}

\item Pour $n = 10$ on obtient l'affichage suivant :

\medskip
\begin{tabularx}{\linewidth}{|*{10}{>{\centering \arraybackslash}X|}}\hline
1&1,800&2,143&2,333&2,455&2,538&2,600&2,647&2,684&2,714\\ \hline
\end{tabularx}
\medskip

Pour $n = 100$, les derniers termes affichés sont :

\medskip
\begin{tabularx}{\linewidth}{|*{10}{>{\centering \arraybackslash}X|}}\hline
2,967&2,968&2,968&2,968&2,969&2,969&2,969&2,970&2,970&2,970\\ \hline
\end{tabularx}
\medskip
 
Quelles conjectures peut-on émettre concernant la suite $\left(v_{n}\right)$ ? 
\item 
	\begin{enumerate}
		\item Démontrer par récurrence que, pour tout entier naturel $n,\: 0 < v_{n} < 3$.  
		\item Démontrer que, pour tout entier naturel $n,\: v_{n+1} - v_{n} = \dfrac{\left(3 - v_{n} \right)^2}{6 - v_{n}}$. 
 
La suite $\left(v_{n}\right)$ est-elle monotone ? 
		\item Démontrer que la suite $\left(v_{n}\right)$ est convergente.
	\end{enumerate}
\end{enumerate}
 
\bigskip

\textbf{Partie B Recherche de la limite de la suite } \boldmath $\left(v_{n}\right)$ \unboldmath

\medskip

On considère la suite\index{suite} $\left(w_{n}\right)$ définie pour tout $n$ entier naturel par 

\[w_{n} = \dfrac{1}{v_{n} - 3}.\]
 
\begin{enumerate}
\item Démontrer que $\left(w_{n}\right)$ est une suite arithmétique de raison $- \dfrac{1}{3}$ 
\item En déduire l'expression de $\left(w_{n}\right)$, puis celle de $\left(v_{n}\right)$ en fonction de $n$. 
\item Déterminer la limite de la suite $\left(v_{n}\right)$. 
\end{enumerate}

\vspace{0,5cm}

\textbf{\textsc{Exercice 4} \hfill 5 points}

\textbf{Candidats AYANT SUIVI l'enseignement de spécialité}
 
\medskip

On considère la suite\index{suite} $\left(u_{n}\right)$ définie par $u_{0} = 3,\: u_{1} = 8$ et, pour tout $n$ supérieur ou égal à 0 :

\[ u_{n + 2} = 5u_{n+1} - 6u_{n}.\]
 
\begin{enumerate}
\item Calculer $u_{2}$ et $u_{3}$. 
\item Pour tout entier naturel $n \geqslant  2$, on souhaite calculer $u_{n}$ à l'aide de l'algorithme\index{algorithme} suivant :

\begin{center}
\begin{tabular}{p{2cm} p{10cm}}
\textbf{Variables :}		&$a, b$ et $c$ sont des nombres réels\\ 
							&$i$ et $n$ sont des nombres entiers naturels supérieurs ou égaux à 2\\ 
\textbf{Initialisation :}	&$a$ prend la valeur 3\\
							&$b$ prend la valeur 8\\ 
\textbf{Traitement :}		&Saisir $n$\\ 
&Pour $i$ variant de $2$ à $n$ faire\\
&\hspace{0.4cm}\begin{tabular}{|l}
$c$ prend la valeur $a$\\
$a$ prend la valeur $b$\\
$b$ prend la valeur \ldots\\
\end{tabular}\\ 
							&Fin Pour\\ 
 \textbf{Sortie :}			&Afficher b\\
\end{tabular}

\end{center}
 
	\begin{enumerate}
		\item Recopier la ligne de cet algorithme comportant des pointillés et les compléter. 
		
On obtient avec cet algorithme le tableau de valeurs suivant:

\medskip

\hspace{-1cm}\begin{tabularx}{1.1\linewidth}{|c|c|c|c|*{5}{>{\scriptsize \centering \arraybackslash}X|}c|} \hline
$n$& 7 &8 &9 &10 &11 &12 &13 &14 &15\\ \hline 
$u_{n}$&\scriptsize\np{4502} &\scriptsize\np{13378} &\scriptsize\np{39878} &\np{119122} &\np{356342} &\np{1066978} &\np{3196838} &\np{9582322} &\scriptsize\np{28730582}\\ \hline
\end{tabularx}

\medskip
 
\item Quelle conjecture peut-on émettre concernant la monotonie de la suite 
$\left(u_{n}\right)$ ?
	\end{enumerate} 
\item Pour tout entier naturel $n$, on note $C_{n}$ la matrice colonne $\begin{pmatrix}u_{n+1}\\u_{n}\end{pmatrix}$.
 
On note $A$ la matrice carrée d'ordre 2 telle que, pour tout entier naturel $n$,

$C_{n+1} = AC_{n}$. 

Déterminer $A$ et prouver que, pour tout entier naturel $n,\: C_{n} = A^nC_{0}$.
\item Soient $P = \begin{pmatrix}2&3\\1&1 		\end{pmatrix},\: D = \begin{pmatrix}2&0\\0&3\end{pmatrix}$ et $Q = \begin{pmatrix}- 1&3\\1&- 2		\end{pmatrix}$. 

Calculer $QP$.
 
On admet que $A = PDQ$.
 
Démontrer par récurrence que, pour tout entier naturel non nul $n,\: A^n = PD^nQ$.
\item À l'aide des questions précédentes, on peut établir le résultat suivant, que l'on admet.
 
Pour tout entier naturel non nul $n$, 

\[A^n = \begin{pmatrix}- 2^{n+1} +3^{n+1}& 3\times 2^{n+1} - 2\times 3^{n+1}\\ 
- 2^n +3^n& 	3 \times 2^n - 2 \times  3^n \end{pmatrix}.\]
	 
En déduire une expression de $u_{n}$ en fonction de $n$.

La suite $\left(u_{n}\right)$ a-t-elle une limite ? 
\end{enumerate}

\newpage
\begin{center}

\textbf{ANNEXE de l'EXERCICE 3, à rendre avec la copie}

\vspace{2cm}
 
Représentation graphique $\mathcal{C}_{1}$ de la fonction $f_{1}$

\vspace{2cm}

\psset{unit=1.7cm}
\begin{pspicture*}(-3.5,-2)(3.5,2)
\psaxes[linewidth=1.5pt]{->}(0,0)(-3.5,-2)(3.5,2)
\psaxes[linewidth=1.5pt]{->}(0,0)(1,1)
\psgrid[gridlabels=0pt,subgriddiv=1,gridcolor=orange](-4,-2)(4,2)
\uput[d](0.5,0){$\vect{\imath}$}
\uput[l](0,0.5){$\vect{\jmath}$}
\uput[dl](0,0){O}
\psplot[plotpoints=8000,linewidth=1.25pt,linecolor=blue]{-3.5}{3.5}{2.71828 x exp 2.71828 x exp 1 add div}
\rput(-1.5,0.35){$\mathcal{C}_{1}$}
\end{pspicture*} 
\end{center}
%%%%%%%%%%  fin Liban mai 2013
\newpage
%%%%%%%%%%  Polynésie juin 2013
\hypertarget{Polynesie}{}

\lfoot{\small{Polynésie}}
\rfoot{\small{7 juin 2013}}
\renewcommand \footrulewidth{.2pt}
\pagestyle{fancy}
\thispagestyle{empty} 

\begin{center} {\Large{\textbf{\decofourleft~Baccalauréat S  Polynésie~\decofourright\\7 juin 2013}}}
\end{center}

\vspace{0,5cm}

\textbf{\textsc{Exercice 1} \hfill 6 points}
 
\textbf{Commun  à tous les candidats}

\medskip

\medskip

On considère la fonction $f$ définie sur $\mathbb{R}$ par 

\[f(x) = (x + 2)\text{e}^{-x}.\]

 On note $\mathscr C$ la courbe représentative de la fonction $f$ dans un repère orthogonal.

\medskip

\begin{enumerate}
\item Étude de la fonction $f$.
	\begin{enumerate}
		\item Déterminer les coordonnées des points d'intersection de la courbe $\mathscr C$ avec les axes du repère.
		\item Étudier les limites de la fonction $f$ en $-\infty$ et en $+\infty$. En déduire les éventuelles asymptotes de la courbe $\mathscr C$.
		\item Étudier les variations de $f$ sur $\mathbb{R}$.
	\end{enumerate}
\item Calcul d'une valeur approchée de l'aire sous une courbe.

On note $\mathscr D$ le domaine compris entre l'axe des abscisses, la courbe $\mathscr C$ et les droites d'équation $x=0$ et $x=1$. On approche l'aire du domaine $\mathscr D$ en calculant une somme d'aires de rectangles.
	\begin{enumerate}
		\item Dans cette question, on découpe l'intervalle $[0~;~1]$ en quatre intervalles de même longueur :
    \begin{itemize}
    \item[$\bullet$] Sur l'intervalle $\left[0~;~\dfrac{1}{4} \right]$, on construit un rectangle de hauteur $f(0)$
    \item[$\bullet$] Sur l'intervalle $\left[\dfrac{1}{4}~;~\dfrac{1}{2} \right]$, on construit un rectangle de hauteur $f\left( \dfrac{1}{4} \right)$
    \item[$\bullet$] Sur l'intervalle $\left[\dfrac{1}{2}~;~\dfrac{3}{4} \right]$, on construit un rectangle de hauteur $f\left(\dfrac{1}{2} \right)$
    \item[$\bullet$] Sur l'intervalle $\left[ \dfrac{3}{4}~;~1 \right]$, on construit un rectangle de hauteur $f\left( \dfrac{3}{4} \right)$
    \end{itemize}
 Cette construction est illustrée ci-dessous.
\begin{center}
\psset{xunit=6cm,yunit=3cm}
\begin{pspicture}(-0.1,-0.2)(1.2,2.2)
    
\psframe*[linecolor=yellow](0,0)(0.25,2)
\psframe*[linecolor=yellow](0.25,0)(0.5,1.7523)
\psframe*[linecolor=yellow](0.50,0)(0.75,1.5163)
\psframe*[linecolor=yellow](0.75,0)(1,1.299)
\uput[d](0.125,1.85){$\mathscr{C}$}
\psframe(0,0)(0.25,2)
\psframe(0.25,0)(0.5,1.7523)
\psframe(0.50,0)(0.75,1.5163)
\psframe(0.75,0)(1,1.299)
\psplot[linewidth=1.25pt,linecolor=blue,plotpoints=8000]{0}{1}{x 2 add 2.71828 x exp div}
\psaxes[linewidth=1.5pt]{->}(0,0)(-0.1,-0.2)(1.2,2.2)
\uput[dl](0,0){O}    
\end{pspicture}    
%\begin{asy}
%import graph;
%usepackage("mathrsfs");
%unitsize(6cm, 3cm);
%
%real f(real x) {return (x+2)*exp(-x);}
%path cf=graph(f, 0,1);
%
%for(int i=0; i<4; ++i) {
%  real x=0.25*i;
%  path p=(x, 0)--(x+.25, 0)--( x+.25, f(x))--(x, f(x))--cycle;
%  filldraw(p, lightyellow, black);
%}
%
%draw(cf, linewidth(1pt));
%xaxis(-.1, 1.1, Ticks(NoZero, Step=1, step=0), Arrow);
%yaxis(-.1, 2.1, Ticks(NoZero, Step=1, step=0), Arrow);
%label("$O$", (0,0), SW);
%label("$\mathscr C$", (.2, f(.2)), SW);
%\end{asy}
      % \includegraphics{BacS-Polynesie-2013-aire}
      
    \end{center}
L'algorithme\index{algorithme} ci-dessous permet d'obtenir une valeur approchée de l'aire du domaine $\mathscr D$ en ajoutant les aires des quatre rectangles précédents :
    \begin{center}
      \begin{tabular}{|rl|}
        \hline
        Variables : & $k$ est un nombre entier \\
        & $S$ est un nombre réel \\
        Initialisation : & Affecter à $S$ la valeur 0 \\
        Traitement : & Pour $k$ variant de 0 à 3 \\
        & $\phantom{a}\left\vert \text{ Affecter à } S \text{ la valeur } S+\dfrac{1}{4} f\left( \dfrac{k}{4} \right)\right.$ \\
        & Fin Pour \\
        Sortie : & Afficher $S$ \\
        \hline
      \end{tabular}
    \end{center}
    Donner une valeur approchée à $10^{-3}$ près du résultat affiché par cet algorithme.
  		\item Dans cette question, $N$ est un nombre entier strictement supérieur à 1. On découpe l'intervalle $[0~;~1]$ en $N$ intervalles de même longueur. Sur chacun de ces intervalles, on construit un rectangle en procédant de la même manière qu'à la question 2.a.

Modifier l'algorithme précédent afin qu'il affiche en sortie la somme des aires des $N$ rectangles ainsi construits.
	\end{enumerate}
\item Calcul de la valeur exacte de l'aire sous une courbe.

Soit $g$ la fonction définie sur $\R$ par 
  
\[g(x)=(- x - 3) \text{e}^{-x}.\]
  
On admet que $g$ est une primitive de la fonction $f$ sur $\mathbb{R}$.
	\begin{enumerate}
		\item Calculer l'aire $\mathscr A$ du domaine $\mathscr D$, exprimée en unités d'aire.
		\item Donner une valeur approchée à $10^{-3}$ près de l'erreur commise en remplaçant $\mathscr A$ par la valeur approchée trouvée au moyen de l'algorithme de la question 2. a, c'est-à-dire l'écart entre ces deux valeurs.
	\end{enumerate}
\end{enumerate}

\vspace{0,5cm}

\textbf{Exercice 2 : \hfill 4 points} 

\emph{\bfseries Commun à tous les candidats}

\medskip

\emph{Cet exercice est un questionnaire à choix multiples. Aucune justification n'est demandée. Pour chacune des questions, une seule des quatre propositions est exacte. Chaque réponse correcte rapporte $1$ point. Une réponse erronée ou une absence de réponse n'ôte pas de point. Le candidat indiquera sur la copie le numéro de la question et la réponse choisie.}

\medskip

\begin{enumerate}
\item Soit $z_1 = \sqrt{6} \text{e}^{\text{i} \frac{\pi}{4}}$ et $z_2 = \sqrt{2} \text{e}^{-\text{i} \frac{\pi}{3}}$. La forme exponentielle de $\text{i} \dfrac{z_1}{z_2}$ est :

\begin{tabularx}{1.0\linewidth}{XXXX}
\textbf{a.} $\sqrt{3}\text{e}^{\text{i} \frac{19\pi}{12}}$ & 
\textbf{b.} $\sqrt{12} \text{e}^{-\text{i} \frac{\pi}{12}}$ & 
\textbf{c.} $\sqrt{3}\text{e}^{\text{i} \frac{7\pi}{12}}$ & 
\textbf{d.} $\sqrt{3}\text{e}^{\text{i} \frac{13\pi}{12}}$
\end{tabularx}

\item L'équation $- z = \overline z$, d'inconnue complexe $z$, admet :
	\begin{enumerate}
		\item une solution
		\item deux solutions
		\item une infinité de solutions dont les points images dans le plan complexe sont situés sur une droite.
		\item une infinité de solutions dont les points images dans le plan complexe sont situés sur un cercle.
	\end{enumerate}
\item Dans un repère de l'espace, on considère les trois points $A(1~;~2~;~3)$, $B(-1~;~5~;~4)$ et $C(-1~;~0~;~4)$. La droite parallèle à la droite $(AB)$ passant par le point $C$ a pour représentation paramétrique :

\begin{tabularx}{1.0\linewidth}{XX}
\textbf{a.} $\begin{cases}
      x = -2t-1 \\ y=3t \\ z=t+4
    \end{cases}, t\in \mathbb{R}$ & 
\textbf{b.} $\begin{cases}
      x=-1 \\ y=7t \\ z=7t+4
    \end{cases}, t \in \mathbb{R}$ \\
\textbf{c.} $\begin{cases}
      x=-1-2t \\ y=5+3t \\ z=4+t
    \end{cases}, t \in \mathbb{R}$ &
\textbf{d.} $\begin{cases}
      x=2t \\ y=-3t \\ z=-t
    \end{cases}, t \in \mathbb{R}$
\end{tabularx}

\item Dans un repère orthonormé de l'espace, on considère le plan $\mathscr P$ passant par le point $D(-1~;~2~;~3)$ et de vecteur normal $\vect{n}(3~;~-5~;~1)$, et la droite $\Delta$ de représentation paramétrique $\begin{cases}
x = t - 7 \\ y = t + 3 \\ z = 2t + 5
  \end{cases}, t \in \mathbb{R}$.
  	\begin{enumerate}
  		\item La droite $\Delta$ est perpendiculaire au plan $\mathscr P$.
  		\item La droite $\Delta$ est parallèle au plan $\mathscr P$ et n'a pas de point commun avec le plan $\mathscr P$.
  		\item La droite $\Delta$ et le plan $\mathscr P$ sont sécants.
  		\item La droite $\Delta$ est incluse dans le plan $\mathscr P$.
  	\end{enumerate}
\end{enumerate}

\newpage

\textbf{Exercice 3 : \hfill 5 points} 

\emph{\bfseries Commun à tous les candidats}

\medskip

\emph{Les $3$ parties peuvent être traitées de façon indépendante.}

\medskip

Thomas possède un lecteur MP3 sur lequel il a stocké plusieurs milliers de morceaux musicaux.

L'ensemble des morceaux musicaux qu'il possède se divise en trois genres distincts selon la répartition suivante :
\begin{center}
30~\% de musique classique, 45~\% de variété, le reste étant du jazz.
\end{center}

Thomas a utilisé deux qualités d'encodage pour stocker ses morceaux musicaux : un encodage de haute qualité et un encodage standard. On sait que : 
\begin{itemize}
\item[$\bullet$] les $\dfrac{5}{6}$ des morceaux de musique classique sont encodés en haute qualité.
\item[$\bullet$] les $\dfrac{5}{9}$ des morceaux de variété sont encodés en qualité standard.
\end{itemize}

\medskip

On considérera les évènements suivants :

\begin{description}
\item[$C$ :] \og Le morceau écouté est un morceau de musique classique \fg{} ;
\item[$V$ :] \og Le morceau écouté est un morceau de variété \fg{} ;
\item[$J$ :] \og Le morceau écouté est un morceau de jazz \fg{} ;
\item[$H$ :] \og Le morceau écouté est encodé en haute qualité \fg{} ;
\item[$S$ :] \og Le morceau écouté est encodé en qualité standard \fg.
\end{description}

\medskip

\textbf{Partie 1}

\medskip

Thomas décide d'écouter un morceau au hasard parmi tous les morceaux stockés sur son MP3 en utilisant la fonction \og lecture aléatoire \fg{}.

\emph{On pourra s'aider d'un arbre de probabilités\index{arbre de probabilités}.}

\begin{enumerate}
\item Quelle est la probabilité qu'il s'agisse d'un morceau de musique classique encodé en haute qualité ?
\item On sait que $P(H)=\dfrac{13}{20}$.
	\begin{enumerate}
		\item Les évènements $C$ et $H$ sont-ils indépendants ?
		\item Calculer $P(J \cap H)$ et $P_J(H)$.
	\end{enumerate}
\end{enumerate}

\medskip

\textbf{Partie 2}

\medskip

Pendant un long trajet en train, Thomas écoute, en utilisant la fonction \og lecture aléatoire \fg{} de son MP3, 60 morceaux de musique.

\begin{enumerate}
\item Déterminer l'intervalle de fluctuation asymptotique au seuil 95~\% de la proportion de morceaux de musique classique dans un échantillon de taille 60.
\item Thomas a comptabilisé qu'il avait écouté 12 morceaux de musique classique pendant son voyage. Peut-on penser que la fonction \og lecture aléatoire \fg{} du lecteur MP3 de Thomas est défectueuse ?
\end{enumerate}

\medskip

\textbf{Partie 3}

\medskip

On considère la variable aléatoire $X$ qui, à chaque chanson stocké sur le lecteur MP3, associe sa durée exprimée en secondes et on établit que $X$ suit la loi normale\index{loi normale} d'espérance $200$ et d'écart-type $20$.

\medskip

\emph{On pourra utiliser le tableau fourni en annexe dans lequel les valeurs sont arrondies au millième le plus proche.}

\medskip

On écoute un morceau musical au hasard.
\begin{enumerate}
\item Donner une valeur approchée à $10^{-3}$ près de $P(180 \leqslant X \leqslant 220)$.
\item Donner une valeur approchée à $10^{-3}$ près de la probabilité\index{probabilité} que le morceau écouté dure plus de 4 minutes.
\end{enumerate}

\newpage

\textbf{Exercice 4 :  \hfill 5 points}

\emph{\bfseries Candidats n'ayant pas suivis l'enseignement de spécialité mathématiques}

\medskip

On considère la suite\index{suite} $(u_n)$ définie par $u_0=\dfrac{1}{2}$ et telle que pour tout entier naturel~$n$, 

\[u_{n+1} = \dfrac{3u_n}{1+2u_n}\]

\begin{enumerate}
\item
  	\begin{enumerate}
  		\item Calculer $u_1$ et $u_2$.
  		\item Démontrer, par récurrence, que pour tout entier naturel $n$, $0 < u_n$.
  	\end{enumerate}
\item On admet que pour tout entier naturel $n$, $u_n<1$.
  	\begin{enumerate}
  		\item Démontrer que la suite $\left(u_n\right)$ est croissante.
  		\item Démontrer que la suite $\left(u_n\right)$ converge.
  	\end{enumerate}
\item Soit $\left(v_n\right)$ la suite définie, pour tout entier naturel $n$, par $v_n = \dfrac{u_n}{1 - u_n}$.
  	\begin{enumerate}
  		\item Montrer que la suite $(v_n)$ est une suite géométrique de raison 3.
  		\item Exprimer pour tout entier naturel $n$, $v_n$ en fonction de $n$.
  		\item En déduire que, pour tout entier naturel $n$, $u_n = \dfrac{3^n}{3^n+1}$.
  		\item Déterminer la limite de la suite $(u_n)$.
  	\end{enumerate}
\end{enumerate}

\newpage

\textbf{Exercice 4 :  \hfill 5 points}

\emph{\bfseries Candidats ayant suivis l'enseignement de spécialité mathématiques}

\medskip

Un opérateur téléphonique A souhaite prévoir l'évolution de nombre de ses abonnés dans une grande ville par rapport à son principal concurrent B à partir de 2013.

En 2013, les opérateurs A et B ont chacun $300$ milliers d'abonnés.

\medskip

Pour tout entier naturel $n$, on note $a_n$ le nombre d'abonnés, en milliers, de l'opérateur A la $n$-ième année après 2013, et $b_n$ le nombre d'abonnés, en milliers, de l'opérateur B la $n$-ième année après 2013.

Ainsi, $a_0 = 300$ et $b_0 = 300$.

\medskip

Des observations réalisées les années précédentes conduisent à modéliser la situation par la relation suivante :

pour tout entier naturel $n$, 
$\begin{cases}
a_{n+1} = 0,7a_n + 0,2b_n + 60 \\
b_{n+1} = 0,1a_n + 0,6b_n + 70
\end{cases}$.

\medskip

On considère les matrices\index{matrice} $M = 
\begin{pmatrix}
  0,7 & 0,2 \\ 0,1 & 0,6
\end{pmatrix}$ et $P =
\begin{pmatrix}
  60 \\ 70
\end{pmatrix}$.

Pour tout entier naturel $n$, on note $U_n =
\begin{pmatrix}
  a_n \\ b_n
\end{pmatrix}$.

\begin{enumerate}
\item
  	\begin{enumerate}
  		\item Déterminer $U_1$.
  		\item Vérifier que, pour tout entier naturel $n$, $U_{n+1} = M \times U_n +P$.
  	\end{enumerate}
\item On note $I$ la matrice $
  	\begin{pmatrix}
    1 & 0 \\ 0 & 1
  	\end{pmatrix}$.
  	\begin{enumerate}
  		\item Calculer $(I - M)\times
\begin{pmatrix}
4 & 2 \\ 1 & 3
\end{pmatrix}$.
  		\item En déduire que la matrice $I - M$ est inversible et préciser son inverse.
  		\item Déterminer la matrice $U$ telle que $U = M \times U + P$.
  	\end{enumerate}
\item Pour tout entier naturel, on pose $V_n = U_n - U$.
  	\begin{enumerate}
  		\item Justifier que, pour tout entier naturel $n$, $V_{n+1} = M \times V_n$.
  		\item En déduire que, pour tout entier naturel $n$, $V_n = M^n \times V_0$.
  	\end{enumerate}
\item On admet que, pour tout entier naturel $n$, 

\renewcommand\arraystretch{1.8}
\[V_n =  \begin{pmatrix}
    \dfrac{-100}{3}\times 0,8^n - \dfrac{140}{3} \times 0,5^n \\
    \dfrac{-50}{3} \times 0,8^n + \dfrac{140}{3} \times 0,5^n 
  \end{pmatrix}\]
\renewcommand\arraystretch{1}

  	\begin{enumerate}
  		\item Pour tout entier naturel $n$, exprimer $U_n$ en fonction de $n$ et en déduire la limite de la suite\index{suite} $(a_n)$.
  		\item Estimer le nombre d'abonnés de l'opérateur A à long terme.
  	\end{enumerate}
\end{enumerate}

\newpage

\begin{center}
  \bfseries \Large ANNEXE de l'exercice 3
\end{center}

$X$ est une variable aléatoire normale d'espérance 200 et d'écart-type 20.

\begin{center}
  \begin{tabularx}{.4\linewidth}{|*{2}{>{\centering \arraybackslash}X|}}\hline
$b$ & $P(X \leqslant b)$ \\\hline
140 & 0,001 \\\hline 
150 & 0,006 \\\hline
160 & 0,023 \\\hline
170 & 0,067 \\\hline
180 & 0,159 \\\hline
190 & 0,309 \\\hline
200 & 0,500 \\\hline
210 & 0,691 \\\hline
220 & 0,841 \\\hline
230 & 0,933 \\\hline
240 & 0,977 \\\hline
250 & 0,994 \\\hline
260 & 0,999 \\\hline
  \end{tabularx}
\end{center}
%%%%%%%%%%   fin Polynésie juin 2013
\newpage
%%%%%%%%%%   Antilles-Guyane juin 2013
\hypertarget{Antilles}{}

\lfoot{\small{Antilles-Guyane}}
\rfoot{\small{18 juin 2013}}
\renewcommand \footrulewidth{.2pt}
\pagestyle{fancy}
\thispagestyle{empty}
\begin{center}\textbf{Durée : 4 heures}

\vspace{0,25cm}

{\Large\textbf{Baccalauréat S Antilles-Guyane 18 juin 2013}}
\end{center}

\vspace{0,25cm}

\textbf{\textsc{Exercice 1} \hfill 5 points}
 
\textbf{Commun à tous les candidats}

\medskip

\begin{multicols}{2}
\textbf{Description de la figure dans l'espace muni du repère orthonormé $\left(A~;~\vect{AB}~;~\vect{AD}~;~\vect{AE}\right)$:}
\\
$ABCDEFGH$ désigne un cube de côté 1.

On appelle $\mathcal{P}$ le plan $(AFH)$.

Le point $I$ est le milieu du segment $[AE]$,

le point $J$ est le milieu du segment $[BC]$,

le point $K$ est le milieu du segment $[HF]$,

le point $L$ est le point d'intersection de la droite $(EC)$ et du plan $\mathcal{P}$.

\smallskip

\begin{center}

\psset{xunit=0.6cm,yunit=0.6cm,algebraic=true,dotstyle=o,dotsize=3pt 0,linewidth=0.8pt,arrowsize=3pt 2,arrowinset=0.25}
\begin{pspicture}(0,0)(8,7)
\pspolygon[linewidth=0pt,linecolor=white,hatchcolor=lightgray,fillstyle=hlines,hatchangle=75.0,hatchsep=0.1](1,1)(5.04,5.26)(3.32,6.58)
\psline(1,1)(5.04,1.26)
\psline(5.04,1.26)(7.36,2.84)
\psline[linestyle=dashed](7.36,2.84)(3.32,2.58)
\psline[linestyle=dashed](3.32,2.58)(1,1)
\psline(1,1)(1,5)
\psline(1,5)(5.04,5.26)
\psline(5.04,5.26)(7.36,6.84)
\psline(3.32,6.58)(7.36,6.84)
\psline(3.32,6.58)(1,5)
\psline[linestyle=dashed](3.32,2.58)(3.32,6.58)
\psline(7.36,2.84)(7.36,6.84)
\psline(5.04,5.26)(5.04,1.26)
\psline(5.04,5.26)(1,1)
\psline[linestyle=dashed](1,1)(3.32,6.58)
\psline(3.32,6.58)(5.04,5.26)
\psline[linestyle=dashed](1,5)(7.36,2.84)
\psline[linestyle=dashed](6.2,2.05)(1,3)
\begin{scriptsize}
\psdots[dotsize=1pt 0,dotstyle=*](1,1)
\rput[bl](0.72,0.66){$A$}
\psdots[dotsize=1pt 0,dotstyle=*](5.04,1.26)
\rput[bl](5.1,1){$B$}
\psdots[dotsize=1pt 0,dotstyle=*](3.32,2.58)
\rput[bl](3.4,2.7){$D$}
\psdots[dotsize=1pt 0,dotstyle=*](1,5)
\rput[bl](0.74,5.12){$E$}
\psdots[dotsize=1pt 0,dotstyle=*](7.36,2.84)
\rput[bl](7.54,2.7){$C$}
\psdots[dotsize=1pt 0,dotstyle=*](5.04,5.26)
\rput[bl](5.24,5){$F$}
\psdots[dotsize=1pt 0,dotstyle=*](3.32,6.58)
\rput[bl](3.12,6.8){$H$}
\psdots[dotsize=1pt 0,dotstyle=*](7.36,6.84)
\rput[bl](7.48,6.84){$G$}
\psdots[dotsize=1pt 0,dotstyle=*](1,3)
\rput[bl](0.7,3.02){$I$}
\psdots[dotsize=1pt 0,dotstyle=*](6.2,2.05)
\rput[bl](6.4,1.7){$J$}
\psdots[dotsize=1pt 0,dotstyle=*](4.18,5.92)
\rput[bl](4.26,5.96){$K$}
\psdots[dotsize=1pt 0,dotstyle=*](3.12,4.28)
\rput[bl](3.06,4.46){$L$}
\end{scriptsize}
\end{pspicture}

\end{center}

\end{multicols}

\medskip

\emph{Ceci est un questionnaire à choix multiples (QCM). Pour chacune des questions, une seule des quatre affirmations est exacte. Le candidat indiquera sur sa copie le numéro de la question et la lettre correspondant à la réponse choisie. Aucune justification n'est demandée. Une réponse exacte rapporte un point, une réponse fausse ou une absence de réponse ne rapporte aucun point.}

\medskip

\begin{enumerate}
\item
	\begin{enumerate}
		\item Les droites $(IJ)$ et $(EC)$ sont strictement parallèles.
		\item Les droites $(IJ)$ et $(EC)$ sont non coplanaires.
		\item Les droites $(IJ)$ et $(EC)$ sont sécantes.
		\item Les droites $(IJ)$ et $(EC)$ sont confondues.
	\end{enumerate}
\item 
	\begin{enumerate}
		\item Le produit scalaire $\vect{AF}\cdot\vect{BG}$ est égal à 0.
		\item Le produit scalaire $\vect{AF}\cdot\vect{BG}$ est égal à $(-1)$.
		\item Le produit scalaire $\vect{AF}\cdot\vect{BG}$ est égal à 1.
		\item Le produit scalaire $\vect{AF}\cdot\vect{BG}$ est égal à 2.
	\end{enumerate}
\item Dans le repère orthonormé $\left(A~;~\vect{AB}~;~\vect{AD}~;~\vect{AE}\right)$:
	\begin{enumerate}
		\item Le plan $\mathcal{P}$ a pour équation cartésienne : $x + y + z - 1=0$.
		\item Le plan $\mathcal{P}$ a pour équation cartésienne : $x- y + z=0$.
		\item Le plan $\mathcal{P}$ a pour équation cartésienne : $- x + y + z=0$.
		\item Le plan $\mathcal{P}$ a pour équation cartésienne : $x + y - z = 0$.
	\end{enumerate}
\item 
	\begin{enumerate}
		\item $\vect{EG}$ est un vecteur normal au plan $\mathcal{P}$.
		\item $\vect{EL}$ est un vecteur normal au plan $\mathcal{P}$.
		\item $\vect{IJ}$ est un vecteur normal au plan $\mathcal{P}$.
		\item $\vect{DI}$ est un vecteur normal au plan $\mathcal{P}$.
	\end{enumerate}
\item
	\begin{enumerate}
		\item $\vect{AL}=\frac12\vect{AH} + \frac12\vect{AF}$.
		\item $\vect{AL}=\frac13\vect{AK}$.
		\item $\vect{ID}=\frac12\vect{IJ}$.
		\item $\vect{AL}=\frac13\vect{AB}+\frac13\vect{AD}+\frac23\vect{AE}$.
	\end{enumerate}
\end{enumerate}

\vspace{0,5cm}

\textbf{\textsc{Exercice 2} \hfill 5 points}
 
\textbf{Commun à tous les candidats}

\medskip

\textbf{Partie A}

\smallskip

Soient $n$ un entier naturel, $p$ un nombre réel compris entre 0 et 1, et $X_n$ une variable aléatoire suivant une loi binomiale\index{loi binomiale} de paramètres $n$ et $p$. On note $F_n = \frac{X_n}{n}$ et $f$ une valeur prise par $F_n$. On rappelle que, pour $n$ assez grand, l'intervalle $\left[p-\frac{1}{\sqrt{n}}~;~p+\frac{1}{\sqrt{n}}\right]$ contient la fréquence $f$ avec une probabilité\index{probabilité} au moins égale à \np{0,95}.

\medskip

En déduire que l'intervalle $\left[f - \frac{1}{\sqrt{n}}~;~f + \frac{1}{\sqrt{n}}\right]$ contient $p$ avec une probabilité au moins égale à \np{0,95}.

\medskip

\textbf{Partie B}

\smallskip

On cherche à étudier le nombre d'étudiants connaissant la signification du sigle URSSAF. Pour cela, on les interroge en proposant un questionnaire à choix multiples. Chaque étudiant doit choisir parmi trois réponses possibles, notées $A$, $B$ et $C$, la bonne réponse étant la $A$.

On note $r$ la probabilité pour qu'un étudiant connaisse la bonne réponse. Tout étudiant connaissant la bonne réponse répond $A$, sinon il répond au hasard (de façon équiprobable).

\begin{enumerate}
\item On interroge un étudiant au hasard. On note:
	\begin{description}
		\item[$A$] l'évènement \og l'étudiant répond $A$\fg{},
		\item[$B$] l'évènement \og l'étudiant répond $B$\fg{},
		\item[$C$] l'évènement \og l'étudiant répond $C$\fg{},
		\item[$R$] l'évènement \og l'étudiant connait la réponse\fg{},
		\item[$\overline{R}$] l'évènement contraire de $R$.
	\end{description}
	\begin{enumerate}
		\item Traduire cette situation à l'aide d'un arbre de probabilité.
		\item Montrer que la probabilité de l'évènement $A$ est $P(A)=\frac13\left(1+2r\right)$.
		\item Exprimer en fonction de $r$ la probabilité qu'une personne ayant choisie $A$ connaisse la bonne réponse.
	\end{enumerate}
\item Pour estimer $r$, on interroge $400$ personnes et on note $X$ la variable aléatoire comptant le nombre de bonnes réponses. On admettra qu'interroger au hasard $400$ étudiants revient à effectuer un tirage avec remise de $400$ étudiants dans l'ensemble de tous les étudiants.
	\begin{enumerate}
		\item Donner la loi de $X$ et ses paramètres $n$ et $p$ en fonction de $r$.
		\item Dans un premier sondage, on constate que $240$ étudiants répondent $A$, parmi les $400$ interrogés.

Donner un intervalle de confiance au seuil de 95~\% de l'estimation de $p$.

En déduire un intervalle de confiance au seuil de 95~\% de $r$.
		\item Dans la suite, on suppose que $r = \np{0,4}$. Compte-tenu du grand nombre d'étudiants, on considérera que $X$ suit une loi normale\index{loi normale}.
			\begin{enumerate}
				\item Donner les paramètres de cette loi normale.
				\item Donner une valeur approchée de $P(X\leqslant 250)$ à $10^{-2}$ près.
				
On pourra s'aider de la table en annexe 1, qui donne une valeur approchée de $P(X\leqslant t)$ où $X$ est la variable aléatoire de la question \textbf{2. c}.
			\end{enumerate}
	\end{enumerate}
\end{enumerate}

\vspace{0,5cm}

\textbf{\textsc{Exercice 3} \hfill 5 points}
 
\textbf{Commun à tous les candidats}

\medskip

Dans tout ce qui suit, $m$ désigne un nombre réel quelconque.

\smallskip

\textbf{Partie A}

\smallskip

Soit $f$ la fonction définie et dérivable sur l'ensemble des nombres réels $\R$ telle que:

\[f(x) = (x + 1)\text{e}^x.\]

\begin{enumerate}
\item Calculer la limite de $f$ en $+ \infty$ et $- \infty$.
\item On note $f'$ la fonction dérivée de la fonction $f$ sur $\R$.\\
Démontrer que pour tout réel $x$, $f'(x) = (x + 2)\text{e}^x$.
\item Dresser le tableau de variation de $f$ sur $\R$.
\end{enumerate}

\smallskip

\textbf{Partie B}

\smallskip

On définie la fonction $g_m$ sur $\R$ par:

\[g_m(x) = x + 1 - m\text{e}^{-x}\]
et on note $\mathcal{C}_m$ la courbe de la fonction $g_m$ dans un repère \Oij du plan.

\begin{enumerate}
\item
	\begin{enumerate}
		\item Démontrer que $g_m(x) = 0$ si et seulement si $f(x)=m$.
		\item Déduire de la partie $A$, sans justification, le nombre de points d'intersection de la courbe $\mathcal{C}_m$ avec l'axe des abscisses en fonction du réel $m$.
	\end{enumerate}
\item On a représenté en annexe 2 les courbes $\mathcal{C}_0$, $\mathcal{C}_{\e}$, et $\mathcal{C}_{-\text{e}}$ (obtenues en prenant respectivement pour $m$ les valeurs 0, $\text{e}$ et $-\text{e}$).

Identifier chacune de ces courbes sur la figure de l'annexe en justifiant.
\item Étudier la position de la courbe $\mathcal{C}_m$ par rapport à la droite $\mathcal{D}$ d'équation 

$y = x + 1$ suivant les valeurs du réel $m$.
\item 
	\begin{enumerate}
		\item On appelle $D_2$ la partie du plan comprise entre les courbes $\mathcal{C}_{\text{e}}$, $\mathcal{C}_{-\text{e}}$, l'axe $(\text{O}y)$ et la droite $x = 2$. Hachurer $D_2$ sur l'annexe 2.
		\item Dans cette question, $a$ désigne un réel positif, $D_a$ la partie du plan comprise entre $\mathcal{C}_{\text{e}}$, $\mathcal{C}_{-\text{e}}$, l'axe $(\text{O}y)$ et la droite $\Delta_a$ d'équation $x=a$. On désigne par $\mathcal{A}(a)$ l'aire de cette partie du plan, exprimée en unités d'aire.
		
Démontrer que pour tout réel $a$ positif: $\mathcal{A}(a) = 2\text{e} - 2\mathbf{\text{e}}^{1 - a}$.

En déduire la limite de $\mathcal{A}(a)$ quand $a$ tend vers $+ \infty$.
	\end{enumerate}
\end{enumerate}

\vspace{0,5cm}

\textbf{\textsc{Exercice 4} \hfill 5 points}
 
\textbf{Commun ayant suivi l'enseignement de spécialité}

\medskip

On définit les suite\index{suite} $\left(u_n\right)$ et $\left(v_n\right)$ sur l'ensemble $\N$ des entiers naturels par:

\[
u_0=0~;~v_0=1~,~\text{et}~
\left\{
\begin{array}{rcl}
u_{n+1}&=&\dfrac{u_n+v_n}{2}\\
v_{n+1}&=&\dfrac{u_n+2v_n}{3}
\end{array}
\right.\]

Le but de cet exercice est d'étudier la convergence des suites $\left(u_n\right)$ et $\left(v_n\right)$.

\medskip

\begin{enumerate}
\item
Calculer $u_1$ et $v_1$.
\item On considère l'algorithme\index{algorithme} suivant:
\begin{center}
\fbox{
\begin{minipage}{6cm}
Variables : $u$, $v$ et $w$ des nombres réels\\
\phantom{Variables:} $N$ et $k$ des nombres entiers\\
Initialisation : $u$ prend la valeur 0\\
\phantom{Initialisation:} $v$ prend la valeur 1\\
Début de l'algorithme\\
Entrer la valeur de $N$\\
Pour $k$ variant de 1 à $N$\\
\phantom{xxxx} $w$ prend la valeur $u$\\
\phantom{xxxx} $u$ prend la valeur $\dfrac{w+v}{2}$\\
\phantom{xxxx} $v$ prend la valeur $\dfrac{w+2v}{3}$\\
Fin du Pour\\
Afficher $u$\\
Afficher $v$\\
Fin de l'algorithme
\end{minipage}
}
\end{center}
	\begin{enumerate}
		\item On exécute cet algorithme en saisissant $N = 2$. Recopier et compléter le tableau donné ci-dessous contenant l'état des variables au cours de l'exécution de l'algorithme.
\begin{center}
\begin{tabular}{|c|c|c|c|}
\hline
$k$ & $w$ & $u$ & $v$ \\
\hline
1& & & \\
\hline
2& & & \\
\hline
\end{tabular}
\end{center}
		\item Pour un nombre $N$ donné, à quoi correspondent les valeurs affichées par l'algorithme par rapport à la situation étudiée dans cet exercice~?
	\end{enumerate}
\item Pour tout entier naturel $n$ on définit le vecteur colonne $X_n$ par $X_n=\begin{pmatrix}
u_n\\v_n
\end{pmatrix}$ et la matrice\index{matrice} $A$ par $A=\begin{pmatrix}
\frac12&\frac12\\\frac13&\frac23
\end{pmatrix}$.
	\begin{enumerate}
		\item Vérifier que, pour tout entier naturel $n$, $X_{n+1}= AX_n$.
		\item Démontrer par récurrence que $X_n = A^nX_0$ pour tout entier naturel $n$.
	\end{enumerate}
\item On définit les matrices $P$, $P'$ et $B$ par $P = \begin{pmatrix}
\frac45&\frac65\\-\frac65&\frac65
\end{pmatrix}$, $P'=\begin{pmatrix}
\frac12&-\frac12\\\frac12&\frac13
\end{pmatrix}$ 
et $B=\begin{pmatrix}
1&0\\0&\frac16
\end{pmatrix}$.
	\begin{enumerate}
		\item
Calculer le produit $PP'$.\\
On admet que $P'BP=A$.\\
Démontrer par récurrence que pour tout entier naturel $n$, $A^n=P'B^nP$.
		\item On admet que pour tout entier naturel $n$, $B^n=\begin{pmatrix}
1&0\\0&\left(\frac16\right)^n
\end{pmatrix}$.\\
En déduire l'expression de la matrice $A^n$ en fonction de $n$.
	\end{enumerate}
\item 
	\begin{enumerate}
		\item Montrer que $X_n=\begin{pmatrix}
\frac35-\frac35\left(\frac16\right)^n\\
\frac35+\frac25\left(\frac16\right)^n
\end{pmatrix}$ pour tout entier naturel $n$.

En déduire les expressions de $u_n$ et $v_n$ en fonction de $n$.
		\item Déterminer alors les limites des suites $\left(u_n\right)$ et $\left(v_n\right)$.
	\end{enumerate}
\end{enumerate}

\vspace{0,5cm}

\textbf{\textsc{Exercice 4} \hfill 5 points}
 
\textbf{Commun n'ayant pas suivi l'enseignement de spécialité}

\medskip

On considère la suite $\left(z_n\right)$ à termes complexes définie par $z_0 = 1 + \text{i}$ et, pour tout entier naturel $n$, par

\[z_{n+1} = \dfrac{z_n + \left|z_n\right|}{3}.\]

Pour tout entier naturel $n$, on pose: $z_n = a_n + \text{i}b_n$, où $a_n$ est la partie réelle de $z_n$ et $b_n$ est la partie imaginaire de $z_n$.

Le but de cet exercice est d'étudier la convergence des suites $\left(a_n\right)$ et $\left(b_n\right)$.

\smallskip

\textbf{Partie A}

\smallskip

\begin{enumerate}
\item Donner $a_0$ et $b_0$.
\item Calculer $z_1$, puis en déduire que $a_1=\dfrac{1 + \sqrt{2}}{3}$ et $b_1 = \dfrac13$.
\item On considère l'algorithme\index{algorithme} suivant:
\begin{center}
\fbox{
\begin{minipage}{6cm}
Variables: $A$ et $B$ des nombres réels\\
\phantom{Variables:} $K$ et $N$ des nombres entiers\\
Initialisation: Affecter à $A$ la valeur 1\\
\phantom{Initialisation:} Affecter à $B$ la valeur 1\\
Traitement:\\
Entrer la valeur de N\\
Pour $K$ variant de 1 à $N$\\
\phantom{xxxx} Affecter à $A$ la valeur $\dfrac{A+\sqrt{A^2+B^2}}{3}$\\
\phantom{xxxx} Affecter à $B$ la valeur $\dfrac{B}{3}$\\
FinPour\\
Afficher A
\end{minipage}
}
\end{center}
	\begin{enumerate}
		\item On exécute cet algorithme en saisissant $N=2$. Recopier et compléter le tableau ci-dessous contenant l'état des variables au cours de l'exécution de l'algorithme (on arrondira les valeurs calculées à $10^{-4}$ près).
\begin{center}
\begin{tabularx}{0.4\linewidth}{|*{3}{>{\centering \arraybackslash}X|}}
\hline
$K$ & $A$ & $B$ \\\hline
1& & \\\hline
2& & \\\hline
\end{tabularx}
\end{center}
		\item Pour un nombre $N$ donné, à quoi correspond la valeur affichée par l'algorithme  par rapport à la situation étudiée dans cet exercice~?
	\end{enumerate}
\end{enumerate}

\smallskip

\textbf{Partie B}

\smallskip

\begin{enumerate}
\item
Pour tout entier naturel $n$, exprimer $z_{n+1}$ en fonction de $a_n$ et $b_n$.

En déduire l'expression de $a_{n+1}$ en fonction de $a_n$ et $b_n$, et l'expression de $b_{n+1}$ en fonction de $a_n$ et $b_n$.
\item Quelle est la nature de la suite $\left(b_n \right)$~? En déduire l'expression de $b_n$ en fonction de $n$, et déterminer  la limite de $\left(b_n \right)$.
\item
	\begin{enumerate}
		\item
On rappelle que pour tous nombres complexes $z$ et $z'$:

\[\left|z + z'\right|\leqslant |z| + \left|z'\right|\qquad\text{(inégalité triangulaire)}.\]

Montrer que pour tout entier naturel $n$,

\[\left|z_{n+1}\right|\leqslant\dfrac{2\left|z_n\right|}{3}.\]

		\item Pour tout entier naturel $n$, on pose $u_n= \left|z_n\right|$.
		
Montrer par récurrence que, pour tout entier naturel $n$,

\[u_n\leqslant \left(\frac23\right)^n\sqrt{2}.\]

En déduire que la suite\index{suite} $\left(u_n \right)$ converge vers une limite que l'on déterminera.
		\item Montrer que, pour tout entier naturel $n$, $\left|a_n\right|\leqslant u_n$. En déduire que la suite $\left(a_n \right)$ converge vers une limite que l'on déterminera.
	\end{enumerate}
\end{enumerate}
%: ANNEXES 
\newpage
\begin{center}
\textbf{Annexe 2}

\smallskip

\textbf{Exercice 3}

\smallskip

\textbf{À rendre avec la copie}
\end{center}

\medskip

\begin{center}

\psset{xunit=2.0cm,yunit=2.0cm}
\begin{pspicture*}(-2,-3)(5,6)
\psgrid[subgriddiv=0,gridlabels=0,gridcolor=lightgray](0,0)(-2,-3)(5,6)
\psset{xunit=2.0cm,yunit=2.0cm,algebraic=true,dotstyle=o,dotsize=3pt 0,linewidth=0.8pt,arrowsize=3pt 2,arrowinset=0.25}
\psaxes[labelFontSize=\scriptstyle,xAxis=true,yAxis=true,Dx=1,Dy=1,ticksize=-2pt 0,subticks=2]{->}(0,0)(-2,-3)(5,6)
\psplot[plotpoints=4000,linecolor=blue]{-2.0}{5.0}{x+1+2.718281828^(-x+1)}
\psplot{-2}{5}{(--1--1*x)/1}
\psplot[plotpoints=4000,linecolor=cyan]{-2.0}{5.0}{x+1-2.718281828^(-x+1)}
\rput[tl](-1.5,4.89){Courbe 1}
\rput[tl](-1.5,0.75){Courbe 2}
\rput[tl](0.2,-1.71){Courbe 3}
\end{pspicture*}
\end{center}
\newpage

\begin{center}
\textbf{
Annexe 2 }

\smallskip

\textbf{Exercice 3}

\smallskip

\textbf{À rendre avec la copie}
\end{center}


\begin{center}
\begin{tabular}{|c|c|c|c|c|c|c|c|c|c|c|c|}
\hline
\multicolumn{5}{|c|}{E12}&\multicolumn{7}{|c|}{=LOI.NORMALE(\$A12+E\$1;240;RACINE(96);VRAI)}\\
\hline
 & A & B & C & D & E & F & G & H & I & J & K \\ \hline
1 & t & 0 & 0,1 & 0,2 & 0,3 & 0,4 & 0,5 & 0,6 & 0,7 & 0,8 & 0,9 \\ \hline
2 & 235 & 0,305 & 0,309 & 0,312 & 0,316 & 0,319 & 0,323 & 0,327 & 0,330 & 0,334 & 0,338 \\ \hline
3 & 236 & 0,342 & 0,345 & 0,349 & 0,353 & 0,357 & 0,360 & 0,364 & 0,368 & 0,372 & 0,376 \\ \hline
4 & 237 & 0,380 & 0,384 & 0,388 & 0,391 & 0,395 & 0,399 & 0,403 & 0,407 & 0,411 & 0,415 \\ \hline
5 & 238 & 0,419 & 0,423 & 0,427 & 0,431 & 0,435 & 0,439 & 0,443 & 0,447 & 0,451 & 0,455 \\ \hline
6 & 239 & 0,459 & 0,463 & 0,467 & 0,472 & 0,476 & 0,480 & 0,484 & 0,488 & 0,492 & 0,496 \\ \hline
7 & 240 & 0,500 & 0,504 & 0,508 & 0,512 & 0,516 & 0,520 & 0,524 & 0,528 & 0,533 & 0,537 \\ \hline
8 & 241 & 0,541 & 0,545 & 0,549 & 0,553 & 0,557 & 0,561 & 0,565 & 0,569 & 0,573 & 0,577 \\ \hline
9 & 242 & 0,581 & 0,585 & 0,589 & 0,593 & 0,597 & 0,601 & 0,605 & 0,609 & 0,612 & 0,616 \\ \hline
10 & 243 & 0,620 & 0,624 & 0,628 & 0,632 & 0,636 & 0,640 & 0,643 & 0,647 & 0,651 & 0,655 \\ \hline
11 & 244 & 0,658 & 0,662 & 0,666 & 0,670 & 0,673 & 0,677 & 0,681 & 0,684 & 0,688 & 0,691 \\ \hline
12 & 245 & 0,695 & 0,699 & 0,702 & \fbox{\textbf{0,706}} & 0,709 & 0,713 & 0,716 & 0,720 & 0,723 & 0,726 \\ \hline
13 & 246 & 0,730 & 0,733 & 0,737 & 0,740 & 0,743 & 0,746 & 0,750 & 0,753 & 0,756 & 0,759 \\ \hline
14 & 247 & 0,763 & 0,766 & 0,769 & 0,772 & 0,775 & 0,778 & 0,781 & 0,784 & 0,787 & 0,790 \\ \hline
15 & 248 & 0,793 & 0,796 & 0,799 & 0,802 & 0,804 & 0,807 & 0,810 & 0,813 & 0,815 & 0,818 \\ \hline
16 & 249 & 0,821 & 0,823 & 0,826 & 0,829 & 0,831 & 0,834 & 0,836 & 0,839 & 0,841 & 0,844 \\ \hline
17 & 250 & 0,846 & 0,849 & 0,851 & 0,853 & 0,856 & 0,858 & 0,860 & 0,863 & 0,865 & 0,867 \\ \hline
18 & 251 & 0,869 & 0,871 & 0,874 & 0,876 & 0,878 & 0,880 & 0,882 & 0,884 & 0,886 & 0,888 \\ \hline
19 & 252 & 0,890 & 0,892 & 0,893 & 0,895 & 0,897 & 0,899 & 0,901 & 0,903 & 0,904 & 0,906 \\ \hline
20 & 253 & 0,908 & 0,909 & 0,911 & 0,913 & 0,914 & 0,916 & 0,917 & 0,919 & 0,921 & 0,922 \\ \hline
21 & 254 & 0,923 & 0,925 & 0,926 & 0,928 & 0,929 & 0,931 & 0,932 & 0,933 & 0,935 & 0,936 \\ \hline
22 & 255 & 0,937 & 0,938 & 0,940 & 0,941 & 0,942 & 0,943 & 0,944 & 0,945 & 0,947 & 0,948 \\ \hline
23 & 256 & 0,949 & 0,950 & 0,951 & 0,952 & 0,953 & 0,954 & 0,955 & 0,956 & 0,957 & 0,958 \\ \hline
24 & 257 & 0,959 & 0,960 & 0,960 & 0,961 & 0,962 & 0,963 & 0,964 & 0,965 & 0,965 & 0,966 \\ \hline
25 & 258 & 0,967 & 0,968 & 0,968 & 0,969 & 0,970 & 0,970 & 0,971 & 0,972 & 0,972 & 0,973 \\ \hline
26 & 259 & 0,974 & 0,974 & 0,975 & 0,976 & 0,976 & 0,977 & 0,977 & 0,978 & 0,978 & 0,979 \\ \hline
27 & 260 & 0,979 & 0,980 & 0,980 & 0,981 & 0,981 & 0,982 & 0,982 & 0,983 & 0,983 & 0,984 \\ \hline
\end{tabular}
\emph{Extrait d'une feuille de calcul}

\end{center}

Exemple d'utilisation: au croisement de la ligne 12 et de la colonne E  le nombre \np{0,706} correspond à $P(X\leqslant \np{245,3})$.
%%%%%%%%%%%   fin Antilles-Guyane juin 2013
\newpage
%%%%%%%%%%%   Asie juin 2013
\hypertarget{Asie}{}

\lfoot{\small{Asie}}
\rfoot{\small{18 juin 2013}}
\renewcommand \footrulewidth{.2pt}
\pagestyle{fancy}
\thispagestyle{empty}

\begin{center}{\Large{\textbf{\decofourleft~Baccalauréat S Asie 
18 juin 2013~\decofourright}}} \end{center}

\vspace{0,25cm}

Dans l'ensemble du sujet, et pour chaque question, toute trace de recherche même incomplète, ou d'initiative même non fructueuse, sera prise en compte dans l'évaluation.

\vspace{0,25cm}

\textbf{\textsc{Exercice 1} \hfill 5 points}

\textbf{Commun  à tous les candidats}

\medskip

Dans cet exercice, les probabilités\index{probabilité} seront arrondies au centième. 

\bigskip

\textbf{Partie A}

\medskip
 
Un grossiste achète des boîtes de thé vert chez deux fournisseurs. Il achète 80\,\% de ses boîtes chez le fournisseur A et 20\,\% chez le fournisseur B.

\medskip
 
10\,\% des boîtes provenant du fournisseur A présentent des traces de pesticides et 20\,\% de celles provenant du fournisseur B présentent aussi des traces de pesticides.
 
On prélève au hasard une boîte du stock du grossiste et on considère les évènements suivants : 

\setlength\parindent{8mm}
\begin{itemize}
\item évènement A : \og la boîte provient du fournisseur A \fg{} ; 
\item évènement B : \og la boîte provient du fournisseur B \fg{} ; 
\item évènement S : \og la boîte présente des traces de pesticides \fg.
\end{itemize}
\setlength\parindent{0mm}

\medskip
 
\begin{enumerate}
\item Traduire l'énoncé sous forme d'un arbre pondéré. 
\item 
	\begin{enumerate}
		\item Quelle est la probabilité de l'évènement $B \cap \overline{S}$ ? 
		\item Justifier que la probabilité que la boîte prélevée ne présente aucune trace de pesticides est égale à $0,88$.
	\end{enumerate} 
\item On constate que la boîte prélevée présente des traces de pesticides. 

Quelle est la probabilité que cette boîte provienne du fournisseur B ?
\end{enumerate}

\bigskip
 
\textbf{Partie B}

\medskip
 
Le gérant d'un salon de thé achète $10$~boîtes chez le grossiste précédent. On suppose que le stock de ce dernier est suffisamment important pour modéliser cette situation par un tirage aléatoire de $10$~boîtes avec remise.
 
On considère la variable aléatoire $X$ qui associe à ce prélèvement de $10$~boîtes, le nombre de boîtes sans trace de pesticides.

\medskip
 
\begin{enumerate}
\item Justifier que la variable aléatoire $X$ suit une loi binomiale\index{loi binomiale} dont on précisera les paramètres. 
\item Calculer la probabilité que les 10 boîtes soient sans trace de pesticides. 
\item Calculer la probabilité qu'au moins $8$~boîtes ne présentent aucune trace de pesticides.
\end{enumerate}
 
\bigskip
 
\textbf{Partie C}

\medskip
 
À des fins publicitaires, le grossiste affiche sur ses plaquettes: \og 88\,\% de notre thé est garanti sans trace de pesticides \fg. 

Un inspecteur de la brigade de répression des fraudes souhaite étudier la validité de l'affirmation. À cette fin, il prélève $50$~boîtes au hasard dans le stock du grossiste et en trouve $12$ avec des traces de pesticides.

\medskip
 
On suppose que, dans le stock du grossiste, la proportion de boîtes sans trace de pesticides est bien égale à $0,88$.
 
On note $F$ la variable aléatoire qui, à tout échantillon de $50$~boîtes, associe la fréquence des boîtes ne contenant aucune trace de pesticides.

\medskip
 
\begin{enumerate}
\item Donner l'intervalle de fluctuation asymptotique de la variable aléatoire $F$ au seuil de 95\,\%. 
\item L'inspecteur de la brigade de répression peut-il décider, au seuil de 95\,\%, que la publicité est mensongère ? 
\end{enumerate}
 
\vspace{0,5cm}

\textbf{\textsc{Exercice 2} \hfill 6 points}

\textbf{Commun  à tous les candidats} 

On considère les fonctions $f$ et $g$ définies pour tout réel $x$ par : 

\[f(x) = \text{e}^x \quad  \text{et}\quad  g(x) = 1 - \text{e}^{- x}.\]
 
Les courbes représentatives de ces fonctions dans un repère orthogonal du plan, notées respectivement $\mathcal{C}_{f}$ et $\mathcal{C}_{g}$, sont fournies en annexe.

\bigskip
 
\textbf{Partie A}

\medskip
  
Ces courbes semblent admettre deux tangentes communes. Tracer aux mieux ces tangentes sur la figure de l'annexe. 

\bigskip
 
\textbf{Partie B}

\medskip 

Dans cette partie, on admet l'existence de ces tangentes communes. 

On note $\mathcal{D}$ l'une d'entre elles. Cette droite est tangente à la courbe $\mathcal{C}_{f}$ au point A d'abscisse $a$ et tangente à la courbe $\mathcal{C}_{g}$ au point B d'abscisse $b$.

\medskip
 
\begin{enumerate}
\item 
	\begin{enumerate}
		\item Exprimer en fonction de $a$ le coefficient directeur de la tangente à la courbe $\mathcal{C}_{f}$ au point A. 
		\item Exprimer en fonction de $b$ le coefficient directeur de la tangente à la courbe $\mathcal{C}_{g}$ au point B. 
		\item En déduire que $b = - a$.
	\end{enumerate} 
\item Démontrer que le réel $a$ est solution de l'équation 

\[2( x - 1)\text{e}^x + 1 = 0.\]

\end{enumerate} 
\bigskip
 
\textbf{Partie C}

\medskip
 
On considère la fonction $\varphi$ définie sur  $\R$ par 

\[\varphi(x) = 2(x -1)\text{e}^x + 1.\]
 
\begin{enumerate}
\item 
	\begin{enumerate}
		\item Calculer les limites de la fonction $\varphi$ en $- \infty$ et $+ \infty$. 
		\item Calculer la dérivée de la fonction $\varphi$, puis étudier son signe. 
		\item Dresser le tableau de variation de la fonction $\varphi$ sur $\R$. Préciser la valeur de $\varphi(0)$.
	\end{enumerate} 
\item
	\begin{enumerate}
		\item Démontrer que l'équation $\varphi(x) = 0$ admet exactement deux solutions dans $\R$. 
		\item On note $\alpha$ la solution négative de l'équation $\varphi(x) = 0$ et $\beta$ la solution positive de cette équation.
		 
À l'aide d'une calculatrice, donner les valeurs de $\alpha$ et $\beta$ arrondies au centième.
	\end{enumerate} 
\end{enumerate}

\bigskip
 
\textbf{Partie D}

\medskip

Dans cette partie, on démontre l'existence de ces tangentes communes, que l'on a admise dans la partie B.
 
On note E le point de la courbe $\mathcal{C}_{f}$ d'abscisse $\alpha$ et F le point de la courbe $\mathcal{C}_{g}$ d'abscisse $- \alpha$ ($\alpha$ est le nombre réel défini dans la partie C).

\medskip
 
\begin{enumerate}
\item Démontrer que la droite (EF) est tangente à la courbe $\mathcal{C}_{f}$ au point E. 
\item Démontrer que (EF) est tangente à $\mathcal{C}_{g}$ au point F. 
\end{enumerate}

\vspace{0,5cm}

\textbf{\textsc{Exercice 3} \hfill 4 points}

\textbf{Commun  à tous les candidats}

\medskip

\emph{Les quatre questions de cet exercice sont indépendantes.\\ 
Pour chaque question, une affirmation est proposée. Indiquer si chacune d'elles est vraie ou fausse, en justifiant la réponse. Une réponse non justifiée ne rapporte aucun point.}

\medskip
 
Dans les questions 1. et 2., le plan est rapporté au repère orthonormé direct \Ouv.\index{complexes} 

On considère les points A, B, C, D et E d'affixes respectives : 

\[a = 2 + 2\text{i},\quad  b = - \sqrt{3} + \text{i},\quad c = 1 + \text{i}\sqrt{3},\quad d = - 1 + \dfrac{\sqrt{3}}{2}\text{i}\quad \text{et}\quad e = - 1 + \left(2 + \sqrt{3} \right)\text{i}.\] 

\begin{enumerate}
\item \textbf{Affirmation 1} : les points A, B et C sont alignés. 
\item \textbf{Affirmation 2} : les points B, C et D appartiennent à un même cercle de centre E. 
\item Dans cette question, l'espace est muni d'un repère \Oijk.
 
On considère les points I(1~;~0~;~0), J(0~;~1~;~0) et K(0~;~0~;~1).  

\textbf{Affirmation 3} : la droite $\mathcal{D}$ de représentation paramétrique $\left\{\begin{array}{l c l}
x &=& 2 - t \\
y &=& 6 - 2 t\\
z &=&-2 + t
\end{array}\right.$  où $t \in \R$, coupe le plan  (IJK) au point E$\left(- \dfrac{1}{2}~;~1~;~\dfrac{1}{2} \right)$.
\item Dans le cube ABCDEFGH, le point T est le milieu du segment [HF].

\medskip

\begin{center}
\psset{unit=1cm}
\begin{pspicture}(8,7.4)
\psline(2.1,1)(6.2,1)(5.6,5)(1.5,5)(2.6,6.9)(6.8,6.9)(5.6,5)%ABFEHGF
\psline(6.2,1)(7.4,3)(6.8,6.9)%BCG
\psline(0,5.42)(1.5,5)
\psline(7.4,3)(8.6,2.7)
\psline(2.1,1)(1.5,5)
\psline[linestyle=dotted,linewidth=1.5pt](1.5,5)(7.4,3)%EC
\psline(4.8,7.6)(4.1,6)
\psline(2.1,1)(1.8,0.4)
\psline[linestyle=dotted,linewidth=1.5pt](4.1,6)(2.1,1)
\psline[linestyle=dotted,linewidth=1.5pt](2.1,1)(3.3,3)(7.4,3)
\psline[linestyle=dotted,linewidth=1.5pt](3.3,3)(2.6,6.9)
\uput[l](2.1,1){A} \uput[dr](6.2,1){B} \uput[ur](7.4,3){C} 
\uput[ur](3.3,3){D} \uput[ul](1.5,5){E} \uput[ul](5.6,5){F} 
\uput[ur](6.8,6.9){G} \uput[ul](2.6,6.9){H} \uput[ul](4.1,6){T} 
\end{pspicture}
\end{center} 

\textbf{Affirmation 4} : les droites (AT) et (EC) sont orthogonales
\end{enumerate}

\vspace{0,5cm}

\textbf{\textsc{Exercice 4} \hfill 5 points}

\textbf{Candidats n'ayant pas choisi l'enseignement de spécialité}  

\bigskip

\textbf{Partie A}

\medskip
 
On considère la suite\index{suite} $\left(u_{n}\right)$ définie par : $u_{0} = 2$ et, pour tout entier nature $n$ : 

\[u_{n+1} = \dfrac{1 + 3u_{n}}{3 + u_{n}}.\] 
 
On admet que tous les termes de cette suite sont définis et strictement positifs.

\medskip
 
\begin{enumerate}
\item Démontrer par récurrence que, pour tout entier naturel $n$, on a : $u_{n} > 1$. 
\item  
	\begin{enumerate}
		\item Établir que, pour tout entier naturel $n$, on a : $u_{n+1}- u_{n} = \dfrac{\left(1 - u_{n} \right)\left(1 + u_{n} \right)}{3+ u_{n}}$.
		\item Déterminer le sens de variation de la suite $\left(u_{n}\right)$. 

En déduire que la suite $\left(u_{n}\right)$ converge. 
	\end{enumerate}
\end{enumerate}
	
\bigskip

\textbf{Partie B}

\medskip

On considère la suite\index{suite} $\left(u_{n}\right)$ 	définie par : $u_{0} = 2$ et, pour tout entier nature $n$ :

\[u_{n+1} = \dfrac{1 + 0,5u_{n}}{0,5 + u_{n}}.\]
 
On admet que tous les termes de cette suite sont définis et strictement positifs.

\medskip
 
\begin{enumerate}
\item On considère l'algorithme suivant :
\begin{center}
\begin{tabular}{|c |l|}\hline
 Entrée& Soit un entier naturel non nul $n$\\ \hline 
Initialisation &Affecter à $u$ la valeur 2\\ \hline 
\multirow{4}{1.2cm}{Traitement et sortie }&POUR $i$ allant de 1 à $n$\\ 
&\hspace{1cm}Affecter \`a $u$ la valeur $\dfrac{1 + 0,5u}{0,5 + u}$\\  
&\hspace{1cm}Afficher $u$\\ \hline 
&FIN POUR\\ \hline
\end{tabular}
\end{center}
 
Reproduire et compléter le tableau suivant, en faisant fonctionner cet algorithme pour $n = 3$. Les valeurs de $u$ seront arrondies au millième. 

\begin{center}
\begin{tabularx}{0.6\linewidth}{|*{4}{>{\centering \arraybackslash}X|}}\hline 
$i$&1&2& 3\\ \hline 
$u$&&&\\ \hline 
\end{tabularx}
\end{center} 
\item Pour $n = 12$, on a prolongé le tableau précédent et on a obtenu : 

\begin{center}
\begin{tabularx}{\linewidth}{|c|*{9}{>{\centering \arraybackslash}X|}}\hline 
$i$&4&5&6&7&8&9&10&11&12\\ \hline
$u$&\footnotesize\np{1,0083}&\footnotesize\np{0,9973}&\footnotesize\np{1,0009}&\footnotesize\np{0,9997}&\footnotesize\np{1,0001}&\footnotesize \np{0,99997}&\footnotesize\np{1,00001}&\footnotesize \np{0,999996}&\footnotesize \np{1,000001}\\ \hline
\end{tabularx}
\end{center}

Conjecturer le comportement de la suite $\left(u_{n}\right)$ à l'infini. 
\item On considère la suite $\left(v_{n}\right)$ définie, pour tout entier naturel $n$, par : $v_{n} = \dfrac{u_{n} - 1}{u_{n} + 1}$. 
	\begin{enumerate}
		\item Démontrer que la suite $\left(v_{n}\right)$ est géométrique de raison $- \dfrac{1}{3}$. 
		\item Calculer $v_{0}$ puis écrire $v_{n}$ en fonction de $n$.
	\end{enumerate} 
\item
	\begin{enumerate}
		\item Montrer que, pour tout entier naturel $n$, on a : $v_{n} \neq 1$. 
		\item montrer que, pour tout entier naturel $n$, on a : $u_{n} = \dfrac{1 + v_{n}}{1 - v_{n}}$.  
		\item Déterminer la limite de la suite $\left(u_{n}\right)$. 
	\end{enumerate}
\end{enumerate}

\vspace{0,5cm}

\textbf{\textsc{Exercice 4} \hfill 5 points}

\textbf{Candidats ayant choisi l'enseignement de spécialité} 
 
Un logiciel permet de transformer un élément rectangulaire d'une photographie.
 
Ainsi, le rectangle initial OEFG est transformé en un rectangle OE$'$F$'$G$'$, appelé image de OEFG.

\begin{center}
\psset{unit=1cm}
\begin{pspicture}(7.5,6.5)
\pspolygon(3.3,0.8)(7,4.8)(5.4,6.25)(1.75,2.2)
\pspolygon(3.3,0.8)(5.1,2.75)(2.1,5.6)(0.3,3.5)
\uput[dr](5.1,2.8){E} \uput[u](2.1,5.6){F} \uput[ul](0.3,3.5){G} 
\uput[d](3.3,0.8){O} \uput[dr](7,4.8){E$'$} \uput[u](5.4,6.25){F$'$} 
\uput[dl](1.75,2.2){G$'$}
\rput(3,0.12){Figure 1} 
\end{pspicture}
\end{center} 
 
L'objet de cet exercice est d'étudier le rectangle obtenu après plusieurs transformations successives.

\bigskip
 
\textbf{Partie A}

\medskip
 
Le plan est rapporté à un repère orthonormé \Oij.
 
Les points E, F et G ont pour coordonnées respectives (2~;~2), $(-1~;~5)$ et $(-3~;~3)$.
 
La transformation du logiciel associe à tout point $M(x~;~y)$ du plan le point $M'(x'~;~y')$, image du point $M$ tel que: 

\renewcommand\arraystretch{1.8}
\[\left\{\begin{array}{l c l}
x'&=&\dfrac{5}{4}x + \dfrac{3}{4}y\\
y'&=&\dfrac{3}{4}x + \dfrac{5}{4}y
\end{array}\right.\]
\renewcommand\arraystretch{1}

\begin{center}
\psset{unit=1cm}
\begin{pspicture}(-4,-2)(3,5.5)
\psaxes[linewidth=1.5pt,Dx=10,Dy=10]{->}(0,0)(-4,-1)(3,5.5)
\pspolygon(0,0)(2,2)(-1,5)(-3,3)
\uput[dl](0,0){O} \uput[ur](2,2){E} \uput[u](-1,5){F} \uput[l](-3,3){G}
\rput(-1,-1.5){Figure 2} 
\end{pspicture}
\end{center}

\begin{enumerate}
\item 
	\begin{enumerate}
		\item Calculer les coordonnées des points E$'$, F$'$ et G$'$, images des points E, F et G par cette transformation. 
		\item Comparer les longueurs OE et OE$'$ d'une part, OG et OG$'$ d'autre part.
		 
Donner la matrice carrée d'ordre 2, notée $A$, telle que: $\begin{pmatrix}x'\\y' \end{pmatrix}= A \begin{pmatrix}x\\y \end{pmatrix}$.
	\end{enumerate}
\end{enumerate}

\bigskip
 
\textbf{Partie B}

\medskip
 
Dans cette partie, on étudie les coordonnées des images successives du sommet F du rectangle OEFG lorsqu'on applique plusieurs fois la transformation du logiciel.

\medskip
 
\begin{enumerate}
\item On considère l'algorithme\index{algorithme} suivant destiné à afficher les coordonnées de ces images successives.
 
Une erreur a été commise.
 
Modifier cet algorithme pour qu'il permette d'afficher ces coordonnées.

\begin{center}
\begin{tabular}{|c|l|}\hline 
Entrée &Saisir un entier naturel non nul $N$\\ \hline 
\multirow{2}{2cm}{Initialisation }&Affecter à $x$ la valeur $- 1$\\ 
&Affecter à $y$ la valeur 5\\ \hline 
\multirow{6}{2cm}{Traitement}&POUR $i$ allant de 1 à $N$\\ 
&Affecter à $a$ la valeur $\frac{5}{4} x + \frac{3}{4}y$\\ 
&Affecter à $b$ la valeur $\frac{3}{4}x + \frac{5}{4}y$\\ 
&Affecter à $x$ la valeur $a$\\ 
&Affecter à $y$ la valeur $b$\\ 
&FIN POUR\\ \hline 
Sortie &Afficher $x$, afficher $y$\\ \hline
\end{tabular}
\end{center} 

\item On a obtenu le tableau suivant :

\[\begin{array}{|*{8}{c|}} \hline
i &1 &2 &3 &4 &5 &10 &15\\ \hline 
x &2,5 &7,25 &15,625 &\np{31,8125} &\np{63,9063} &\np{2047,9971} &\np{65535,9999}\\ \hline 
y &5,5 &8,75 &16,375 &\np{32,1875} &\np{64,0938} &\np{2048,0029} &\np{65536,0001}\\ \hline
\end{array}\]
 
Conjecturer le comportement de la suite des images successives du point F. 
\end{enumerate}

\bigskip
 
\textbf{Partie C}

\medskip

Dans cette partie, on étudie les coordonnées des images successives du sommet E du rectangle OEFG. On définit la suite des points $E_{n}\left(x_{n}~;~y_{n}\right)$ du plan par $E_{0} =$ E et la relation de récurrence :

\[\begin{pmatrix}x_{n+1}\\y_{n+1}\end{pmatrix} = A\begin{pmatrix}x_{n}\\y_{n}\end{pmatrix},\] 

où $\left(x_{n+1}~;~y_{n+1}\right)$ désignent les coordonnées du point $E_{n+1}$.

Ainsi $x_{0} = 2$ et $y_{0} = 2$. 

\medskip

\begin{enumerate}
\item On admet que, pour tout entier $n \geqslant 1$, la matrice\index{matrice} $A^n$ peut s'écrire sous la forme : $A^{n} = \begin{pmatrix}\alpha_{n}&\beta_{n}\\\beta_{n}&\alpha_{n}\end{pmatrix}$. 

Démontrer par récurrence que, pour tout entier naturel $n \geqslant 1$, on a : 
\[\alpha_{n} = 2^{n-1}  + \dfrac{1}{2^{n+1}} \quad \text{et}\quad  \beta_{n} = 2^{n-1}  - \dfrac{1}{2^{n+1}}.\] 
 
\item 
	\begin{enumerate}
		\item Démontrer que, pour tout entier naturel $n$, le point $E_{n}$ est situé sur la droite d'équation $y = x$.
		 
On pourra utiliser que, pour tout entier naturel $n$, les coordonnées $\left(x_{n}~;~y_{n}\right)$ du point $E_{n}$ vérifient :

\[\begin{pmatrix}x_{n}\\y_{n}\end{pmatrix} = A^n \begin{pmatrix}2\\2\end{pmatrix}.\]
 
		\item Démontrer que la longueur O$E_{n}$ tend vers $+ \infty$ quand $n$ tend vers $+ \infty$.
	\end{enumerate}
\end{enumerate}

\newpage

\begin{center} 
\textbf{Annexe}

\vspace{0,5cm}
 
\textbf{à rendre avec la copie}

\vspace{0,5cm}
 
\textbf{Exercice 2}

\vspace{0,5cm} 

\psset{unit=1.3cm}
\begin{pspicture*}(-5.1,-3.5)(5.1,5.1)
\psaxes[linewidth=1.5pt]{->}(0,0)(-5,-3.5)(5,5)
\psaxes[linewidth=1.5pt](0,0)(-5,-3.5)(5,5)
\psgrid[gridlabels=0pt,subgriddiv=1,gridwidth=0.2pt,gridcolor=orange]
\psplot[plotpoints=8000,linewidth=1.25pt,linecolor=blue]{-4.5}{1.55}{2.71828 x exp}
\psplot[plotpoints=8000,linewidth=1.25pt,linecolor=red]{-1.5}{5}{1 2.71828 x neg exp sub}
\uput[dr](0,0){O}
\rput(1.5,3.5){$\mathcal{C}_{f}$}
\rput(2.4,1.12){$\mathcal{C}_{g}$}
\end{pspicture*}
\vspace{0,5cm}
\end{center}
%%%%%%%%%%%   fin Asie juin 2013
\newpage
%%%%%%%%%%%  Centres étrangers juin 2013
\hypertarget{Centres etrangers}{}

\lhead{\small Baccalauréat S}
\lfoot{\small Centres étrangers}
\rfoot{\small 12 juin 2013}

\begin{center}\textbf{Durée  : 4 heures }

\vspace{0,25cm}

{\Large\textbf{\decofourleft~Baccalauréat S Centres étrangers
~\decofourright\\12 juin 2013}}
\end{center}

\vspace{0,25cm}

L'usage des calculatrices est autorisé selon les termes de la circulaire \no 99-186 du 16 novembre 1999.

\emph{Il est rappelé que la qualité de la rédaction, la clarté et la précision des raisonnements entreront pour une part importante dans l'appréciation des copies.}

\subsection*{Exercice 1 \hfill 6 points}

\textbf{\emph{Commun à tous les candidats}}

\medskip

Un industriel fabrique des vannes électroniques destinées à des  circuits hydrauliques.

Les quatre parties A, B, C, D sont indépendantes.

\bigskip

\textbf{Partie A}

\medskip

La durée de vie d'une vanne, exprimée en heures, est une variable aléatoire $T$ qui suit la loi exponentielle\index{loi exponentielle} de paramètre $\lambda = \numprint{0,0002} $.

\begin{enumerate}
\item Quelle est la durée de vie moyenne d'une vanne ? 

\item Calculer la probabilité, à $0,001$ près, que la durée de vie d'une vanne soit supérieure à \np{6000} heures.
\end{enumerate}
\bigskip

\textbf{Partie B}

\parbox{8cm}{ Avec trois vannes identiques $V_{1}$, $V_{2}$ et $V_{3}$, on fabrique le circuit hydraulique ci-contre.\\
Le circuit est en état de marche si $V_{1}$ est en état d arche ou si $V_{2}$ et $V_{3}$ le sont simultanément.}\hfill\parbox{10cm}{\psset{unit=0.8}
\begin{pspicture}(-1,-2)(10,3)
%\psgrid
\psline(0,0)(2,0)
\psline(2,0)(4,1)
\psframe(4,0.5)(5,1.5)
\psline(5,1)(8,0)
\psline(2,0)(3,-1)
\psframe(3,-1.5)(4,-0.5)
\psline(4,-1)(5,-1)
\psframe(5,-1.5)(6,-0.5)
\psline(6,-1)(8,0)
\psline(8,0)(10,0)
\uput[u](4.5,0.7){$V_{1}$}
\uput[d](3.5,-0.7){$V_{2}$}\uput[d](5.5,-0.7){$V_{3}$}
\end{pspicture}}

On assimile à une expérience aléatoire le fait que chaque vanne est ou n'est pas en état de marche après \numprint{6000} heures. On note :

\begin{itemize}
\item[$\bullet~~$] $F_{1}$ l'évènement : \og la vanne $V_{1}$ est en état de marche après \np{6000} heures \fg.
\item[$\bullet~~$] $F_{2}$ l'évènement : \og la vanne $V_{2}$ est en état de marche après \np{6000} heures \fg.
\item[$\bullet~~$] $F_{3}$ l'évènement : \og la vanne $V_{3}$ est en état de marche après \np{6000} heures \fg.
\item[$\bullet~~$] $E$ : l'évènement : \og le circuit est en état de marche après \np{6000} heures \fg.
\end{itemize}

On admet que les évènements $F_{1}$, $F_{2}$ et $F_{3}$ sont deux à deux indépendants et ont chacun une  probabilité égale à $0,3$.

\parbox{8cm}{\begin{enumerate}
\item L'arbre probabiliste ci-contre représente une partie de la situation.

Reproduire cet arbre et placer les probabilités sur les branches. 
\item Démontrer que $P(E) = 0,363$. 

\item Sachant que le circuit est en état de marche après \np{6000} heures, calculer la probabilité que la vanne $V_{1}$ soit en état de marche à ce moment là. Arrondir au millième.

\end{enumerate}}\hfill
\parbox{8cm}{
\psset{nodesep=2mm,levelsep=20mm,treesep=10mm}
\pstree[treemode=R]{\TR{}}
{\TR{}~[tnpos=r]{$F_1$}
\pstree
	{\TR{}~[tnpos=a]{$\overline{F_{1}}$}}
		{
			\pstree
				{\TR{}~[tnpos=r]{$F_{2}$}}
				{\TR{}~[tnpos=r]{$F_{3}$}
				\Tn
				}
				\Tn
		}
}
}
\bigskip

\textbf{Partie C}

\medskip

L'industriel affirme que seulement 2\,\% des vannes qu'il fabrique sont défectueuses. On suppose que cette affirmation est vraie, et l'on note $F$ la variable aléatoire égale à la fréquence de vannes défectueuses dans un échantillon aléatoire de $400$ vannes prises dans la production totale.
\begin{enumerate}
\item  Déterminer l'intervalle $I$ de fluctuation asymptotique au seuil de 95\,\% de la variable $F$, 

\item On choisit $400$ vannes au hasard dans la production, On assimile ce choix à un tirage aléatoire de $400$ vannes, avec remise, dans la production.

Parmi ces $400$ vannes, $10$ sont défectueuses.
 
Au vu de ce résultat peut-on remettre en cause. au seuil de 95\,\%, l'affirmation de l'industriel? 
\end{enumerate}
\bigskip
 \textbf{Partie D}

\medskip

\emph{Dans cette partie, les probabilités calculées seront arrondies au millième.}

L'industriel commercialise ses vannes auprès de nombreux clients, La demande mensuelle est une variable aléatoire $D$ qui suit la loi normale\index{loi normale} d'espérance $\mu = 800$ et d'écart-type $\sigma = 40$.
\begin{enumerate}
\item Déterminer $P(760\leqslant D \leqslant 840)$.
\item Déterminer $P(D\leqslant 880)$. 

\item L'industriel pense que s'il constitue un stock mensuel de $880$ vannes, il n'aura pas plus de 1\,\% de chance d'être en rupture de stock. A-t-il raison ? 
\end{enumerate}

\vspace{0,5cm}

\subsection*{Exercice 2 \hfill 4 points}

\emph{\textbf{Commun à tous les candidats}}

\medskip

\emph{Les quatre questions sont indépendantes.}
 
\emph{Pour chaque question, une affirmation est proposée. Indiquer si elle est vraie ou fausse en justifiant la réponse. Une réponse non justifiée ne sera pas prise en compte.}
\medskip

Dans l'espace muni d'un repère orthonormé, on considère 
\begin{itemize}
\item les points A $(12~;~0~;~0)$, B $( 0~;~-15~;~0)$, C $( 0~;~0~;~20)$, D $(2~;~7~;~- 6)$, E $(7~;~3~;~-3)$;
\item le plan $\mathscr{P}$ d'équation cartésienne : $2x + y - 2z - 5 = 0 $
\end{itemize}

\bigskip

\textbf{Affirmation 1}

Une équation cartésienne du plan parallèle à $\mathscr{P}$ et passant par le point A est :
 
\[2x + y + 2z - 24 = 0\]

\textbf{Affirmation 2}

Une représentation paramétrique de la droite (AC) est : $\left\{\begin{array}{*{3}{l}}x&=&9 - 3t\\y&=&0\\
z&=&5 + 5t\end{array}\right.,~t\in\mathbb{R}$.

\textbf{Affirmation 3}

La droite (DE) et le plan $\mathscr{P}$ ont au moins un point commun.

\textbf{Affirmation 4}

La droite (DE) est orthogonale au plan (ABC).

\vspace{0,5cm}

\subsection*{Exercice 3\hfill 5 points}

\textbf{Commun à tous les candidats}

\medskip

On considère la fonction $g$ définie pour tout réel $x$ de l'intervalle $[0 ~;~1]$ par : 

\[g(x) = 1 + \mathrm{e}^{-x}.\]

On admet que, pour tout réel $x$ de l'intervalle $[0~;~ 1]$, $g(x) >0$.

\medskip

\parbox{8cm}{On note $\mathscr{C}$ la courbe représentative de la fonction $g$ dans un repère orthogonal, et $\mathscr{D}$ le domaine plan compris d'une part entre l'axe des abscisses et la courbe $\mathscr{C}$, d'autre part entre les droites d'équation $x = 0$ et $x = 1 $.\\
La courbe $\mathscr{C}$ et le domaine $\mathscr{D}$ sont représentés ci-contre.}\hfill\parbox{8cm}{

\psset{xunit=4.0cm,yunit=2.0cm}
\begin{pspicture*}(-0.2,-0.3)(1.5,2.2)
\psgrid[subgriddiv=0,gridlabels=0,gridcolor=lightgray](0,0)(-0.2,-0.2)(1.2,2.2)
\psset{xunit=4.0cm,yunit=2.0cm,algebraic=true,dotstyle=o,dotsize=3pt 0,linewidth=0.8pt,arrowsize=3pt 2,arrowinset=0.25}
\psaxes[labelFontSize=\scriptstyle,xAxis=true,yAxis=true,Dx=1,Dy=1,ticksize=-2pt 0,subticks=2]{->}(0,0)(-0.2,-0.2)(1.2,2.2)
\def\f{1+EXP(-x)}
\psplot[plotpoints=200]{0.0}{1.0}{\f}
\def\inf{0} \def\sup{1}
\pscustom[fillstyle=hlines,linestyle=solid,linewidth=0.5pt, hatchcolor=orange]
{
\psplot{\inf}{\sup}{\f} % courbe de f sur [inf ; sup]
\lineto(\sup,0)\lineto(\inf,0)\closepath % indispensable !
}
\uput[u](0.68,1.51){$\mathscr{C}$}\\
\uput[u](0.5,1){$\mathscr{D}$}
\uput[d](1.2,0){$x$}
\uput[l](0,2){$y$}
\uput[ul](0,0){0}
\end{pspicture*}}

\bigskip

Le but de cet exercice est de partager le domaine $\mathscr{D}$ en deux domaines de même aire, d'abord par une droite parallèle à l'axe des ordonnées (partie A), puis par une droite parallèle à l'axe des abscisses (partie B).

\bigskip

\textbf{Partie A}

\medskip

\parbox{8cm}{Soit $a$ un réel tel que $0\leqslant a\leqslant 1$.\\
On note $\mathscr{A}_{1}$ l'aire du domaine compris entre la courbe $\mathscr{C}$, l'axe $(Ox)$,les droites d'équation $x = 0$ et $x =a$ , puis $\mathscr{A}_{2}$ celle du domaine compris entre la courbe $\mathscr{C}$, $(Ox)$ et les droites d'équation $x = a$ et $x = 1$.\\ 
$\mathscr{A}_{1}$ et $\mathscr{A}_{2}$ sont exprimées en unités d'aire. }\hfill
\parbox{8cm}{

\psset{xunit=4.0cm,yunit=2.0cm}
\begin{pspicture*}(-0.2,-0.3)(1.4,2.2)
\psgrid[subgriddiv=0,gridlabels=0,gridcolor=lightgray](0,0)(-0.2,-0.2)(1.2,2.2)
\psset{xunit=4.0cm,yunit=2.0cm,algebraic=true,dotstyle=o,dotsize=3pt 0,linewidth=0.8pt,arrowsize=3pt 2,arrowinset=0.25}
\psaxes[labelFontSize=\scriptstyle,xAxis=true,yAxis=true,Dx=1,Dy=1,ticksize=-2pt 0,subticks=2]{->}(0,0)(-0.2,-0.2)(1.2,2.2)
\def\f{1+EXP(-x)}
\psplot[plotpoints=200]{0.0}{1.0}{\f}
\def\inf{0} \def\sup{0.4}
\pscustom[fillstyle=hlines,linestyle=solid,linewidth=0.5pt, hatchcolor=orange]
{
\psplot{\inf}{\sup}{\f} % courbe de f sur [inf ; sup]
\lineto(\sup,0)\lineto(\inf,0)\closepath % indispensable !
}
\def\inf{0.4} \def\sup{1}
\pscustom[fillstyle=hlines,linestyle=solid,linewidth=0.5pt, hatchcolor=orange, hatchangle=-45]
{
\psplot{\inf}{\sup}{\f} % courbe de f sur [inf ; sup]
\lineto(\sup,0)\lineto(\inf,0)\closepath % indispensable !
}
\psline(0.4,0)(0.4,2)
\uput[u](0.2,1){$\mathscr{A}_{1}$}
\uput[u](0.6,1){$\mathscr{A}_{2}$}
\uput[u](0.68,1.51){$\mathscr{C}$}
\uput[d](0.4,0){$a$}
\uput[d](1.2,0){$x$}
\uput[l](0,2){$y$}
\uput[ul](0,0){0}
\end{pspicture*}}

\begin{enumerate}
\item 
	\begin{enumerate}
		\item Démontrer que $\mathscr{A}_{1}= a - \mathrm{e}^{-a} + 1$.
		\item Exprimer $\mathscr{A}_{2}$ en fonction de $a$.
	\end{enumerate}
\item Soit $f$ la fonction définie pour tout réel $x$ de l'intervalle $[0~;~1]$ par : 

\[f(x) =2x - 2\,\mathrm{e}^{- x} + \dfrac{1}{\mathrm{e}}.\]

	\begin{enumerate}
		\item Dresser le tableau de variation de la fonction $f$ sur l'intervalle $[0~;~1]$. On précisera les valeurs exactes de $f(0)$ et $f(1)$. 
		\item Démontrer que la fonction $f$ s'annule une fois et une seule sur l'intervalle $[0~;~1]$. en un réel $\alpha$. Donner la valeur de $\alpha$ arrondie au centième.
	\end{enumerate}
\item En utilisant les questions précédentes, déterminer une valeur approchée du réel $a$ pour lequel les aires $\mathscr{A}_{1}$ et $\mathscr{A}_{2}$ sont égales.
\end{enumerate}

\textbf{Partie B}

\medskip

Soit $b$ un réel positif.

Dans cette partie, on se propose de partager le domaine $\mathscr{D}$ en deux domaines de même aire par la droite d'équation $y=b$. On admet qu'il existe un unique réel $b$ positif solution. 
\begin{enumerate}
\item Justifier l'inégalité $b<1 + \dfrac{1}{\mathrm{e}}$. On pourra utiliser un argument graphique. 
\item Déterminer la valeur exacte du réel $b$. 
\end{enumerate}

\subsection*{Exercice 4 \hfill 5 points}

\textbf{Candidats n'avant pas choisi la spécialité mathématique }

\medskip

L'objet de cet exercice est l'étude de la suite $\left(u_{n}\right)$ définie par son premier terme 

$u_{1}=\dfrac{3}{2}$ et la relation  de récurrence : $u_{n+1} =\dfrac{nu_{n}+1}{2(n + 1)}$.

\medskip

\textbf{Partie A - Algorithmique et conjectures }

\medskip

\parbox{5cm}{Pour calculer et afficher le terme $u_{9}$ de la suite, un élève propose l'algorithme\index{algorithme} ci-contre.

Il a oublié de compléter deux lignes.}\hfill
\parbox{7cm}{
\begin{tabularx}{\linewidth}{|l|*{2}{>{\arraybackslash}X|}} \hline
Variables&$n$ est un entier naturel \\
&$u$ est un réel \\
\hline
Initialisation &Affecter à $n$ la valeur 1\\
&Affecter à $u$ la valeur 1,5 \\
\hline
Traitement &Tant que $n<9 $\\
&\hspace*{1cm}Affecter à $u$ la valeur \dots\\ 
&\hspace*{1cm}Affecter à $n$ la valeur \dots\\
%\hline
&Fin Tant que \\
\hline
Sortie&Afficher la variable $u$ \\
\hline
\end{tabularx}}

\medskip

\begin{enumerate}
\item Recopier et compléter les deux lignes de l'algorithme où figurent des points de suspension. 
\item Comment faudrait-il modifier cet algorithme pour qu'il calcule et affiche tous les termes de la suite de $u_{2}$ jusqu'à $u_{9}$ ? 
\item  Avec cet algorithme modifié, on a obtenu les résultats suivants, arrondis au dix-millième: 

\medskip

\begin{tabularx}{\linewidth}{|l|*{10}{>{\centering \arraybackslash}X|}} \hline
n &1&2 &3 &4 &5 &6 &\dots&99 &100 \\
\hline
$u_{n}$ &1,5 &0,625 &0,375 &0,2656 &\np{0,2063} &\np{0,1693}&\dots&\np{0,0102} &\np{0,0101} \\
\hline
\end{tabularx}

\medskip

Au vu de ces résultats, conjecturer le sens de variation et la convergence de la suite $\left(u_{n}\right)$. 
\end{enumerate}

\bigskip

\textbf{Partie B - Étude mathématique}

\medskip

On définit une suite\index{suite} auxiliaire $\left(v_{n}\right)$ par : pour tout entier $n\geqslant 1$, $v _{n} = nu_{n} -1$.

\medskip
 
\begin{enumerate}
\item Montrer que la suite $\left(v_{n}\right)$ est géométrique ; préciser sa raison et son premier terme. 

\item En déduire que, pour tout entier naturel $n\geqslant 1$, on a : $u_{n}= \dfrac{1 + (0,5)^{n}}{n}$.
\item Déterminer la limite de la suite $\left(u_{n}\right)$.
\item Justifier que, pour tout entier $n\geqslant 1$ , on a : $u_{n+1}- u_{n}=- \dfrac{1 + (1 + 0,5n)(0,5)^{n}}{n(n + 1)}$.

En déduire le sens de variation de la suite $\left(u_{n}\right)$.
\end{enumerate}

\bigskip

\textbf{Partie C - Retour à l'algorithmique }

En s'inspirant de la partie A, écrire un algorithme permettant de déterminer et d'afficher le plus petit entier $n$ tel que $u_{n} < 0,001$. 

\vspace{0,5cm}

\subsection*{Exercice 4 \hfill 5 points}

\textbf{Candidats ayant choisi la spécialité mathématique }

\medskip

Une espèce d'oiseaux ne vit que sur deux îles A et B d'un archipel.

Au début de l'année 2013, 20 millions d'oiseaux de cette espèce sont présents sur l'île A et 10 millions sur l'île B.

Des observations sur plusieurs années ont permis aux ornithologues d'estimer que, compte tenu des naissances, décès, et migrations entre les deux îles, on retrouve au début de chaque année les proportions suivantes : 
\begin{itemize}
\item sur l'île A : 80\,\% du nombre d'oiseaux présents sur l'île A au début de l'année précédente et 30\,\% du nombre d'oiseaux présents sur l'île B au début de l'année précédente;
\item sur l'île B : 20\,\% du nombre d'oiseaux présents sur l'île A au début de l'année précédente et  70\,\% du nombre d'oiseaux présents sur l'île B au début de l'année précédente.
\end{itemize}

\medskip

Pour tout entier naturel $n$, on note $a_{n}$ (respectivement $b_{n}$) le nombre d' oiseaux (en millions) présents sur l'île A (respectivement B) au début de l'année $(2013 + n)$.

\bigskip

\textbf{Partie A - Algorithmique et conjectures }

\medskip

\parbox{5cm}{On donne ci-contre un algorithme\index{algorithme} qui doit afficher le nombre d'oiseaux vivant sur chacune des deux iles,  pour chaque année comprise entre 2013 et une année  choisie par l'utilisateur.}\hfill
\parbox{6.5cm}{\fbox{\begin{minipage}{6.5cm}
\textbf{\emph{Début de l'algorithme}}\\
Lire $n$\\
Affecter à $a$ la valeur 20\\
Affecter à $b$ la valeur 10\\
Affecter à $i$ la valeur 2013\\
Afficher $i$\\
Afficher $a$\\
Afficher $b$\\ 
Tant que $i < n$ faire\\
\hspace*{1cm} Affecter à $c$ la valeur $(0,8a + 0,3b)$\\
\hspace*{1cm} Affecter à $b$ la valeur $(0,2a + 0,7 b)$\\
\hspace*{1cm} Affecter à $a$ la valeur $c$\\
Fin du Tant que\\
\textbf{\emph{Fin de l 'algorithme}}
\end{minipage}}}

\begin{enumerate}
\item Cet algorithme comporte des oublis dans le traitement. Repérer ces oublis et les corriger. 

\item On donne ci-dessous une copie d'écran des résultats obtenus après avoir corrigé l'algorithme précédent dans un logiciel d'algorithmique, l'utilisateur avant choisi l'année 2020.

\begin{center}
\fbox{\begin{minipage}{11cm}
$\star\star\star$ Algorithme lancé $\star\star\star$\\
En l'année 2013, $a$ prend la valeur 20  et $b$ prend la valeur 10\\ 
En l'année 2014, $a$ prend la valeur 19 et $b$ prend la valeur 11\\
En l'année 2015, $a$ prend la valeur 18,5 et $b$ prend la valeur 11,5\\
En l'année 2016, $a$ prend la valeur 18,25 et $b$ prend la valeur 11,75\\
En l'année 2017, $a$ prend la valeur 18,125 et $b$ prend la valeur 11,875\\
En l'année 2018. $a$ prend la valeur \np{18,0425} et $b$ prend la valeur \np{11,9375}\\ 
En l'année 2019, $a$ prend la valeur \np{18,03125} et $b$ prend la valeur \np{11,96875}\\ 
En l'année 2020, $a$ prend la valeur \np{18,015625} et $b$ prend la valeur \np{11,984375}\\ 
$\star\star\star$ Algorithme terminé $\star\star\star$
\end{minipage}}
\end{center}
Au vu de ces résultats, émettre des conjectures concernant le sens de variation et la convergence des suites $\left(a_{n}\right)$ et $\left(b_{n}\right)$.
\end{enumerate}

\bigskip

\textbf{Partie B - Étude mathématique}

\medskip

On note $U_{n}$ la matrice\index{matrice} colonne $\begin{pmatrix}a_{n}\\b_{n}\end{pmatrix}$.

\medskip

\begin{enumerate}
\item Montrer que, pour tout entier naturel $n$, $U_{n+1}=MU_{n}$, où $M$ est une matrice carrée d'ordre 2 que l'on déterminera.

On admet alors que $U_{n}=M^{n}U_{0}$ pour tout entier naturel $n\geqslant 1$.
\item À l'aide d'un raisonnement par récurrence, justifier que, pour tout  entier naturel $n\geqslant 1$ : 
\[M^{n}= \begin{pmatrix}
0,6 + 0,4\times 0,5^{n}&0,6 - 0,6\times 0,5^{n}\\
0,4 - 0,4\times 0,5^{n}&0,4 + 0,6\times 0,5^{n}
\end{pmatrix}.\]

On ne détaillera le calcul que pour le premier des coefficients de la matrice $M^{n}$.

\item Exprimer $a_{n}$ en fonction de $n$, pour tout entier naturel $n\geqslant 1$.

\item Avec ce modèle, peut-on dire qu'au bout d'un grand nombre d'années, le nombre d'oiseaux sur l'île A va se stabiliser? Si oui, préciser vers quelle valeur.
\end{enumerate}
\label{fin}
%%%%%%%%%%   fin Centres étrangers juin 2013
\newpage
%%%%%%%%%%   Métropole juin 2013
\hypertarget{Metropole}{}

\lfoot{\small{Métropole}}
\rfoot{\small 20 juin 2013}
\pagestyle{fancy}
\thispagestyle{empty}
\begin{center}
{\Large \textbf{\decofourleft~Baccalauréat S Métropole 20 juin 2013~\decofourright}}
\end{center}

\vspace{0,25cm}

\textbf{\textsc{Exercice 1} \hfill 4 points} %\section{exo 1}
 
\textbf{Commun à tous les candidats}

\medskip

Une jardinerie vend de jeunes plants d'arbres qui proviennent de trois horticulteurs : 35\,\% des plants proviennent de l'horticulteur H$_{1}$, 25\,\% de l'horticulteur H$_{2}$ et le reste de l'horticulteur H$_{3}$. Chaque horticulteur livre deux catégories d'arbres : des conifères et des arbres à feuilles.
 
La livraison de l'horticulteur H$_{1}$ comporte 80\,\% de conifères alors que celle de l'horticulteur H$_{2}$ n'en comporte que 50\,\% et celle de l'horticulteur H$_{3}$ seulement 30\,\%.

\medskip
 
\begin{enumerate}
\item Le gérant de la jardinerie choisit un arbre au hasard dans son stock. 

On envisage les événements suivants :

\setlength\parindent{6mm} 
\begin{itemize}
\item $H_{1}$ : \og l'arbre choisi a été acheté chez l'horticulteur H$_{1}$ \fg, 
\item $H_{2}$ : \og l'arbre choisi a été acheté chez l'horticulteur H$_{2}$ \fg, 
\item $H_{3}$ : \og l'arbre choisi a été acheté chez l'horticulteur H$_{3}$ \fg, 
\item $C$\phantom{$_{3}$} : \og l'arbre choisi est un conifère \fg, 
\item $F$\phantom{$_{3}$} : \og l'arbre choisi est un arbre feuillu \fg.
\end{itemize}
\setlength\parindent{0mm} 

\medskip
 
	\begin{enumerate}
		\item Construire un arbre pondéré traduisant la situation. 
		\item Calculer la probabilité que l'arbre choisi soit un conifère acheté chez l'horticulteur H$_{3}$. 
		\item Justifier que la probabilité de l'évènement $C$ est égale à $0,525$. 
		\item L'arbre choisi est un conifère.
		 
Quelle est la probabilité qu'il ait été acheté chez l'horticulteur H$_1$ ? On arrondira à $10^{-3}$.
	\end{enumerate}	 
\item On choisit au hasard un échantillon de $10$~arbres dans le stock de cette jardinerie. On suppose que ce stock est suffisamment important pour que ce choix puisse être assimilé à un tirage avec remise de $10$~arbres dans le stock.
 
On appelle $X$ la variable aléatoire qui donne le nombre de conifères de l'échantillon choisi. 
	\begin{enumerate}
		\item Justifier que $X$ suit une loi binomiale\index{loi binomiale} dont on précisera les paramètres. 
		\item Quelle est la probabilité que l'échantillon prélevé comporte exactement $5$~conifères? 

On arrondira à $10^{-3}$. 
		\item Quelle est la probabilité que cet échantillon comporte au moins deux arbres feuillus ?
		 
On arrondira à $10^{-3}$.
	\end{enumerate} 
\end{enumerate}

\vspace{0,5cm}

\textbf{\textsc{Exercice 2} \hfill 7 points}
 
\textbf{Commun à tous les candidats}

\medskip

Sur le graphique ci-dessous, on a tracé, dans le plan muni d'un repère orthonormé \Oij, la courbe représentative $\mathcal{C}$ d'une fonction $f$ définie et dérivable sur l'intervalle $] 0~;~+ \infty[$. 


\begin{center}
\psset{unit=1.2cm}
\begin{pspicture}(-1.5,-2)(9,2.5)
\psaxes[linewidth=1pt,Dx=20,Dy=20](0,0)(-1.5,-2)(9,2.5)
\psaxes[linewidth=1.5pt]{->}(0,0)(1,1)
\psline(1,0)(1,2)
\psline(-1.5,2)(9,2)
\uput[dr](1,0){A}\uput[u](1,2){B}\uput[ul](0,2){C}\uput[dl](0,0){O}
\uput[u](8,0.8){\blue $\mathcal{C}$}
\uput[d](0.5,0){$\vect{\imath}$}\uput[l](0,0.5){$\vect{\jmath}$}
\psplot[plotpoints=1000,linewidth=1pt,linecolor=blue]{0.278}{9}{x ln 2 mul 2 add x div}
\end{pspicture}
\end{center}
 
On dispose des informations suivantes :
 
\setlength\parindent{6mm} 
\begin{itemize}
\item les points A, B, C ont pour coordonnées respectives (1\,, 0), (1\,, 2), (0\,, 2); 
\item la courbe $\mathcal{C}$ passe par le point B et la droite (BC) est tangente à $\mathcal{C}$ en B; 
\item il existe deux réels positifs $a$ et $b$ tels que pour tout réel strictement positif $x$, 
\end{itemize}
\setlength\parindent{0mm} 
			 
\[f(x) = \dfrac{a+ b\ln x}{x}. \]

\begin{enumerate}
\item 
	\begin{enumerate}
		\item En utilisant le graphique, donner les valeurs de $f(1)$ et $f'(1)$. 
 
		\item Vérifier que pour tout réel strictement positif $x,\: f'(x) = \dfrac{(b - a) - b \ln x}{x^2}$. 
		\item En déduire les réels $a$ et $b$.
	\end{enumerate} 	
\item 
	\begin{enumerate}
		\item Justifier que pour tout réel $x$ appartenant à l'intervalle $]0\,, +\infty[,\: f'(x)$ a le même signe que $- \ln x$. 
		\item Déterminer les limites de $f$ en 0 et en $+ \infty$. On pourra remarquer que pour tout réel  
$x$ strictement positif, $f(x) = \dfrac{2}{x} + 2\;\dfrac{\ln x}{x}$. 
		\item En déduire le tableau de variations de la fonction $f$.
	\end{enumerate} 	
\item
	\begin{enumerate}
		\item Démontrer que l'équation $f(x) = 1$ admet une unique solution $\alpha$ sur l'intervalle $]0\,, 1]$. 
		\item Par un raisonnement analogue, on démontre qu'il existe un unique réel $\beta$ de l'intervalle $]1\,, + \infty]$ tel que $f(\beta) = 1$.
 
Déterminer l'entier $n$ tel que $n < \beta < n + 1$.
	\end{enumerate}		
\item On donne l'algorithme\index{algorithme} ci-dessous.

\begin{center}
\begin{tabular}{|l l|}\hline 
Variables :& $a, b$ et $m$ sont des nombres réels.\\ 
Initialisation :& Affecter à $a$ la valeur $0$. \\
	&Affecter à $b$ la valeur 1.\\ 
Traitement :& Tant que $b - a > 0,1$\\ 
&\begin{tabular}{l|l}
~~& Affecter à $m$ la valeur $\dfrac{1}{2}(a + b)$.\\ 
~~& Si $f(m) < 1$ alors Affecter à $a$ la valeur $m$.\\ 
~~&Sinon Affecter à $b$ la valeur $m$.\\ 
~~&Fin de Si.\\
\end{tabular}\\ 
&Fin de Tant que.\\ 
Sortie :&Afficher $a$.\\
& Afficher $b$.\\ \hline
\end{tabular} 
\end{center} 

	\begin{enumerate}
		\item Faire tourner cet algorithme en complétant le tableau ci-dessous que l'on recopiera sur la copie.

\begin{center} 
\begin{tabularx}{\linewidth}{|*{6}{>{\centering \arraybackslash}X|}}\hline
&étape 1 &étape 2 &étape 3 &étape 4 &étape 5 \\ \hline
$a$&0&&&&\\ \hline 
$b$&1&&&&\\ \hline 
$b - a$&&&&&\\ \hline 
$m$&&&&&\\ \hline
\end{tabularx}
\end{center} 

		\item Que représentent les valeurs affichées par cet algorithme ? 
		\item Modifier l'algorithme ci-dessus pour qu'il affiche les deux bornes d'un encadrement de $\beta$ d'amplitude $10^{-1}$.
	\end{enumerate} 	
\item Le but de cette question est de démontrer que la courbe $\mathcal{C}$ partage le rectangle OABC en deux domaines d'aires égales. 
	\begin{enumerate}
		\item Justifier que cela revient à démontrer que $\displaystyle\int_{\frac{1}{\text{e}}}^1 f(x)\:\text{d}x = 1$. 
		\item En remarquant que l'expression de $f(x)$ peut s'écrire $\dfrac{2}{x} + 2 \times \dfrac{1}{x}  \times  \ln x$, terminer la démonstration.
	\end{enumerate} 
\end{enumerate}

\vspace{0,5cm}

\newpage

\textbf{\textsc{Exercice 3} \hfill 4 points} %\section{exo 3}
 
\textbf{Commun à tous les candidats}

\medskip

Pour chacune des quatre propositions suivantes, indiquer si elle est vraie ou fausse et justifier la réponse choisie. 

Il est attribué un point par réponse exacte correctement justifiée. Une réponse non justifiée n'est pas prise en compte. Une absence de réponse n'est pas pénalisée.\index{complexes}

\bigskip
 
\begin{enumerate}
\item \textbf{Proposition 1 :} Dans le plan muni d'un repère orthonormé, l'ensemble des points $M$ dont l'affixe $z$ vérifie l'égalité $|z - \text{i}| = |z + 1|$ est une droite. 

\medskip

\item \textbf{Proposition 2 :} Le nombre complexe $\left(1 + \text{i}\sqrt{3}\right)^4$ est un nombre réel. 

\medskip

\item Soit ABCDEFGH un cube.\\

\parbox{0.5\linewidth}{ 
\textbf{Proposition 3 :} Les droites (EC) et (BG) sont orthogonales.} \hfill  
\parbox{0.4\linewidth}{\psset{unit=0.7cm}\begin{pspicture}(5.6,5.6)
\psframe(0.6,0.6)(3.7,3.7)%ABFE
\psline(3.7,0.6)(5,2.1)(5,5.2)(3.7,3.7)%BCGF
\psline(5,5.2)(1.9,5.2)(0.6,3.7)%GHE
\psline[linestyle=dotted,linewidth=1.4pt](0.6,0.6)(1.9,2.1)(5,2.1)%ADC
\psline[linestyle=dotted,linewidth=1.4pt](1.9,2.1)(1.9,5.2)%DH
\uput[dl](0.6,0.6){A} \uput[dr](3.7,0.6){B} \uput[r](5,2.1){C} 
\uput[l](1.7,2.1){D} \uput[l](0.6,3.7){E} \uput[r](3.7,3.7){F} 
\uput[ur](5,5.2){G} \uput[ul](1.9,5.2){H} 
\end{pspicture}}

\medskip

\item L'espace est muni d'un repère orthonormé \Oijk. Soit le plan $\mathcal{P}$ d'équation cartésienne $x + y + 3z + 4 = 0$. On note S le point de coordonnées $(1\,, -2\,, - 2)$.
 
\textbf{Proposition 4 :} La droite qui passe par S et qui est perpendiculaire au plan $\mathcal{P}$ a pour représentation paramétrique $\left\{\begin{array}{l @{\;=\;} l}
x&2 + t\\
y& - 1 + t\\
z&1 + 3t
\end{array}\right.$, $t \in \textbf{R}$.  
\end{enumerate}

\vspace{0,5cm}

\newpage

\textbf{\textsc{Exercice 4} \hfill 5 points}
 
\textbf{Candidats n'ayant pas suivi l'enseignement de spécialité }

\medskip

Soit la suite\index{suite} numérique $\left(u_{n}\right)$ définie sur $\textbf{N}$ par : 

\[u_{0} = 2 \quad \text{et pour tout entier naturel } \:n, \:u_{n+1} = \dfrac{2}{3}u_{n} + \dfrac{1}{3}n + 1.\]
 
\begin{enumerate}
\item 
	\begin{enumerate}
		\item Calculer $u_{1}, u_{2}, u_{3}$ et $u_{4}$. On pourra en donner des valeurs approchées à $10^{- 2}$ près. 
		\item Formuler une conjecture sur le sens de variation de cette suite.
	\end{enumerate} 

\item 
	\begin{enumerate}
		\item Démontrer que pour tout entier naturel $n$, 

\[u_{n} \leqslant n + 3.\]
 
		\item Démontrer que pour tout entier naturel $n$, 
 
\[u_{n+1} - u_{n} = \dfrac{1}{3} \left(n + 3 - u_{n}\right).\]
 
		\item En déduire une validation de la conjecture précédente.
	\end{enumerate} 	
\item On désigne par $\left(v_{n}\right)$ la suite\index{suite} définie sur $\textbf{N}$ par $v_{n} = u_{n} - n$. 
	\begin{enumerate}
		\item Démontrer que la suite $\left(v_{n}\right)$ est une suite géométrique de raison $\dfrac{2}{3}$. 
		\item En déduire que pour tout entier naturel $n$,
		
		\[u_{n} = 2\left(\dfrac{2}{3} \right)^n + n\]
		 
		\item Déterminer la limite de la suite $\left(u_{n}\right)$.
	\end{enumerate} 	
\item Pour tout entier naturel non nul $n$, on pose: 
 
\[S_{n} = \sum_{k=0}^n u_{k} = u_{0} + u_{1} + \ldots + u_{n}\quad \text{et}  
\quad T_{n} = \dfrac{S_{n}}{n^2}.\]

	\begin{enumerate}
		\item Exprimer $S_{n}$ en fonction de $n$. 
		\item Déterminer la limite de la suite $\left(T_{n}\right)$.
	\end{enumerate} 
\end{enumerate}

\vspace{0,5cm}

\textbf{\textsc{Exercice 4} \hfill 5 points}
 
\textbf{Candidats ayant  suivi l'enseignement de spécialité }

\medskip

On étudie la population d'une région imaginaire. Le 1\up{er} janvier 2013, cette région comptait \np{250000}~habitants dont 70\,\% résidaient à la campagne et 30\,\% en ville.
 
L'examen des données statistiques recueillies au cours de plusieurs années amène à choisir de modéliser l'évolution de la population pour les années à venir de la façon suivante : 

\setlength\parindent{8mm}
\begin{itemize}
\item[$\bullet~~$] l'effectif de la population est globalement constant,
\item[$\bullet~~$] chaque année, 5\,\% de ceux qui résident en ville décident d'aller s'installer à la campagne et 1\,\% de ceux qui résident à la campagne choisissent d'aller habiter en ville.
\end{itemize}
\setlength\parindent{0mm}

\bigskip
 
Pour tout entier naturel $n$, on note $v_{n}$ le nombre d'habitants de cette région qui résident en ville au 1\up{er}~janvier de l'année $(2013 + n)$ et $c_{n}$ le nombre de ceux qui habitent à la campagne à la même date.

\medskip
 
\begin{enumerate}
\item Pour tout entier naturel $n$, exprimer $v_{n+1}$ et $c_{n+1}$ en fonction de $v_{n}$ et $c_{n}$.  
\item Soit la matrice\index{matrice} $A = \begin{pmatrix}0,95&0,01\\0,05& 0,99\end{pmatrix}$. 

On pose $X = \begin{pmatrix}a\\b\end{pmatrix}$ où $a,\: b$ sont deux réels fixés et $Y = AX$.
 
Déterminer, en fonction de $a$ et $b$, les réels $c$ et $d$ tels que $Y = \begin{pmatrix}c\\d\end{pmatrix}$.
\end{enumerate}
 
Les résultats précédents permettent d'écrire que pour tout entier naturel $n$, 

$X_{n+1} = AX_{n}$ où $X_{n} = \begin{pmatrix}v_{n}\\c_{n}\end{pmatrix}$. On peut donc en déduire que pour tout entier naturel $n,\: X_{n} = A^n X_{0}$. 

\medskip

\begin{enumerate}
\item[\textbf{3.}] Soient les matrices $P = \begin{pmatrix}1&- 1\\5&1\end{pmatrix}$ et $Q = \begin{pmatrix}1&1\\- 5&1\end{pmatrix}$. 
	\begin{enumerate}
		\item Calculer $PQ$ et $QP$. En déduire la matrice $P^{-1}$ en fonction de $Q$. 
		\item Vérifier que la matrice\index{matrice} $P^{-1}AP$ est une matrice diagonale $D$ que l'on précisera. 
		\item Démontrer que pour tout entier naturel $n$ supérieur ou égal à $1$, $A^n = P D^n P^{- 1}$.
	\end{enumerate} 
\item[\textbf{4.}] Les résultats des questions précédentes permettent d'établir que 
 
\[v_{n} = \dfrac{1}{6}\left(1 + 5 \times  0,94^n\right)v_{0} + \dfrac{1}{6}\left(1 - 0,94^n\right)c_{0}.\] 

Quelles informations peut-on en déduire pour la répartition de la population de cette région à long terme ? 
\end{enumerate}
%%%%%%%%%%   fin Métropole juin 2013
\newpage
%%%%%%%%%%   Antilles-Guyane septembre 2013
\hypertarget{Antillessep}{}

\lfoot{\small{Antilles-Guyane}}
\rfoot{\small{11 septembre 2013}}
\renewcommand \footrulewidth{.2pt}
\pagestyle{fancy}
\thispagestyle{empty}
\begin{center}\textbf{Durée : 4 heures}

\vspace{0,25cm}

{\Large\textbf{Baccalauréat S Antilles-Guyane 11 septembre 2013}}
\end{center}

\vspace{0,225cm}

\textbf{\textsc{Exercice 1} \hfill 5 points}
 
\textbf{Commun à tous les candidats}

\medskip

\textbf{Partie A}

\medskip
 
\textbf{Restitution organisée de connaissances}

\medskip
 
Soit $\Delta$ une droite de vecteur directeur $\vect{v}$ et soit P un plan.
 
On considère deux droites sécantes et contenues dans P : la droite D$_{1}$ de vecteur directeur $\vect{u_{1}}$ et la droite D$_{2}$ de vecteur directeur $\vect{u_{2}}$.
 
Montrer que $\Delta$ est orthogonale à toute droite de P si et seulement si $\Delta$ est orthogonale à D$_{1}$ et à D$_{2}$.

\bigskip

\index{géométrie dans l'espace}
 
\textbf{Partie B}

\medskip
 
Dans l'espace muni d'un repère orthonormé, on considère les trois points 

\[\text{A}(0~;~- 1~;~1),\quad  \text{B}(4~;~-3~;~0)\:\: \text{et}\:\: \text{C}(- 1~;~-2~;~-1).\]
 
On appelle P le plan passant par A, B et C.
 
 
On appelle $\Delta$ la droite ayant pour représentation paramétrique  $\left\{\begin{array}{l c l}
x &=& t\\y &=& 3t - 1\\z &=& -2t + 8
\end{array}\right.$ avec $t$ appartenant à $\R$.
  
Pour chacune des affirmations suivantes, indiquer si elle est vraie ou fausse et justifier la réponse.

\medskip
 
\begin{enumerate}
\item \textbf{Affirmation 1} : $\Delta$ est orthogonale à toute droite du plan P. 
\item \textbf{Affirmation 2} : les droites $\Delta$ et (AB) sont coplanaires. 
\item \textbf{Affirmation 3} : Le plan P a pour équation cartésienne $x + 3y - 2z + 5 = 0$. 
\item On appelle D la droite passant par l'origine et de vecteur directeur $\vect{u}(11~;~- 1~;~4)$.
 
\textbf{Affirmation 4} : La droite D est strictement parallèle au plan d'équation $x + 3y - 2z + 5 = 0$. 
\end{enumerate}

\vspace{0,5cm}

\textbf{\textsc{Exercice 2} \hfill 6 points}
 
\textbf{Commun à tous les candidats}

\medskip
 
Pour tout réel $k$ strictement positif, on désigne par $f_{k}$ la fonction définie et dérivable sur l'ensemble des nombres réels $\R$ telle que : 

\[f_{k}(x) = kx\text{e}^{-kx}.\]
 
On note $\mathcal{C}_{k}$ sa courbe représentative dans le plan muni d'un repère orthogonal \Oij.

\medskip
 
\textbf{Partie A : Étude du cas}\boldmath $k = 1$\unboldmath 

\medskip

On considère donc la fonction $f_{1}$ définie sur $\R$ par 

\[f_{1}(x) = x\text{e}^{- x}.\] 

\begin{enumerate}
\item Déterminer les limites de la fonction $f_{1}$ en $- \infty$ et en $+ \infty$. En déduire que la courbe $\mathcal{C}_{1}$ admet une asymptote que l'on précisera. 
\item Étudier les variations de $f_{1}$ sur $\R$ puis dresser son tableau de variation sur $\R$. 
\item Démontrer que la fonction $g_{1}$ définie et dérivable sur $\R$ telle que : 

\[g_{1}(x) = - (x + 1)\text{e}^{- x}\]
 
est une primitive de la fonction $f_{1}$ sur $\R$. 
\item Étudier le signe de $f_{1}(x)$ suivant les valeurs du nombre réel $x$. 
\item Calculer, en unité d'aire, l'aire de la partie du plan délimitée par la courbe $\mathcal{C}_{1}$, l'axe des abscisses et les droites d'équation $x = 0$ et $x = \ln 10$. 
\end{enumerate} 

\bigskip

\textbf{Partie B : Propriétés graphiques} 

\medskip

On a représenté sur le graphique ci-dessous les courbes $\mathcal{C}_{2}$, $\mathcal{C}_{a}$ et $\mathcal{C}_{b}$ où $a$ et $b$ sont des réels strictement positifs fixés et T la tangente à $\mathcal{C}_{b}$ au point O origine du repère.
 
\begin{center}
\psset{unit=7cm,comma=true}
\begin{pspicture*}(-0.25,-0.2)(1.4,0.65)
\psgrid[gridlabels=0pt,subgriddiv=5,gridwidth=0.3pt,subgridwidth=0.15pt,gridcolor=orange,subgridcolor=orange]
\psaxes[linewidth=1.25pt,Dx=0.2,Dy=0.2](0,0)(-0.25,-0.19)(1.39,0.65)
\psplot[plotpoints=5000,linewidth=1.25pt,linecolor=blue]{-0.1}{1.4}{2 x mul  2.71828 x 2 mul exp div}
\psplot[plotpoints=5000,linewidth=1.25pt,linecolor=green]{-0.1}{1.4}{10 x mul  2.71828 x 10 mul exp div}
\psplot[plotpoints=5000,linewidth=1.25pt,linecolor=red]{-0.1}{1.4}{3 x mul  2.71828 x 3 mul exp div}
\psplot{-0.1}{1.4}{3 x mul}
\uput[l](0.17,0.5){$T$}\uput[dl](0.28,0.18){$\mathcal{C}_{a}$}
\uput[d](0.68,0.26){$\mathcal{C}_{b}$}
\uput[ur](0.9,0.3){$\mathcal{C}_{2}$}
\end{pspicture*}
\end{center}
 
\begin{enumerate}
\item Montrer que pour tout réel $k$ strictement positif, les courbes 
$\mathcal{C}_{k}$ passent par un même point.
\item  
	\begin{enumerate}
			\item Montrer que pour tout réel $k$ strictement positif et tout réel $x$ on a 
			
			\[f'_{k}(x) = k(1 - kx)\text{e}^{- kx}.\]
			 
			\item Justifier que, pour tout réel $k$ strictement positif, $f_{k}$ admet un maximum et calculer ce maximum. 
			\item En observant le graphique ci-dessus, comparer $a$ et 2. Expliquer la démarche. 
			\item Écrire une équation de la tangente à $\mathcal{C}_{k}$ au point O origine du repère. 
			\item En déduire à l'aide du graphique une valeur approchée de $b$.
	\end{enumerate} 
\end{enumerate} 

\vspace{0,5cm}

\textbf{\textsc{Exercice 3} \hfill 4 points}
 
\textbf{Commun à tous les candidats}

\medskip

\index{loi normale}

Une entreprise industrielle fabrique des pièces cylindriques en grande quantité. Pour toute pièce prélevée au hasard, on appelle $X$ la variable aléatoire qui lui associe sa longueur en millimètre et $Y$ la variable aléatoire qui lui associe son diamètre en millimètre.
 
On suppose que $X$ suit la loi normale de moyenne $\mu_{1} = 36$ et d'écart-type $\sigma_{1} = 0,2$ et que $Y$ suit 
la loi normale de moyenne $\mu_{2} = 6$ et d'écart-type $\sigma_{2} = 0,05$.

\medskip
 
\begin{enumerate}
\item Une pièce est dite conforme pour la longueur si sa longueur est comprise entre $\mu_{1} - 3\sigma_{1}$ et $\mu_{1} + 3\sigma_{1}$. Quelle est une valeur approchée à $10^{- 3}$ près de la probabilité $p_{1}$ pour qu'une pièce prélevée au hasard soit conforme pour la longueur ?

\medskip 
\parbox{0.65\linewidth}{\item Une pièce est dite conforme pour le diamètre si son diamètre est compris entre 5,88~mm et 6,12~mm. Le tableau donné ci-contre a été obtenu à l'aide d'un tableur. Il indique pour chacune des valeurs de $k$, la probabilité que $Y$ soit inférieure ou égal à cette valeur.
 
Déterminer à $10^{- 3}$ près la probabilité $p_{2}$ pour qu'une pièce prélevée au hasard soit conforme pour le diamètre (on pourra s'aider du tableau ci-contre).}\hfill
\parbox{0.3\linewidth}{$\begin{array}{|c|c|}\hline 
k		& p(Y \leqslant k)\\ \hline 
5,8 	&\np{3,16712}\text{E}-05\\ \hline 
5,82 	&\np{0,000159109}\\ \hline 
5,84 	&\np{0,000687138}\\ \hline 
5,86 	&\np{0,00255513}\\ \hline 
5,88 	&\np{0,008197536}\\ \hline 
5,9 	&\np{0,022750132}\\ \hline 
5,92 	&\np{0,054799292}\\ \hline 
5,94 	&\np{0,11506967} \\ \hline
5,96 	&\np{0,211855399}\\ \hline 
5,98 	&\np{0,344578258}\\ \hline 
6 		&\np{0,5}\\ \hline 
6,02 	&\np{0,655421742}\\ \hline 
6,04 	&\np{0,788144601}\\ \hline 
6,06 	&\np{0,88493033}\\ \hline 
6,08 	&\np{0,945200708}\\ \hline 
6,1 	&\np{0,977249868}\\ \hline 
6,12 	&\np{0,991802464}\\ \hline 
6,14 	&\np{0,99744487}\\ \hline 
6,16 	&\np{0,999312862}\\ \hline 
6,18 	&\np{0,999840891}\\ \hline  
6,2 	&\np{0,999968329}\\ \hline
\end{array}$} 

\item On prélève une pièce au hasard. On appelle $L$ l'évènement \og la pièce est conforme pour la longueur \fg{} et $D$ l'évènement \og la pièce est conforme pour le diamètre \fg. On suppose que les évènements $L$ et $D$ sont indépendants.
	\begin{enumerate}
		\item Une pièce est acceptée si elle est conforme pour la longueur et pour le diamètre.
		 
Déterminer la probabilité pour qu'une pièce prélevée au hasard ne soit pas acceptée (le résultat sera arrondi à $10^{-2}$). 
		\item Justifier que la probabilité qu'elle soit conforme pour le diamètre sachant qu'elle n'est pas conforme pour la longueur, est égale à $p_{2}$.
	\end{enumerate} 
\end{enumerate}

\vspace{0,5cm}

\textbf{\textsc{Exercice 4} \hfill 5 points}
 
\textbf{Candidats n'ayant pas suivi l'enseignement de spécialité }

\begin{center}  \emph{Les deux parties sont indépendantes}\end{center} 

Le robot Tom doit emprunter un pont sans garde-corps de 10 pas de long et de 2 pas de large. Sa démarche est très particulière :

\medskip
\setlength\parindent{6mm}
\begin{itemize}
\item[$\bullet~~$] Soit il avance d'un pas tout droit ; 
\item[$\bullet~~$] Soit il se déplace en diagonale vers la gauche (déplacement équivalent à un pas vers la gauche et un pas tout droit) ; 
\item[$\bullet~~$] Soit il se déplace en diagonale vers la droite (déplacement équivalent à un pas vers la droite et un pas tout droit).
\end{itemize}
\setlength\parindent{0mm}
 
On suppose que ces trois types de déplacement sont aléatoires et équiprobables.

\medskip
 
L'objectif de cet exercice est d'estimer la probabilité $p$ de l'évènement $S$ \og Tom traverse le pont\fg{} c'est-à-dire \og Tom n'est pas tombé dans l'eau et se trouve encore sur le pont au bout de 10 déplacements \fg.

\medskip
 
\textbf{Partie A} : modélisation et simulation

\medskip
 
On schématise le pont par un rectangle dans le plan muni d'un repère orthonormé (O , I, J) comme l'indique la figure ci-dessous. On suppose que Tom se trouve au point de coordonnées (0~;~0) au début de la traversée. On note $(x~;~y)$ les coordonnées de la position de Tom après $x$ déplacements. 

\begin{center}
\psset{unit=1cm}
\begin{pspicture*}(-1.1,-2.2)(10.5,2.3)
\psframe[fillstyle=solid,fillcolor=lightgray](0,-1)(10,1)
\psgrid[gridlabels=0pt,subgriddiv=1,gridwidth=0.2pt,griddots=5]
\psaxes[linewidth=1pt](0,0)(-1.1,-2.2)(10.5,2.3)
\psline[linewidth=1.25pt]{->}(0,0)(1,0)
\psline[linewidth=1.25pt]{->}(0,0)(1,1)
\psline[linewidth=1.25pt]{->}(0,0)(1,-1)
\rput(-0.5,0.4){départ}
\uput[ul](0,0){O}\uput[ur](1,0){I}\uput[ur](0,1){J}
\end{pspicture*} 
\end{center}

\index{algorithme}

On a écrit l'algorithme suivant qui simule la position de Tom au bout de $x$ déplacements :

\begin{center}
\begin{tabular}{|l|}\hline 
$x, y, n$ sont des entiers\\
Affecter à $x$ la valeur 0\\
Affecter à $y$ la valeur 0\\ 
Tant que $y \geqslant - 1$ et $y \leqslant 1$ et $x \leqslant 9$\\ 
\hspace{1.5cm}Affecter à $n$ une valeur choisie au hasard entre $- 1,\: 0$ et $1$\\
\hspace{1.5cm}Affecter à $y$ la valeur $y + n$\\ 
\hspace{1.5cm}Affecter à $x$ la valeur $x + 1$ \\
Fin tant que\\ 
Afficher \og la position de Tom est \fg{} $(x~;~y)$ \\ \hline
\end{tabular}
\end{center} 
 
\begin{enumerate}
\item On donne les couples suivants : $(-1~;~1)$ ; (10~;~0); (2~;~4) ; (10~;~2).
 
Lesquels ont pu être obtenus avec cet algorithme ? Justifier la réponse. 
\item Modifier cet algorithme pour qu'à la place de \og la position de Tom est $(x~;~y)$ \fg, il affiche finalement \og Tom a réussi la traversée\fg{} ou \og Tom est tombé \fg.
\end{enumerate}

\bigskip
 
\textbf{Partie B}

\medskip
 
Pour tout $n$ entier naturel compris entre 0 et 10, on note : 

$A_{n}$ l'évènement \og après $n$ déplacements, Tom se trouve sur un point d'ordonnée $- 1$ \fg.

$B_{n}$ l'évènement \og après $n$ déplacements, Tom se trouve sur un point d'ordonnée 0 \fg.

$C_{n}$ l'évènement \og après $n$ déplacements, Tom se trouve sur un point d'ordonnée 1 \fg.
 
On note $a_{n}, b_{n}, c_{n}$ les probabilités respectives des évènements $A_{n}, B_{n}, C_{n}$.

\medskip
 
\begin{enumerate}
\item Justifier que $a_{0} = 0, b_{0} = 1, c_{0} = 0$. 
\item Montrer que pour tout entier naturel $n$ compris entre $0$ et $9$, on a 
 
\[\left\{\begin{array}{l c l}
a_{n+1} &=& \dfrac{a_{n} + b_{n}}{3}\\ 
b_{n+1} &=& \dfrac{a_{n} + b_{n} + c_{n}}{3}
\end{array}\right.\]

\index{probabilité}

On pourra s'aider d'un arbre pondéré. 
\item Calculer les probabilités $p\left(A_{1}\right),\: p\left(B_{1}\right)$ et $p\left(C_{1}\right)$. 
\item Calculer la probabilité que Tom se trouve sur le pont au bout de deux déplacements. 

\parbox{0.35\linewidth}{\item À l'aide d'un tableur, on a obtenu la feuille de calcul ci-contre qui donne des valeurs approchées de $a_{n},\: b_{n},\: c_{n}$ pour $n$ compris entre 0 et 10.
 
Donner une valeur approchée à $0,001$ près de la probabilité que Tom traverse le pont (on pourra s'aider du tableau ci-contre).} \hfill
\parbox{0.62\linewidth}{\begin{tabularx}{\linewidth}{|*{4}{>{\centering \arraybackslash}X|}}\hline
$n$ &$a_{n}$ &$b_{n}$ &$c_{n}$ \\ \hline
0	&0				&1				&0\\ \hline
1	&\np{0,333333} 	&\np{0,333333} 	&\np{0,333333}\\ \hline 
2 	&\np{0,222222} 	&\np{0,333333} 	&\np{0,222222}\\ \hline 
3 	&\np{0,185185} 	&\np{0,259259} 	&\np{0,185185}\\ \hline 
4 	&\np{0,148148} 	&\np{0,209877} 	&\np{0,148148}\\ \hline
5 	&\np{0,119342} 	&\np{0,168724} 	&\np{0,119342}\\ \hline
6 	&\np{0,096022} 	&\np{0,135802} 	&\np{0,096022}\\ \hline
7 	&\np{0,077275} 	&\np{0,109282} 	&\np{0,077275}\\ \hline 
8 	&\np{0,062186} 	&\np{0,087944} 	&\np{0,062186}\\ \hline 
9 	&\np{0,050043} 	&\np{0,070772} 	&\np{0,050043}\\ \hline 
10 	&\np{0,040272} 	&\np{0,056953} 	&\np{0,040272}\\ \hline 
\end{tabularx}} 
\end{enumerate}

\vspace{0,5cm}

\textbf{\textsc{Exercice 4} \hfill 5 points}
 
\textbf{Candidats ayant  suivi l'enseignement de spécialité}

\medskip

\textbf{Partie A}

\medskip
 
On considère l'algorithme suivant :

\begin{center}
\begin{tabular}{|l|}\hline
A et X sont des nombres entiers\\
Saisir un entier positif A\\
Affecter à X la valeur de A\\ 
Tant que X supérieur ou égal à 26\\
\hspace{1.25cm}Affecter à X la valeur X - 26\\
Fin du tant que\\ 
Afficher X\\ \hline
\end{tabular}
\end{center}
 
\begin{enumerate}
\item Qu'affiche cet algorithme quand on saisit le nombre 3 ? 
\item Qu'affiche cet algorithme quand on saisit le nombre 55 ? 
\item Pour un nombre entier saisi quelconque, que représente le résultat fourni par cet algorithme?
\end{enumerate}

\bigskip
 
\textbf{Partie B}

\medskip
 
On veut coder un bloc de deux lettres selon la procédure suivante (détaillée en quatre étapes) :

\medskip
 
$\bullet~~$\textbf{Étape 1} : chaque lettre du bloc est remplacée par un entier en utilisant le tableau ci-dessous: 

\medskip
\begin{tabularx}{\linewidth}{|*{13}{>{\centering \arraybackslash}X|}}\hline
A	&B	&C	&D	&E	&F	&G	&H	&I	&J	&K	&L	&M\\\hline
0	&1	&2	&3	&4	&5	&6	&7	&8	&9	&10	&11	&12\\\hline \hline
N	&O	&P	&Q	&R	&S	&T	&U	&V	&W	&X	&Y	&Z \\ \hline 
13	&14	&15	&16	&17	&18	&19	&20	&21	&22	&23	&24	&25\\\hline 
\end{tabularx}
\medskip

\index{matrice}

On obtient une matrice colonne $\begin{pmatrix}x_{1}\\x_{2}\end{pmatrix}$ où $x_{1}$ correspond à la première lettre du mot et $x_{2}$ correspond à la deuxième lettre du mot.
 
$\bullet~~$\textbf{Étape 2} : $\begin{pmatrix}x_{1}\\x_{2}\end{pmatrix}$ est transformé en $\begin{pmatrix}y_{1}\\y_{2}\end{pmatrix}$ tel que 

\[\begin{pmatrix}y_{1}\\y_{2}\end{pmatrix} = \begin{pmatrix}3&1\\5&2\end{pmatrix}\begin{pmatrix}x_{1}\\x_{2}\end{pmatrix}\]

La matrice $C = \begin{pmatrix}3&1\\5&2\end{pmatrix}$ est appelée la matrice de codage.
 
$\bullet~~$\textbf{Étape 3} : $\begin{pmatrix}y_{1}\\y_{2}\end{pmatrix}$ est transformé en $\begin{pmatrix}z_{1}\\z_{2}\end{pmatrix}$ tel que
 
\[\left\{\begin{array}{l c l c l c l l}
z_{1}& \equiv& y_{1}\: (26)& \text{avec}\:\: 0 &\leqslant& z_{1}&\leqslant& 25\\
z_{2}& \equiv& y_{2}\: (26)& \text{avec}\:\: 0 &\leqslant& z_{2}&\leqslant& 25
\end{array}\right.\] 

$\bullet~~$\textbf{Étape 4} : $\begin{pmatrix}z_{1}\\z_{2}\end{pmatrix}$ est transformé en un bloc de deux lettres en utilisant le tableau de correspondance donné dans l'étape 1.

\begin{center}
\begin{tabular}{|l} 
\textbf{Exemple} : \\
RE $\to \begin{pmatrix}17\\4\end{pmatrix}\to \begin{pmatrix}55\\93\end{pmatrix} \to \begin{pmatrix}3\\15\end{pmatrix}\to $ DP\\
Le bloc RE est donc codé en DP\\
\end{tabular}
\end{center}
 
Justifier le passage de $\begin{pmatrix}17\\4\end{pmatrix}$ à $\begin{pmatrix}55\\93\end{pmatrix}$ puis à  $\begin{pmatrix}3\\15\end{pmatrix}$.

\medskip

\begin{enumerate}
\item Soient $x_{1},\:x_{2},\:x'_{1},\:x'_{2}$ quatre nombres entiers compris entre 0 et 25 tels que $\begin{pmatrix}x_{1}\\x_{2}\end{pmatrix}$ et $\begin{pmatrix}x'_{1}\\x'_{2}\end{pmatrix}$ sont transformés lors du procédé de codage en $\begin{pmatrix}z_{1}\\z_{2}\end{pmatrix}$. 	
	\begin{enumerate}
		\item Montrer que $\left\{\begin{array}{l c l}
		3x_{1}+ x_{2} & \equiv& 3x'_{1} + x'_{2}  \quad (26)\\
5x_{1}+ 2x_{2}&\equiv&5x'_{1} + 2x'_{2} \quad (26).
\end{array}\right.$ 
		\item En déduire que $x_{1} \equiv  x'_{1}\quad  (26)$ et $x_{2} \equiv  x'_{2} \quad (26)$ puis que $x_{1} = x'_{1}$ et $x_{2} = x'_{2}$.
	\end{enumerate} 
\item On souhaite trouver une méthode de décodage pour le bloc DP : 
	\begin{enumerate}
		\item Vérifier que la matrice $C' = \begin{pmatrix}2&- 1\\- 5&3\end{pmatrix}$ est la matrice inverse de $C$. 
		\item Calculer $\begin{pmatrix}y_{1}\\y_{2}\end{pmatrix}$ tels que $\begin{pmatrix}y_{1}\\y_{2}\end{pmatrix} = \begin{pmatrix}2&- 1\\- 5&3\end{pmatrix}\begin{pmatrix}3\\15\end{pmatrix}$. 
		\item Calculer $\begin{pmatrix}x_{1}\\x_{2}\end{pmatrix}$ 	tels que $\left\{\begin{array}{l c l}
x_{1}&\equiv &y_{1}\quad (26)\:\: \text{avec}\:0 \leqslant x_{1} \leqslant 25\\
x_{2}&\equiv &y_{2}\quad (26)\:\: \text{avec}\:0 \leqslant x_{2} \leqslant 25\\
\end{array}\right.$ 
		\item Quel procédé général de décodage peut-on conjecturer ?
	\end{enumerate} 
\item Dans cette question nous allons généraliser ce procédé de décodage.
 
On considère un bloc de deux lettres et on appelle $z_{1}$ et $z_{2}$ les deux entiers compris entre 0 et 25 associés à ces lettres à l'étape 3. On cherche à trouver deux entiers $x_{1}$ et $x_{2}$ compris 
entre 0 et 25 qui donnent la matrice colonne $\begin{pmatrix}z_{1}\\z_{2}\end{pmatrix}$ par les étapes 2 et 3 du procédé de codage.
 
Soient $y'_{1}$ et $y'_{2}$ tels que $\begin{pmatrix}y'_{1}\\y_{2}\end{pmatrix} = C' \begin{pmatrix}z_{1}\\z_{2}\end{pmatrix}$ où $C'	=  \begin{pmatrix}2&- 1\\- 5&3\end{pmatrix}$.

Soient $x_{1}$ et $x_{2}$, les nombres entiers tels que $\left\{\begin{array}{l c l}
x_{1}&\equiv & y'_{1} \quad (26) \: \text{avec}\:0 \leqslant x_{1}\leqslant 25\\  
x_{2}&\equiv &y'_{2} \quad (26) \: \text{avec}\:0 \leqslant x_{2}\leqslant 25  
\end{array}\right.$

Montrer que $\left\{\begin{array}{l c l}
3x_{1}+ x_{2} & \equiv& z_{1}  \quad (26)\\
5x_{1}+ 2x_{2}&\equiv&z_{2} \quad (26).
\end{array}\right.$.
 
Conclure. 
\item Décoder QC. 
\end{enumerate}
%%%%%%%%%%%   fin Antilles-Guyane septembre 2013
\newpage
%%%%%%%%%%%   Métropole septembre 2013
\hypertarget{Metropolesep}{}

\lfoot{\small{Métropole}}
\rfoot{\small 12 septembre 2013}
\pagestyle{fancy}
\thispagestyle{empty}
\begin{center}

{\Large \textbf{\decofourleft~Baccalauréat S  Métropole 12 septembre 2013~\decofourright}}

 \end{center}

\vspace{0,25cm}

\textbf{\textsc{Exercice 1} \hfill 6 points}
 
\textbf{Commun à tous les candidats}

\medskip

Soit $f$ une fonction définie et dérivable sur $\R$. On note $\mathcal{C}$ sa courbe représentative dans le plan muni d'un repère \Oij.

\medskip
 
\textbf{Partie A}

\medskip
 
Sur les graphiques ci-dessous, on a représenté la courbe $\mathcal{C}$ et trois autres courbes $\mathcal{C}_{1}$, $\mathcal{C}_{2}$, $\mathcal{C}_{3}$ avec la tangente en leur point d'abscisse $0$. 

\begin{center}
\psset{unit=1cm}
\begin{pspicture*}(-7.5,-1)(2,4.5)
\psaxes[linewidth=1.25pt,Dx=10,Dy=10](0,0)(-7.5,-1)(2,4.5)
\psaxes[linewidth=1.25pt]{->}(1,1)
\multido{\n=-7+1}{10}{\psline(\n,0)(\n,0.1)}
\multido{\n=-1+1}{6}{\psline(0,\n)(0.1,\n)}
\psplot[plotpoints=5000,linewidth=1.25pt,linecolor=blue]{-7.5}{1.2}{x 2 add 2.71828 x 2 div exp mul}
\uput[dl](0,0){O}
\uput[d](0.5,0){$\vect{\imath}$}
\uput[l](0,0.5){$\vect{\jmath}$}
\uput[r](1,4.3){$\mathcal{C}$}
\end{pspicture*}

\vspace{0.5cm}

\begin{tabularx}{\linewidth}{|*{3}{>{\centering \arraybackslash}X|}}\hline
\psset{unit=0.8cm}
\begin{pspicture*}(-3,-3.5)(2,1.5)
\psaxes[linewidth=1.25pt,Dx=10,Dy=10](0,0)(-3,-3.5)(2,1.5)
\psaxes[linewidth=1.25pt,Dx=10,Dy=10]{->}(1,1)
\multido{\n=-7+1}{10}{\psline(\n,0)(\n,0.1)}
\multido{\n=-1+1}{6}{\psline(0,\n)(0.1,\n)}
\psplot[plotpoints=5000,linewidth=1.25pt]{-2.5}{1.2}{ x 2 mul  2.71828 x 2 div exp mul 2 sub}
\uput[dl](0,0){O}
\uput[d](0.5,0){$\vect{\imath}$}
\uput[l](0,0.5){$\vect{\jmath}$}
\uput[r](1.4,0.8){$d_{1}$}
\uput[u](-2,-3.3){$\mathcal{C}_{1}$}
\psplot[linestyle=dashed]{-1}{1.5}{2 x mul 2 sub}
\end{pspicture*}&
\begin{pspicture*}(-3,-3.5)(1.8,1.5)
\psaxes[linewidth=1.25pt,Dx=10,Dy=10](0,0)(-3,-3.5)(1.8,1.5)
\psaxes[linewidth=1.25pt,Dx=10,Dy=10]{->}(1,1)
\multido{\n=-7+1}{10}{\psline(\n,0)(\n,0.1)}
\multido{\n=-3+1}{6}{\psline(0,\n)(0.1,\n)}
\uput[dl](0,0){O}
\uput[d](0.5,0){$\vect{\imath}$}
\uput[l](0,0.5){$\vect{\jmath}$}
\uput[r](0.9,1){$d_{2}$}
\uput[u](-2,-1.2){$\mathcal{C}_{2}$}
\psplot[plotpoints=5000,linewidth=1.25pt]{-3}{1}{4 x mul 2 sub 2.71828 x  exp mul}
\psplot[linestyle=dashed]{-1}{1.5}{2 x mul 2 sub}
\end{pspicture*}&\begin{pspicture*}(-3,-2.5)(1.8,2.5)
\psaxes[linewidth=1.25pt,Dx=10,Dy=10](0,0)(-3,-2.5)(1.8,2.5)
\psaxes[linewidth=1.25pt,Dx=10,Dy=10]{->}(1,1)
\multido{\n=-7+1}{10}{\psline(\n,0)(\n,0.1)}
\multido{\n=-3+1}{6}{\psline(0,\n)(0.1,\n)}
\uput[dl](0,0){O}
\uput[d](0.5,0){$\vect{\imath}$}
\uput[l](0,0.5){$\vect{\jmath}$}
\uput[r](1.3,1){$d_{3}$}
\uput[u](1,1.2){$\mathcal{C}_{3}$}
\psplot[plotpoints=5000,linewidth=1.25pt]{-3}{0.9}{3.5 x mul 2 sub 2.71828 x  exp mul 0.5 add}
\psplot[linestyle=dashed]{-1}{1.5}{1.5 x mul 1.5 sub}
\end{pspicture*}\\ \hline
\end{tabularx}
\end{center}

\index{lecture graphique}

\begin{enumerate}
\item Donner par lecture graphique, le signe de $f(x)$ selon les valeurs de $x$. 
\item On désigne par $F$ une primitive de la fonction $f$ sur $\R$.
	\begin{enumerate}
		\item À l'aide de la courbe $\mathcal{C}$, déterminer $F'(0)$ et $F'(- 2)$. 
		\item L'une des courbes $\mathcal{C}_{1}$, $\mathcal{C}_{2}$, $\mathcal{C}_{3}$ est la courbe représentative de la fonction $F$. 

Déterminer laquelle en justifiant l'élimination des deux autres.
	\end{enumerate} 
\end{enumerate}

\bigskip

\textbf{Partie B}

\medskip

Dans cette partie, on admet que la fonction $f$ évoquée dans la \textbf{partie A} est la fonction définie sur $\R$ par 

\[f(x) = (x + 2) \text{e}^{\frac{1}{2}x}.\]

\begin{enumerate}
\item L'observation de la courbe $\mathcal{C}$ permet de conjecturer que la fonction $f$ admet un minimum. 
	\begin{enumerate}
		\item Démontrer que pour tout réel $x,\: f'(x) = \dfrac{1}{2}(x + 4)\text{e}^{\frac{1}{2}x}$. 
		\item En déduire une validation de la conjecture précédente.
	\end{enumerate} 
\item On pose $I =  \displaystyle\int_{0}^1  f(x)\:\text{d}x$. 
	\begin{enumerate}
		\item Interpréter géométriquement le réel $I$.  
		\item Soient $u$ et $v$ les fonctions définies sur $\R$ par $u(x) = x$ et $v(x) = \text{e}^{\dfrac{1}{2}x}$.
		 
Vérifier que $f = 2\left(u'v + uv'\right)$. 
		\item En déduire la valeur exacte de l'intégrale $I$.
	\end{enumerate} 
\item On donne l'algorithme ci-dessous.

\index{algorithme}

\begin{center}
\begin{tabular}{|ll|}\hline
Variables: 	&$k$ et $n$ sont des nombres entiers naturels.\\ 
			&$s$ est un nombre réel.\\ 
Entrée : 	&Demander à l'utilisateur la valeur de $n$.\\ 
Initialisation :& Affecter à $s$ la valeur $0$.\\ 
Traitement :& 	Pour $k$ allant de $0$ à $n - 1$\\ 
	& \hspace{0,5cm}| Affecter à $s$ la valeur $s + \dfrac{1}{n}f\left(\dfrac{k}{n}\right)$.\\ 
	&Fin de boucle.\\ 
Sortie :& 	Afficher $s$.\\ \hline
\end{tabular}
\end{center}
 
On note $s_{n}$ le nombre affiché par cet algorithme lorsque l'utilisateur entre un entier naturel strictement positif comme valeur de $n$. 
	\begin{enumerate}
		\item Justifier que $s_{3}$ représente l'aire, exprimée en unités d'aire, du domaine hachuré sur le graphique ci-dessous où les trois rectangles ont la même largeur.
		
\begin{center}
\psset{xunit=4cm,yunit=4cm}
\begin{pspicture}(-0.5,-0.2)(1.5,1.25)
\psaxes[linewidth=1.25pt,Dy=2]{->}(0,0)(-0.5,-0.2)(1.5,1.25)
\pscurve[linecolor=blue](0,0.5)(0.333,0.65)(0.666,0.9)(1,1.15)
\psframe[fillstyle=vlines](0,0)(0.333,0.5)
\psframe[fillstyle=vlines](0.333,0)(0.666,0.65)
\psframe[fillstyle=vlines](0.666,0)(1,0.9)
\psline[linestyle=dashed](1,0.9)(1,1.15)
\uput[l](0,0.25){1}
\uput[ul](0.333,0.65){$\mathcal{C}$}
\end{pspicture}
\end{center}		 
		\item Que dire de la valeur de $s_{n}$ fournie par l'algorithme proposé lorsque $n$ devient grand ?
	\end{enumerate} 
\end{enumerate} 

\vspace{0,5cm}

\textbf{\textsc{Exercice 2} \hfill 4 points}
 
\textbf{Commun à tous les candidats}

\medskip

\emph{Cet exercice est un questionnaire à choix multiples.}

\medskip
 
Pour chaque question, trois réponses sont proposées et une seule d'entre elles est exacte.
 
\textbf{Le candidat portera sur la copie le numéro de la question suivi de la réponse choisie et justifiera son choix.}

\medskip
 
li est attribué un point par réponse correcte et convenablement justifiée. Une réponse non justifiée ne sera pas prise en compte. Aucun point n'est enlevé en l'absence de réponse ou en cas de réponse fausse.

\index{géométrie dans l'espace}

\medskip
 
Pour les questions 1 et 2, l'espace est muni d'un repère orthonormé \Oijk. 
 
 ,  
La droite $\mathcal{D}$ est définie par la représentation paramétrique $\left\{\begin{array}{l c l}
x	&=	&5 - 2t\\
y	&=	& 1 + 3t\\
z	&=	&4
\end{array}\right.,\: t \in \R.$

\medskip
  
\begin{enumerate}
\item On note $\mathcal{P}$ le plan d'équation cartésienne $3x + 2y + z - 6 = 0$. 
	\begin{enumerate}
		\item La droite $\mathcal{D}$ est perpendiculaire au plan $\mathcal{P}$. 
		\item La droite $\mathcal{D}$ est parallèle au plan $\mathcal{P}$. 
		\item La droite $\mathcal{D}$ est incluse dans le plan $\mathcal{P}$.
	\end{enumerate} 
\item On note $\mathcal{D}'$ la droite qui passe par le point A de coordonnées $(3~;~1~;~1)$ et a pour vecteur directeur $\vect{u} = 2\vect{i} - \vect{j} + 2\vect{k}$. 
	\begin{enumerate}
		\item Les droites $\mathcal{D}$ et $\mathcal{D}'$ sont parallèles. 
		\item Les droites $\mathcal{D}$ et $\mathcal{D}'$ sont sécantes. 
		\item Les droites $\mathcal{D}$ et $\mathcal{D}'$ ne sont pas coplanaires.
	\end{enumerate}
	 
\hspace{-0.8cm}Pour les questions 3 et 4, le plan est muni d'un repère orthonormé direct d'origine O.

\medskip
 
\item Soit $\mathcal{E}$ l'ensemble des points $M$ d'affixe $z$ vérifiant $|z + \text{i}| = |z - \text{i}|$.
	\begin{enumerate}
		\item $\mathcal{E}$ est l'axe des abscisses. 
		\item $\mathcal{E}$ est l'axe des ordonnées. 
		\item $\mathcal{E}$ est le cercle ayant pour centre O et pour rayon 1.
	\end{enumerate} 
\item On désigne par B et C deux points du plan dont les affixes respectives $b$ et $c$ vérifient l'égalité $\dfrac{c}{b} = \sqrt{2}\text{e}^{\text{i}\dfrac{\pi}{4}}$. 
	\begin{enumerate}
		\item Le triangle OBC est isocèle en O. 
		\item Les points O, B, C sont alignés. 
		\item Le triangle OBC est isocèle et rectangle en B.
	\end{enumerate} 
\end{enumerate}

\vspace{0,5cm}

\textbf{\textsc{Exercice 3} \hfill 5 points}
 
\textbf{Commun à tous les candidats}

\medskip
 
Dans une usine, on utilise deux machines A et B pour fabriquer des pièces. 


\medskip

\begin{enumerate}
\item La machine A assure 40\,\% de la production et la machine B en assure 60\,\%.
 
On estime que 10\,\% des pièces issues de la machine A ont un défaut et que 9\,\% des pièces issues de la machine B ont un défaut.
 
On choisit une pièce au hasard et on considère les évènements suivants : 

\index{probabilité}

\setlength\parindent{5mm}
\begin{itemize}
\item $A$ : \og La pièce est produite par la machine A \fg
\item $B$ : \og La pièce est produite par la machine B \fg
\item $D$ : \og La pièce a un défaut \fg.
\item $\overline{D}$,  l'évènement contraire de l'évènement $D$.
\end{itemize}
\setlength\parindent{0mm}
 
	\begin{enumerate}
		\item Traduire la situation à l'aide d'un arbre pondéré. 
		\item Calculer la probabilité que la pièce choisie présente un défaut et ait été fabriquée par la machine A. 
		\item Démontrer que la probabilité $P(D)$ de l'évènement $D$ est égale à $0,094$. 
		\item On constate que la pièce choisie a un défaut.
		 
Quelle est la probabilité que cette pièce provienne de la machine A ?
	\end{enumerate} 
\item On estime que la machine A est convenablement réglée si 90\,\% des pièces qu'elle fabrique sont conformes.
 
On décide de contrôler cette machine en examinant $n$ pièces choisies au hasard ($n$ entier naturel) dans la production de la machine A. On assimile ces $n$ tirages à des tirages successifs indépendants et avec remise.
 
On note $X_{n}$ le nombre de pièces qui sont conformes dans l'échantillon de $n$ pièces, et  $F_{n} = \dfrac{X_{n}}{n}$ la proportion correspondante.
\index{loi binomiale}
	\begin{enumerate}
		\item Justifier que la variable aléatoire $X_{n}$ suit une loi binomiale et préciser ses paramètres. 
		\item Dans cette question, on prend $n = 150$.
		 
Déterminer l'intervalle de \index{fluctuation asymptotique}fluctuation asymptotique $I$ au seuil de 95\,\% de la variable aléatoire $F_{150}$. 
		\item Un test qualité permet de dénombrer $21$ pièces non conformes sur un échantillon de $150$ pièces produites.
		 
Cela remet-il en cause le réglage de la machine ? Justifier la réponse.
	\end{enumerate} 
\end{enumerate}

\vspace{0,5cm}

\textbf{\textsc{Exercice 4} \hfill 5 points}
 
\textbf{Candidats n'ayant pas suivi l'enseignement de spécialité}

\medskip

\index{suite}
 
On considère la suite $\left(u_{n}\right)$ définie sur $\N$ par : 

\[u_{0} = 2\quad  \text{et pour tout entier naturel } n,\: u_{n+1} = \dfrac{u_{n}+ 2}{2u_{n} + 1}.\] 

On admet que pour tout entier naturel $n, u_{n} > 0$.

\medskip
 
\begin{enumerate}
\item 
	\begin{enumerate}
		\item Calculer $u_{1}, u_{2}, u_{3}, u_{4}$. On pourra en donner une valeur approchée à $10^{-2}$ près. 
		\item Vérifier que si $n$ est l'un des entiers 0, 1, 2, 3, 4 alors $u_{n} - 1$ a le même signe que $(- 1)^n$. 
		\item Établir que pour tout entier naturel $n, u_{n+ 1} - 1 = \dfrac{- u_{n} + 1}{ 2u_{n} + 1}$. 
		\item Démontrer par récurrence que pour tout entier naturel $n, \:u_{n} - 1$ a le même signe que $(- 1)^n$	
	\end{enumerate} 
\item Pour tout entier naturel $n$, on pose $v_{n} = \dfrac{u_{n} - 1}{u_{n} + 1}$. 
	\begin{enumerate}
		\item Établir que pour tout entier naturel $n, 
 v_{n+1} = \dfrac{- u_{n} + 1}{3u_{n} + 3}$. 
		\item Démontrer que la suite $\left(v_{n}\right)$ est une suite géométrique de raison $- \dfrac{1}{3}$.
		
En déduire l'expression de $v_{n}$ en fonction de $n$. 
		\item On admet que pour tout entier naturel $n, u_{n} =  \dfrac{1 + v_{n}}{1 - v_{n}}$.
		  
Exprimer $u_{n}$ en fonction de $n$ et déterminer la limite de la suite $\left(u_{n}\right)$.
	\end{enumerate}
\end{enumerate}

\vspace{0,5cm}

\textbf{\textsc{Exercice 4} \hfill 5 points}
 
\textbf{Candidats ayant suivi l'enseignement de spécialité}

\medskip

\emph{Les parties A et B peuvent être traitées indépendamment l'une de l'autre}

\medskip
 
Dans un village imaginaire isolé, une nouvelle maladie contagieuse mais non mortelle a fait son apparition.
 
Rapidement les scientifiques ont découvert qu'un individu pouvait être dans l'un des trois états suivants :

\setlength\parindent{6mm} 
\begin{description}
\item[ ] $S$ : \og l'individu est sain, c'est-à-dire non malade et non infecté \fg, 
\item[ ] $I$ : \og l'individu est porteur sain, c'est-à-dire non malade mais infecté \fg, 
\item[ ] $M$ : \og l'individu est malade et infecté \fg.
\end{description}
\setlength\parindent{0mm} 

\bigskip
 
\textbf{Partie A}

\medskip

\index{graphe}

Les scientifiques estiment qu'un seul individu est à l'origine de la maladie sur les $100$~personnes que compte la population et que, d'une semaine à la suivante, un individu change d'état suivant le processus suivant :

\setlength\parindent{6mm} 
\begin{itemize}
\item parmi les individus sains, la proportion de ceux qui deviennent porteurs sains est égale à $\dfrac{1}{3}$ et la proportion de ceux qui deviennent malades est égale à $\dfrac{1}{3}$, 
\item parmi les individus porteurs sains, la proportion de ceux qui deviennent malades est égale à $\dfrac{1}{2}$.
\end{itemize}
\setlength\parindent{0mm} 
 
La situation peut être représentée par un \index{graphe} graphe probabiliste comme ci-dessous.

\begin{center}
\psset{unit=1cm}
\begin{pspicture}(6,4)
%\psgrid
\cnodeput(3,3){S}{$S$}
\cnodeput(1,1){I}{$I$}
\cnodeput(5,1){M}{$M$}
\nccircle{->}{S}{0.4cm}
\nccircle[angleA=-180]{<-}{I}{0.4cm}
\nccircle[angleA=-180]{->}{M}{0.4cm}
\ncarc{<-}{I}{S}\rput(1.55,2.2){$\dfrac{1}{3}$}
\ncarc{->}{S}{M}\rput(4.45,2.2){$\dfrac{1}{3}$}
\ncarc{->}{I}{M}\rput(3,0.7){$\dfrac{1}{2}$}
\rput(3.8,3.4){$\dfrac{1}{3}$}\rput(5.6,0.4){1} \rput(0.4,0.4){$\dfrac{1}{2}$}
\end{pspicture}
\end{center}
 
On note $P_{n} = \left(s_{n}\quad i_{n}\quad m_{n}\right)$ la matrice ligne donnant l'état probabiliste au bout de $n$ semaines où $s_{n}, i_{n}$ et $m_{n}$désignent respectivement la probabilité que l'individu soit sain, porteur sain ou malade la $n$-ième semaine. 

On a alors $P_{0} = (0,99\quad 0\quad 0,01)$ et pour tout entier naturel $n$, 

\[\renewcommand\arraystretch{1.7}
\left\{\begin{array}{l c l}
s_{n+1}&=&\dfrac{1}{3}s_{n}\\
i_{n+1}&=&\dfrac{1}{3}s_{n} + \dfrac{1}{2}i_{n}\\
m_{n+1}&=&\dfrac{1}{3}s_{n} + \dfrac{1}{2}i_{n} + m_{n}
\end{array}\right.\]

\begin{enumerate}
\item Écrire la matrice $A$ appelée \emph{matrice de transition}, telle que pour tout entier naturel $n,\:$

$P_{n+1} = P_{n} \times A$.
\item Démontrer par récurrence que pour tout entier naturel $n$ non nul, 
$P_{n} = P_{0} \times A^n$.
\item Déterminer l'état probabiliste $P_{4}$ au bout de quatre semaines. On pourra arrondir les valeurs à $10^{- 2}$.
 
Quelle est la probabilité qu'un individu soit sain au bout de quatre semaines ? 
\end{enumerate}

\bigskip

\textbf{Partie B}

\medskip

La maladie n'évolue en réalité pas selon le modèle précédent puisqu'au bout de 4 semaines de recherche, les scientifiques découvrent un vaccin qui permet d'enrayer l'endémie et traitent immédiatement l'ensemble de la population.
 
L'évolution hebdomadaire de la maladie après vaccination est donnée par la matrice de transition : 

\renewcommand\arraystretch{1.7}
\[B = \begin{pmatrix}\dfrac{5}{12}&\dfrac{1}{4} &\dfrac{1}{3}\\
\dfrac{5}{12}&\dfrac{1}{4} &\dfrac{1}{3}\\
\dfrac{1}{6}&\dfrac{1}{2}&\dfrac{1}{3}
\end{pmatrix}.\]
\renewcommand\arraystretch{1}

On note $Q_{n}$ la matrice ligne donnant l'état probabiliste au bout de $n$ semaines après la mise en place de ces nouvelles mesures de vaccination. Ainsi, $Q_{n} = \left(S_{n}\quad I_{n}\quad M_{n}\right)$ où $S_{n},\: I_{n}$ et $M_{n}$ désignent respectivement la  probabilité que l'individu soit sain, porteur sain et malade la $n$-ième semaine après la vaccination. 

Pour tout  entier naturel $n$, on a alors $Q_{n+1}  = Q_{n} \times B$.

D'après la partie A, $Q_{0}  = P_{4}$. Pour la suite, on prend $Q_{0} = (0,01\quad  0,10\quad 0,89)$ où les coefficients ont été 
arrondis à $10^{–2}$.

\medskip
 
\begin{enumerate}
\item Exprimer $S_{n+1}, I_{n+1}$ et $M_{n+1}$ en fonction de $S_{n},\: I_{n}$ et $M_{n}$.
\item Déterminer la constante réelle $k$ telle que $B^2  = kJ$ où $J$ est la matrice carrée d'ordre 3 dont tous les coefficients sont égaux à 1.
 
On en déduit que pour tout entier $n$ supérieur ou égal à 2,  $B^n  = B^2$.
\item 
	\begin{enumerate}
		\item Démontrer que pour tout entier $n$ supérieur ou égal à 2, $Q_{n} = \left(\dfrac{1}{3}\quad \dfrac{1}{3}\quad 
\dfrac{1}{3}\right)$. 
		\item Interpréter ce résultat en terme d'évolution de la maladie. 
		
Peut-on espérer éradiquer la maladie grâce au vaccin ?
	\end{enumerate}
\end{enumerate}
%%%%%%%%%%%   fin Métropole septembre 2013
\newpage
%%%%%%%%%%%   Nouvelle-Calédonie novembre 2013
\hypertarget{Caledonienov}{}

\lfoot{\small{Nouvelle-Calédonie}}
\rfoot{\small{14 novembre 2013}}
\marginpar{\rotatebox{90}{\textbf{A. P. M. E. P.}}}
\renewcommand \footrulewidth{.2pt}
\pagestyle{fancy}
\thispagestyle{empty}

\begin{center}\textbf{Durée : 4 heures}

\vspace{0,5cm}

{\Large\textbf{\decofourleft~Baccalauréat S  Nouvelle-Calédonie~\decofourright\\14 novembre 2013}}
\end{center}

\vspace{0,5cm}

\textbf{\textsc{Exercice 1}\hfill 5 points}

\textbf{Commun à tous les candidats}

\medskip

Soit $f$ la fonction dérivable, définie sur l'intervalle $]0~;~ +\infty[$ par 

\[f(x) = \text{e}^x + \dfrac{1}{x}.\] 
 
\begin{enumerate}
\item \textbf{Étude d'une fonction auxiliaire} 
	\begin{enumerate}
		\item Soit la fonction $g$ dérivable, définie sur $[0~;~ +\infty[$ par 
		
\[g(x) = x^2\text{e}^x - 1.\]
		 
Étudier le sens de variation de la fonction $g$. 
		\item Démontrer qu'il existe un unique réel $a$ appartenant à $[0~;~ +\infty[$ tel que $g(a) = 0$.
		 
Démontrer que $a$ appartient à l'intervalle [0,703~;~0,704[. 
		\item Déterminer le signe de $g(x)$ sur $[0~;~ +\infty[$.
	\end{enumerate} 
\item \textbf{Étude de la fonction } \boldmath $f$ \unboldmath 
	\begin{enumerate}
		\item Déterminer les limites de la fonction $f$ en $0$ et en $+ \infty$. 
		\item On note $f'$ la fonction dérivée de $f$ sur l'intervalle $]0~;~ +\infty[$.
		 
Démontrer que pour tout réel strictement positif $x,\: f'(x) =  \dfrac{g(x)}{x^2}$. 
		\item En déduire le sens de variation de la fonction $f$ et dresser son tableau de variation sur l'intervalle $]0~;~ +\infty[$. 
		\item Démontrer que la fonction $f$ admet pour minimum le nombre réel 
		
$m = \dfrac{1}{a^2} + \dfrac{1}{a}$. 
		\item Justifier que $3,43 < m < 3,45$.
	\end{enumerate} 
\end{enumerate}

\vspace{0,5cm}

\textbf{\textsc{Exercice 2}\hfill 5 points}

\textbf{Commun à tous les candidats}

\medskip

Soient deux suites $\left(u_{n}\right)$ et $\left(v_{n}\right)$ définies par $u_{0} = 2$ et $v_{0} = 10$ et pour tout entier naturel $n$,\index{suite}
  		 
\[u_{n+1} = \dfrac{2u_{n} + v_{n}}{3}	\quad \text{et}\quad	 v_{n+1} = 	\dfrac{u_{n} + 3v_{n}}{4}.\]
 
\textbf{PARTIE A}

\medskip
 
On considère l'algorithme suivant :\index{algorithme}

\begin{center}
\begin{tabularx}{0.5\linewidth}{|lX|}\hline 
\textbf{Variables :}& 	$N$ est un entier\\ 
&$U, V, W$ sont des réels\\ 
&$K$ est un entier \\
\textbf{Début :}& Affecter $0$ à $K$\\ 
&Affecter 2 à $U$ \\
&Affecter 10 à $V$\\ 
&Saisir $N$\\ 
&Tant que $K < N$\\
&\hspace{0,6cm} Affecter $K + 1$ à $K$\\
&\hspace{0,6cm} Affecter $U$ à $W$\\ 
&\hspace{0,6cm} Affecter $\dfrac{2U+V}{3}$ à $U$\\ 
&\hspace{0,6cm}	Affecter $\dfrac{W+3V}{4}$	à $V$\\  
&Fin tant que\\ 
&Afficher $U$ \\
&Afficher $V$\\ 
\textbf{Fin}&\\ \hline
\end{tabularx}
\end{center}
 
On  exécute cet algorithme en saisissant $N = 2$. Recopier et compléter le tableau donné ci-dessous donnant l'état des variables au cours de l'exécution de l'algorithme.

\begin{center}
\definecolor{gristab}{gray}{0.80}
\begin{tabularx}{0.6\linewidth}{|*{4}{>{\centering \arraybackslash}X|}}\hline  
$K$& $W$&$U$&$V$\\ \hline 
0&\multicolumn{1}{>{\columncolor{gristab}}X|}{\quad}&&\\ \hline 
1&&&\\ \hline
2&&&\\ \hline
\end{tabularx}
\end{center}

\textbf{PARTIE  B}

\medskip
 
\begin{enumerate}
\item 
	\begin{enumerate}
		\item Montrer que pour tout entier naturel $n,\: v_{n+1} - u_{n+1} = \dfrac{5}{12} \left(v_{n} - u_{n}\right)$. 
		\item Pour tout entier naturel $n$ on pose $w_{n} = v_{n} - u_{n}$.
		 
Montrer que pour tout entier naturel $n,\: w_{n} = 8 \left(\dfrac{5}{12} \right)^n$.
	\end{enumerate} 
\item 
	\begin{enumerate}
		\item Démontrer que la suite $\left(u_{n}\right)$ est croissante et que la suite $\left(v_{n}\right)$ est décroissante. 
		\item Déduire des résultats des questions 1. b. et 2. a. que pour tout entier naturel $n$ on a $u_{n} \leqslant 10$ et $v_{n} \geqslant 2$. 
		\item En déduire que tes suites $\left(u_{n}\right)$ et $\left(v_{n}\right)$ sont convergentes.
	\end{enumerate} 
\item Montrer que les suites $\left(u_{n}\right)$ et $\left(v_{n}\right)$ ont la même limite. 
\item Montrer que la suite $\left(t_{n}\right)$ définie par $t_{n} = 3u_{n} + 4v_{n}$ est constante. 

En déduire que la limite  commune des suites $\left(u_{n}\right)$ et $\left(v_{n}\right)$ est $\dfrac{46}{7}$. 
\end{enumerate}

\vspace{0,5cm}

\textbf{\textsc{Exercice 3}\hfill 5 points}

\textbf{Commun à tous les candidats}

\medskip
 
\emph{Tous les résultats numériques devront être donnés sous forme décimale et arrondis au dix-millième} 

\medskip

Une usine fabrique des billes sphériques dont le diamètre est exprimé en millimètres.
 
Une bille est dite hors norme lorsque son diamètre est inférieur à 9~mm ou supérieur à 11~mm.

\bigskip
 
\textbf{Partie A}

\medskip
 
\begin{enumerate}
\item On appelle $X$ la variable aléatoire qui à chaque bille choisie au hasard dans la production associe son diamètre exprimé en mm.
 
On admet que la variable aléatoire $X$ suit la loi normale \index{loi normale} d'espérance $10$ et d'écart-type $0,4$. 

Montrer qu'une valeur approchée à \np{0,0001} près de la probabilité qu'une bille soit hors norme est \np{0,0124}. On pourra utiliser la table de valeurs donnée en annexe. 
\item On met en place un contrôle de production tel que 98\,\% des billes hors norme sont écartés et 99\,\% des billes correctes sont conservées.
 
On choisit une bille au hasard dans la production. On note $N$ l'évènement : \og la bille choisie est aux normes \fg, $A$ l'évènement : \og la bille choisie est acceptée à l'issue du contrôle \fg. 
	\begin{enumerate}
		\item Construire un arbre pondéré qui réunit les données de l'énoncé. 
		\item Calculer la probabilité de l'évènement $A$. 
		\item Quelle est la probabilité pour qu'une bille acceptée soit hors norme ?
	\end{enumerate}
\end{enumerate}
 
\bigskip
 
\textbf{Partie B}

\medskip

Ce contrôle de production se révélant trop coûteux pour l'entreprise, il est abandonné : dorénavant, toutes les billes produites sont donc conservées, et elles sont conditionnées par sacs de 100 billes. 

On considère que la probabilité qu'une bille soit hors norme est de \np{0,0124}.
 
On admettra que prendre au hasard un sac de $100$~billes revient à effectuer un tirage avec remise de $100$~billes dans l'ensemble des billes fabriquées.
 
On appelle $Y$ la variable aléatoire qui à tout sac de $100$~billes associe le nombre de billes hors norme de ce sac.

\medskip
 
\begin{enumerate}
\item Quelle est la loi suivie par la variable aléatoire $Y$ ? 
\item Quels sont l'espérance et l'écart-type de la variable aléatoire $Y$ ? 
\item Quelle est la probabilité pour qu'un sac de $100$~billes contienne exactement deux billes hors norme ? 
\item Quelle est la probabilité pour qu'un sac de $100$~billes contienne au plus une bille hors norme ? 
\end{enumerate}

\vspace{0,5cm}

\textbf{\textsc{Exercice 4}\hfill 5 points}

\textbf{Pour les candidats n'ayant pas suivi l'enseignement de spécialité}

\medskip
 
Le plan est rapporté à un repère orthonormal direct \Ouv.

On note $\C$ l'ensemble des nombres complexes.

\medskip
 
Pour chacune des propositions suivantes, dire si elle est vraie ou fausse en justifiant la réponse.\index{complexes}

\bigskip
 
\begin{enumerate}
\item \textbf{Proposition} : Pour tout entier naturel $n :\: (1 + \text{i})^{4n} = (- 4)^n$. 
\item Soit (E) l'équation $(z - 4)\left(z^2 - 4z + 8\right) = 0$ où $z$ désigne un nombre complexe.
 
\textbf{Proposition} : Les points dont les affixes sont les solutions, dans $\C$, de (E) sont les sommets d'un triangle d'aire 8. 
\item \textbf{Proposition} : Pour tout nombre réel $\alpha,\: 1 + \text{e}^{2i\alpha} = 2\text{e}^{\text{i}\alpha} \cos(\alpha)$. 
\item Soit A le point d'affixe $z_{\text{A}} = \dfrac{1}{2}(1 + \text{i})$ et $M_{n}$ le point d'affixe $\left(z_{\text{A}}\right)^n$ où $n$ désigne un entier  naturel supérieur ou égal à $2$.
 
\textbf{Proposition}: si $n - 1$ est divisible par 4, alors les points O, A et $M_{n}$ sont alignés. 
\item Soit j le nombre complexe de module 1 et d'argument $\dfrac{2\pi}{3}$. 

\textbf{Proposition} : $1 + \text{j} + \text{j}^2 = 0$. 
\end{enumerate}

\vspace{0,5cm}

\textbf{\textsc{Exercice 4}\hfill 5 points}

\textbf{Pour les candidats ayant suivi l'enseignement de spécialité}

\medskip
 
On note $E$ l'ensemble des vingt-sept nombres entiers compris entre $0$ et $26$.\index{codage}
 
On note $A$ l'ensemble dont les éléments sont les vingt-six lettres de l'alphabet et un séparateur entre deux mots, noté \og $\star$ \fg{} considéré comme un caractère.
 
Pour coder les éléments de $A$, on procède de la façon suivante :

\medskip
 
$\bullet~~$Premièrement : On associe à chacune des lettres de l'alphabet, rangées par ordre alphabétique, un nombre entier naturel compris entre 0 et 25, rangés par ordre croissant. On a donc $a \to 0,\: b \to 1, \ldots z \to 25$.
 
On associe au séparateur \og $\star$ \fg le nombre 26.

\begin{center}
\begin{tabularx}{0.9\linewidth}{|*{14}{>{\centering \arraybackslash}X|}}\hline 
$a$&$b$&$c$&$d$&$e$&$f$&$g$&$h$&$i$&$j$&$k$&$l$&$m$&$n$\\ \hline
0&1&2&3&4&5&6&7&8&9&10&11&12&13\\ \hline
\end{tabularx}

\medskip \medskip
\begin{tabularx}{0.9\linewidth}{|*{13}{>{\centering \arraybackslash}X|}X}\cline{1-13}
$o$&$p$&$q$&$r$&$s$&$t$&$u$&$v$&$w$&$x$&$y$&$z$&$\star$&\\ \cline{1-13} 
14&15&13&17&18&19&20&21&22&23&24&25&26&\\ \cline{1-13}
\end{tabularx}
\end{center}

On dit que $a$ a pour rang $0, b$ a pour rang 1, ... , $z$ a pour rang $25$ et le séparateur \og $\star$ \fg{} a pour rang $26$.
 
$\bullet~~$Deuxièmement : à chaque élément $x$ de $E$, l'application $g$ associe le reste de la division euclidienne de $4x + 3$ par $27$.
 
On remarquera que pour tout $x$ de $E,\: g(x)$ appartient à $E$.
 
$\bullet~~$Troisièmement : Le caractère initial est alors remplacé par le caractère de rang $g(x)$.
 
Exemple :
 
$s \to 18, \quad  g(18) = 21$ et $21 \to v$. Donc la lettre $s$ est remplacée lors du codage par la lettre $v$.

\bigskip
 
\begin{enumerate}
\item Trouver tous les entiers $x$ de $E$ tels que $g(x) = x$ c'est-à-dire invariants par $g$.

En déduire les caractères invariants dans ce codage. 
\item Démontrer que, pour tout entier naturel $x$ appartenant à $E$ et tout entier naturel $y$ appartenant à $E$, si $y \equiv 4x + 3$  modulo 27 alors $x \equiv 7y + 6$ modulo 27.
 
En déduire que deux caractères distincts sont codés par deux caractères distincts. 
\item Proposer une méthode de décodage. 
\item Décoder le mot \og $vfv$ \fg. 
\end{enumerate}

\newpage
\begin{center}
{\large \textbf{Annexe\\ Exercice 3} }

\vspace{2cm}

\begin{tabularx}{0.4\linewidth}{|c|*{2}{>{\centering \arraybackslash}X|}}\hline 
&A&B\\ \hline
1&$d$	&$P(X < d)$\\ \hline
2&0		&3,06E-138\\ \hline 
3&1		&2,08E-112\\ \hline  
4&2		&2,75E-89\\ \hline 
5&3		&7,16E-69\\ \hline 
6&4		&3,67E-51\\ \hline 
7&5		&3,73E-36\\ \hline
8&6		&7,62E-24\\ \hline
9&7		&3,19E-14\\ \hline 
10&8	&2,87E-07\\ \hline
11&9  	&0,00620967\\ \hline 
12&10	&0,5\\ \hline 
13&11	&0,99379034\\ \hline 
14&12	& 0,99999971\\ \hline 
15&13 	&1\\ \hline 
16&14	& 1\\ \hline 
17&15	&1\\ \hline 
18&16	&1 \\ \hline
19&17	&1\\ \hline 
20&18	&1\\ \hline 
21&19 	&1\\ \hline 
22&20	&1\\ \hline 
23&21	&1\\ \hline 
24&22	&1\\ \hline 
25&	&\\ \hline 
\end{tabularx}

\emph{Copie d'écran d'une feuille de calcul} 
\end{center}
%%%%%%%%%%%%%   fin Nouvelle-Calédonie novembre 2013
\newpage
%%%%%%%%%%%%%   Amérique du Sud novembre 2013
\hypertarget{AmeriSud}{}

\lfoot{\small{Amérique du Sud}}
\rfoot{\small 21  novembre 2013}
\pagestyle{fancy}
\thispagestyle{empty}
\begin{center}\textbf{Durée : 4 heures}

\vspace{0,25cm}

{\Large\textbf{\decofourleft~Baccalauréat S Amérique du Sud
~\decofourright\\21 novembre 2013}}
\end{center}

\vspace{0,25cm}

\textbf{Exercice 1 \hfill 6 points}

\textbf{Commun à tous les candidats}

\medskip
 
\textbf{Partie A}

\medskip
 
Soit $f$ la fonction définie sur $\R$ par 

\[f(x) = x \text{e}^{1-x}.\]
 
\begin{enumerate}
\item Vérifier que pour tout réel $x,\: f(x)= \text{e} \times \dfrac{x}{\text{e}^x}$. 
\item Déterminer la limite de la fonction $f$ en $- \infty$. 
\item Déterminer la limite de la fonction $f$ en $+ \infty$. Interpréter graphiquement cette limite. 
\item Déterminer la dérivée de la fonction $f$. 
\item Étudier les variations de la fonction $f$ sur $\R$ puis dresser le tableau de variation.
\end{enumerate}

\bigskip
 
\textbf{Partie B}

\medskip
 
Pour tout entier naturel $n$ non nul, on considère les fonctions $g_{n}$ et $h_{n}$ définies sur $\R$ par : 

\[g_{n}(x) = 1 + x + x^2 + \cdots + x^n \quad \text{et}\quad  h_{n}(x) = 1 + 2x + \cdots  + nx^{n-1}.\]
 
\begin{enumerate}
\item Vérifier que, pour tout réel $x :\: (1 - x)g_{n}(x) = 1 - x^{n+1}$. 

On obtient alors, pour tout réel $x \neq 1 :\:\: g_{n}(x) = \dfrac{1 - x^{n+1}}{1 - x}$. 
\item Comparer les fonctions $h_{n}$ et $g'_{n}$, $g'_{n}$ étant la dérivée de la fonction $g_{n}$. 
 	 
En déduire que, pour tout réel $x \neq 1 :\: h_{n}(x) = \dfrac{nx^{n+1} -(n+1)x^n + 1}{(1-x)^2}$.
 
\item Soit $S_{n} = f(1) + f(2) + ... + f(n)$, $f$ étant la fonction définie dans la partie A.
 
En utilisant les résultats de la \textbf{partie B}, déterminer une expression de $S_{n}$ puis sa limite quand $n$ tend vers $+ \infty$. 
\end{enumerate}

\vspace{0,5cm}

\textbf{Exercice 2 \hfill 4 points}

\textbf{Commun à tous les candidats}

\medskip

On considère le cube ABCDEFGH, d'arête de longueur 1, représenté ci-dessous et on munit l'espace du repère orthonormé $\left(\text{A}~;~\vect{\text{AB}},~\vect{\text{AD}},~\vect{\text{AE}}\right)$.\index{géométrie dans l'espace} 

\begin{figure}
\begin{center}
\psset{unit=0.9cm}
\begin{pspicture}(10,10.2)
\pspolygon(0.9,1.3)(7,0.4)(9.2,3)(9.2,8.9)(3.2,9.9)(0.9,7.4)%BCDHEF
\psline(0.9,7.4)(7,6.4)(9.2,8.9)%FGH
\psline(7,6.4)(7,0.4)%GC
\psline(3.2,9.9)(7,6.4)(0.9,1.3)%EGB
\psline[linestyle=dashed](0.9,1.3)(3.2,9.9)%GBE
\psline[linestyle=dashed](0.9,7.4)(9.2,3)%FD
\psline[linestyle=dashed](0.9,1.3)(3,4.2)(3.2,9.9)%BAE
\psline[linestyle=dashed](3,4.2)(9.2,3)%AD
\uput[dl](0.9,1.3){B} \uput[dr](7,0.4){C} \uput[r](9.2,3){D} \uput[ul](3,4.2){A} 
\uput[ul](0.9,7.4){F} \uput[dr](7,6.4){G} \uput[ur](9.2,8.9){H} \uput[ul](3.2,9.9){E} 
\uput[u](3.6,5.95){K}\psdots(3.6,5.95)  
\end{pspicture}
\end{center}
\end{figure}

\medskip
 
\begin{enumerate}
\item Déterminer une représentation paramétrique de la droite (FD). 
\item Démontrer que le vecteur $\vect{n}\begin{pmatrix}1\\- 1\\1\end{pmatrix}$ est un vecteur normal au plan (BGE) et déterminer une équation du plan (BGE). 
\item Montrer que la droite (FD) est perpendiculaire au plan (BGE) en un point K de coordonnées K$\left(\frac{2}{3}~;~\frac{1}{3}~;~\frac{2}{3}\right)$. 
\item Quelle est la nature du triangle BEG ? Déterminer son aire. 
\item En déduire le volume du tétraèdre BEGD. 
\end{enumerate}

\vspace{0,5cm}

\textbf{Exercice 3 \hfill 5 points}

\textbf{Candidats n'ayant pas suivi l'enseignement de spécialité}

\medskip
 
Le plan complexe est rapporté à un repère orthonormé direct.
 
On considère l'équation \index{complexes} 

\[(E) :\quad  z^2 - 2z\sqrt{3} + 4 = 0.\]
 
\begin{enumerate}
\item Résoudre l'équation $(E)$ dans l'ensemble $\C$ des nombres complexes. 
\item On considère la suite $\left(M_{n}\right)$ des points d'affixes $z_{n} = 2^n \text{e}^{\text{i}(- 1)^n\frac{\pi}{6}}$, définie pour $n \geqslant 1$. 
	\begin{enumerate}
		\item Vérifier que $z_{1}$ est une solution de $(E)$. 
		\item Écrire $z_{2}$ et $z_{3}$ sous forme algébrique. 
		\item Placer les points $M_{1},\: M_{2},\: M_{3}$ et $M_{4}$ sur la figure donnée en annexe et tracer, sur la figure donnée en annexe, les segments $\left[M_{1}, M_{2}\right],\: \left[M_{2}, M_{3}\right]$ et $\left[M_{3}, M_{4}\right]$.
	\end{enumerate} 
\item Montrer que, pour tout entier $n \geqslant 1$,\: $z_{n} = 2^n \left(\dfrac{\sqrt{3}}{2} + \dfrac{(- 1)^n \text{i}}{2}\right)$. 
\item Calculer les longueurs $M_{1}M_{2}$ et $M_{2}M_{3}$.

\medskip

Pour la suite de l'exercice, on admet que, pour tout entier $n \geqslant 1$,\: $M_{n}M_{n+1} = 2^n \sqrt{3}$. 
\item On note $\ell^n = M_{1}M_{2} + M_{2}M_{3} + \cdots 	+ M_{n}M_{n+1}$.
	\begin{enumerate}
		\item Montrer que, pour tout entier $n \geqslant 1,\: \ell^n = 2\sqrt{3}\left(2^n - 1\right)$. 
		\item Déterminer le plus petit entier $n$ tel que $\ell^n \geqslant  \np{1000}$.
	\end{enumerate} 
\end{enumerate}

\newpage

\begin{center}
\textbf{\large ANNEXE}

\textbf{À rendre avec la copie}

\vspace{2cm}

\textbf{Exercice 3 : Candidats n'ayant pas suivi l'enseignement de spécialité}

\vspace{3cm}

\psset{unit=0.66cm}
\begin{pspicture}(-1,-9)(17,9)
\multido{\n=0+2}{9}{\psline[linestyle=dashed](\n,-9)(\n,9)}
\multido{\n=-8+2}{9}{\psline[linestyle=dashed](-0.5,\n)(17,\n)}
\psaxes[linewidth=1.5pt,Dx=2,Dy=2](0,0)(-0.9,-9)(17,9)
\uput[dl](0,0){O}
\end{pspicture}
\end{center}

\newpage

\vspace{0,5cm}

\textbf{Exercice 3 \hfill 5 points}

\textbf{Candidats ayant  suivi l'enseignement de spécialité}

\medskip

Le gestionnaire d'un site web, composé de trois pages web numérotées de 1 à 3 et reliées entre elles par des liens hypertextes, désire prévoir la fréquence de connexion sur chacune de ses pages web.\index{matrices}

\medskip
 
Des études statistiques lui ont permis de s'apercevoir que :
 
$\bullet~~$ Si un internaute est sur la page \no 1, alors il ira, soit sur la page \no 2 avec la probabilité $\dfrac{1}{4}$, soit sur la page \no 3 avec la probabilité $\dfrac{3}{4}$.

$\bullet~~$ Si un internaute est sur la page \no 2, alors, soit il ira sur la page \no 1 avec la probabilité $\dfrac{1}{2}$ soit il  
restera sur la page \no 2 avec la probabilité $\dfrac{1}{4}$, soit il ira sur la page \no 3 avec la probabilité $\dfrac{1}{4}$. 

$\bullet~~$ Si un internaute est sur la page \no 3, alors, soit il ira sur la page \no 1 avec la probabilité $\dfrac{1}{2}$, soit il  
ira sur la page \no 2 avec la probabilité $\dfrac{1}{4}$,soit il restera sur la page \no 3 avec la probabilité $\dfrac{1}{4}$. 

\medskip

Pour tout entier naturel $n$, on définit les évènements et les probabilités suivants : 

$A_{n}$ : \og Après la $n$-ième navigation, l'internaute est sur la page \no 1 \fg{} et on note $a_{n} = P\left(A_{n}\right)$.
 
$B_{n}$ : \og Après la $n$-ième navigation, l'internaute est sur la page \no 2 \fg{} et on note $b_{n} = P\left(B_{n}\right)$.
 
$C_{n}$ : \og Après la $n$-ième navigation, l'internaute est sur la page \no 3 \fg{} et on note $c_{n} = P\left(C_{n}\right)$.

\medskip
 
\begin{enumerate}
\item Montrer que, pour tout entier naturel $n$, on a $a_{n+1} = \dfrac{1}{2} b_{n} + \dfrac{1}{2}c_{n}$. 

On admet que, de m\^eme, $b_{n+1} = \dfrac{1}{4}a_{n} + \dfrac{1}{4}b_{n} + \dfrac{1}{4}c_{n}$ et $c_{n+1} = \dfrac{3}{4}a_{n} + \dfrac{1}{4}b_{n} + \dfrac{1}{4}c_{n}$. 

Ainsi : 
\renewcommand\arraystretch{1.8}
\[\left\{\begin{array}{l c l}
a_{n+1} &=& \dfrac{1}{2} b_{n} + \dfrac{1}{2}c_{n}\\
b_{n+1} &=& \dfrac{1}{4}a_{n} + \dfrac{1}{4}b_{n} + \dfrac{1}{4}c_{n}\\
c_{n+1} &=& \dfrac{3}{4}a_{n} + \dfrac{1}{4}b_{n} + \dfrac{1}{4}c_{n}
\end{array}\right.\]
\renewcommand\arraystretch{1}
 
\item Pour tout entier naturel $n$, on pose $U_{n} = \begin{pmatrix} a_{n}\\b_{n}\\c_{n}\end{pmatrix}$.
 
$U_{0} = \begin{pmatrix} a_{0}\\b_{0}\\c_{0}\end{pmatrix}$ représente la situation initiale, avec $a_{0} + b_{0} + c_{0} = 1$.
 
Montrer que, pour tout entier naturel $n,\: U_{n+1} = MU_{n}$ où $M$ est une matrice $3 \times 3$ que l'on précisera.
 
En déduire que, pour tout entier naturel $n, U_{n} = M^nU_{0}$. 
\item Montrer qu'il existe une seule matrice colonne $U =\begin{pmatrix}x\\y\\z\end{pmatrix}$ telle que : $x + y + z = 1$ et $MU = U$. 
\item Un logiciel de calcul formel a permis d'obtenir l'expression de $M^n,\: n$ étant un entier naturel non nul :

\[M^n = \begin{pmatrix}
\frac{1}{3} + \frac{\left( \frac{- 1}{2}\right)^n \times 2}{3}&\frac{1}{3} + \frac{\left( \frac{- 1}{2}\right)^n }{- 3}&\frac{1}{3} + \frac{\left(\frac{- 1}{2}\right)^n}{- 3}\\
\frac{1}{4}&\frac{1}{4}&\frac{1}{4}\\
\frac{5}{12} + \frac{\left(-\left(\frac{- 1}{2}\right)^n\right) \times 2}{3}&\frac{5}{12} + \frac{-\left(\frac{- 1}{2}\right)^n}{-3}&\frac{5}{12} + \frac{-\left(\frac{- 1}{2}\right)^n }{- 3}
\end{pmatrix}\] 

Pour tout entier naturel $n$ non nul, exprimer $a_{n}, \: b_{n}$ et $c_{n}$ en fonction de $n$. En déduire que les suites $\left(a_{n}\right), \: \left(b_{n}\right)$ et $\left(c_{n}\right)$ convergent vers des limites que l'on précisera. 
\item Interpréter les résultats obtenus et donner une estimation des pourcentages de fréquentation du site à long terme. 
\end{enumerate}

\vspace{0,5cm}

\textbf{Exercice 4 \hfill 5 points}

\textbf{Commun à tous les candidats}

\medskip
 
Dans cet exercice, les résultats seront arrondis à $10^{-4}$ près.

\medskip
 
\textbf{Partie A}

\medskip
 
En utilisant sa base de données, la sécurité sociale estime que la proportion de Français présentant, à la naissance, une malformation cardiaque de type anévrisme est de 10\,\%. L'étude a également permis de prouver que 30\,\% des Français présentant, à la naissance, une malformation cardiaque de type anévrisme, seront victimes d'un accident cardiaque au cours de leur vie alors que cette proportion n'atteint plus que 8\,\% pour ceux qui ne souffrent pas de cette malformation congénitale.\index{probabilité}

\medskip
 
On choisit au hasard une personne dans la population française et on considère les évènements :
 
$M$ : \og La personne présente, à la naissance, une malformation cardiaque de type anévrisme \fg

$C$ : \og La personne est victime d'un accident cardiaque au cours de sa vie \fg.

\medskip
 
\begin{enumerate}
\item 
	\begin{enumerate}
		\item Montrer que $P(M \cap C) = 0,03$. 
		\item Calculer $P(C)$.
	\end{enumerate} 
\item On choisit au hasard une victime d'un accident cardiaque. Quelle est la probabilité qu'elle présente une malformation cardiaque de type anévrisme ?
\end{enumerate}

\bigskip
 
\textbf{Partie B}

\medskip 

La sécurité sociale décide de lancer une enquête de santé publique, sur ce problème de malformation cardiaque de type anévrisme, sur un échantillon de $400$~personnes, prises au hasard dans la population française.
 
On note $X$ la variable aléatoire comptabilisant le nombre de personnes de l'échantillon présentant une malformation cardiaque de type anévrisme.

\medskip
 
\begin{enumerate}
\item Définir la loi de la variable aléatoire $X$. 
\item Déterminer $P(X = 35)$. 
\item Déterminer la probabilité que $30$ personnes de ce groupe, au moins, présentent une malformation cardiaque de type anévrisme.
\end{enumerate}
 
\bigskip
 
\textbf{Partie C}

\medskip
 
\begin{enumerate}
\item On considère la variable aléatoire $F$, définie par $F = \dfrac{X}{400},\:X$ étant la variable aléatoire de 
la \textbf{partie B}.
 
Déterminer l'intervalle de fluctuation asymptotique de la variable aléatoire $F$ au seuil de $95$\,\%. 
\item Dans l'échantillon considéré, $60$~personnes présentent une malformation cardiaque de type anévrisme.
 
Qu'en pensez-vous? 
\end{enumerate}
%%%%%%%%%%%%%   fin Amérique du Sud novembre 2013
\newpage
%%%%%%%%%%%%%   Nouvelle-Calédonie mars 2014
\hypertarget{Caledoniemars}{}

\lfoot{\small{Nouvelle-Calédonie}}
\rfoot{\small{7 mars 2014}}
\marginpar{\rotatebox{90}{\textbf{A. P. M. E. P.}}}
\renewcommand \footrulewidth{.2pt}
\pagestyle{fancy}
\thispagestyle{empty}

\begin{center}\textbf{Durée : 4 heures}

\vspace{0,5cm}

{\Large\textbf{\decofourleft~Baccalauréat S  Nouvelle-Calédonie~\decofourright\\7 mars 2014}}
\end{center}

\vspace{0,5cm}

\textbf{\textsc{Exercice 1}\hfill 4 points}

\textbf{Commun à tous les candidats}

\medskip
 
\emph{Cet exercice est un QCM (questionnaire à choix multiple). Pour chaque question, une seule des quatre réponses proposées est exacte.\\ 
\textbf{Le candidat indiquera SUR la copie le numéro de la question et la réponse choisie.}\\ 
Chaque réponse exacte rapporte un point. Aucune justification n'est demandée. Aucun point n'est enlevé en l'absence de réponse ou en cas de réponse fausse.}

\medskip
 
Le plan complexe est rapporté au repère orthonormal direct \Ouv. Soit $z$ un nombre complexe de la forme $x + \text{i}y$, où $x$ et $y$ sont des réels.

\medskip
 
\begin{enumerate}
\item Soit $z$ le nombre complexe \index{complexes} d'affixe $(1 + \text{i})^4$. L'écriture exponentielle de $z$ est : 
	\begin{enumerate}
		\item $\sqrt{2}\text{e}^{\text{i}\pi}$
		\item $4\text{e}^{\text{i}\pi}$ 
		\item $\sqrt{2}\text{e}^{\text{i}\frac{\pi}{4}}$ 
		\item $4\text{e}^{\text{i}\frac{\pi}{4}}$
	\end{enumerate} 
\item L'ensemble des points $M$ du plan d'affixe $z = x + \text{i}y$ tels que $|z - 1 + \text{i}| = \left|\sqrt{3} - \text{i}\right|$ a pour équation : 
	\begin{enumerate}
		\item $(x - 1)^2 + (y + 1)^2 = 2$ 
		\item $(x + 1)^2 + (y - 1)^2 = 2$ 
		\item $(x - 1)^2 + (y + 1)^2 = 4$ 
		\item $y = x +  \frac{\sqrt{3} - 1}{2}$ 
	\end{enumerate} 
\item On considère la suite de nombres complexes $\left(Z_{n}\right)$ définie pour tout entier naturel $n$ par $Z_{0} = 1 + \text{i}$ et $Z_{n+1} = \frac{1 + \text{i}}{2}Z_{n}$. On note $M_{n}$ le point du plan d'affixe $Z_{n}$. 
	\begin{enumerate}
		\item Pour tout entier naturel $n$, le point $M_{n}$ appartient au cercle de centre O et de rayon $\sqrt{2}$.
		\item Pour tout entier naturel $n$, le triangle O$M_{n}M_{n + 1}$ est équilatéral. 
		\item La suite $\left(U_{n}\right)$ définie par $U_{n} = \left|Z_{n}\right|$  est convergente. 
		\item Pour tout entier naturel $n$, un argument de $\dfrac{Z_{n+1} - Z_{n}}{Z_{n}}$  est $\frac{\pi}{2}$. 
		\end{enumerate} 
\item Soit A, B, C trois points du plan complexe d'affixes respectives : 

\[Z_{\text{A}}= - 1 - \text{i} \quad ;\quad  Z_{\text{B}} = 2 - 2\text{i}\quad  \text{et}\quad  Z_{\text{C}} = 1 + 5\text{i}.\]

On pose $Z = \dfrac{Z_{\text{C}} - Z_{\text{A}}}{Z_{\text{B}} - Z_{\text{A}}}$.
	\begin{enumerate}
		\item $Z$ est un nombre réel. 
		\item Le triangle ABC est isocèle en A. 
		\item Le triangle ABC est rectangle en A. 
		\item Le point $M$ d'affixe $Z$ appartient à la médiatrice du segment [BC].
	\end{enumerate} 
\end{enumerate}

\vspace{0,5cm}

\textbf{\textsc{Exercice 2}\hfill 6 points}

\textbf{Commun à tous les candidats}

\medskip
 
\textbf{Les parties A, B et C sont indépendantes}

\medskip
 
\textbf{Partie A}

\medskip
 
\textbf{Restitution organisée des connaissances}
\index{ROC}
\medskip
 
L'objectif de cette partie est de démontrer le théorème suivant :
 
\begin{tabularx}{\linewidth}{m{1.5cm}|X}
&Si $X$ est une variable aléatoire suivant la loi normale centrée réduite, alors pour tout réel $\alpha$ appartenant à l'intervalle ]0~;~ 1[, il existe un unique réel strictement positif $\chi_{\alpha}$ tel que $P\left(- \chi_{\alpha} < X < \chi_{\alpha}\right) = 1 - \alpha$.
\end{tabularx} 

\medskip

Soit $f$  la fonction définie sur l'ensemble des nombres réels $\R$ par 

\[f(t) = \dfrac{1}{\sqrt{2\pi}}\text{e}^{- \frac{t^2}{2}}.\]
 
Soit $H$ la fonction définie et dérivable sur $[0~;~+ \infty[$ par 

\[H(x) = P(- x \leqslant X \leqslant x) = \displaystyle\int_{- x}^{x} f(t)\:\text{d}t.\]

\begin{enumerate}
\item Que représente la fonction $f$ pour la loi normale \index{loi normale} centrée réduite ? 
\item Préciser $H(0)$ et la limite de $H(x)$ quand $x$ tend vers $+ \infty$. 
\item À l'aide de considérations graphiques, montrer que pour tout nombre réel positif $x,\: H(x) = 2\displaystyle\int_{0}^{x} f(t)\:\text{d}t$. 
\item En déduire que la dérivée $H'$ de la fonction $H$ sur $[0~;~+ \infty[$ est la fonction $2f$ et dresser le tableau de variations de $H$ sur $[0~;~+ \infty[$. 
\item Démontrer alors le théorème énoncé. 
\end{enumerate}

\bigskip

\textbf{Partie B}

\medskip
 
Un laboratoire se fournit en pipettes auprès de deux entreprises, notées A et B.
 
60\,\% des pipettes viennent de l'entreprise A et 4,6\,\% des pipettes de cette entreprise possèdent un défaut.
 
Dans le stock total du laboratoire, 5\,\% des pièces présentent un défaut. On choisit au hasard une pipette dans le stock du laboratoire et on note :

\setlength\parindent{8mm} 
\begin{description}
\item[ ] $A$ l'évènement : \og La pipette est fournie par l'entreprise A \fg{} ;
\item[ ] $B$ l'évènement : \og La pipette est fournie par l'entreprise B \fg{} ; 
\item[ ] $D$ l'évènement : \og La pipette a un défaut \fg.
\end{description}
\setlength\parindent{0mm} 

\medskip
 
\begin{enumerate}
\item La pipette choisie au hasard présente un défaut ; quelle est la probabilité qu'elle vienne de l'entreprise A ? 
\item Montrer que $p(B \cap D) = \np{0,0224}$. 
\item Parmi les pipettes venant de l'entreprise B, quel pourcentage de pipettes présente un défaut ? 
\end{enumerate}

\bigskip
 
\textbf{Partie C}

\medskip
 
Une pipette est dite conforme si sa contenance est comprise, au sens large entre 98 millilitres (mL) et 102 mL.
 
Soit $X$ la variable aléatoire qui à chaque pipette prise au hasard dans le stock d'un laboratoire associe sa contenance (en millilitres).

On admet que $X$ suit une loi normale \index{loi normale} de moyenne $\mu$ et écart type $\sigma$ tels que $\mu = 100$ et $\sigma^2 = \np{1,0424}$.

\medskip

\begin{enumerate}
\item Quelle est alors la probabilité, à $10^{-4}$ près, pour qu'une pipette prise au hasard soit conforme ? On pourra s'aider de la table ci-dessous ou utiliser une calculatrice.

\begin{center}
\begin{tabularx}{\linewidth}{|m{2.5cm}|*{5}{>{\centering \arraybackslash}X|}}\hline 
Contenance $x$ (en mL)& 95 &96 &97 &98 &99\\ \hline 
$P(X \leqslant x)$ $\left(\text{arrondi à }\:10^{- 5}\right)$&\np{0,00000} &\np{0,00004} &\np{0,00165} &\np{0,02506} &\np{0,16368}\\ \hline\hline 
Contenance $x$ (en mL)&100 &101 &102 &103 &104\\ \hline
$P(X \leqslant x)$ $\left(\text{arrondi à }\:10^{- 5}\right)$&0,5 &\np{0,83632} &\np{0,97494} &\np{0,99835} 
&\np{0,99996}\\ \hline  
\end{tabularx}
\end{center}
 
Pour la suite, on admet que la probabilité pour qu'une pipette soit non-conforme est $p = 0,05$. 
\item On prélève dans le stock du laboratoire des échantillons de pipettes de taille $n$, où $n$ est un  entier naturel supérieur ou égal à $100$. On suppose que le stock est assez important pour considérer ces tirages comme indépendants.
 
Soit $Y_{n}$ la variable aléatoire qui à chaque échantillon de taille $n$ associe le nombre de pipettes non-conformes de l'échantillon. 
	\begin{enumerate}
		\item Quelle est la loi suivie par la variable aléatoire $Y_{n}$ ? 
		\item Vérifier que $n \geqslant 30,\: np \geqslant 5$ et $n(1 - p) \geqslant 5$. 
		\item Donner en fonction de $n$ l'intervalle de fluctuation asymptotique \index{intervalle de fluctuation asymptotique} au seuil de 95\,\% de la fréquence des pipettes non-conformes dans un échantillon.
	\end{enumerate} 
\end{enumerate}

\vspace{0,5cm}

\textbf{\textsc{Exercice 3}\hfill 5 points}

\textbf{Commun à tous les candidats}

\medskip

\textbf{Partie A}

\medskip
 
Soit $f$ la fonction dérivable, définie sur l'intervalle $]0~;~+ \infty [$ par 

\[f(x) = x\ln (x).\]
 
\begin{enumerate}
\item Déterminer les limites de $f$ en $0$ et en $+ \infty$. 
\item On appelle $f'$ la fonction dérivée de $f$ sur $]0~;~+ \infty [$.
 Montrer que 
 
 $f'(x) = \ln(x) + 1$. 
\item Déterminer les variations de $f$ sur $]0~;~+ \infty [$.
\end{enumerate}

\bigskip
 
\textbf{Partie B}

\medskip
 
Soit $\mathcal{C}$ la courbe représentative de la fonction $f$ dans un repère orthonormal.
 
Soit $\mathcal{A}$ l'aire, exprimée en unités d'aire, de la partie du plan comprise entre l'axe des abscisses, la courbe $\mathcal{C}$ et les droites d'équations respectives $x = 1$ et $x = 2$.
 
On utilise l'algorithme suivant pour calculer, par la méthode des rectangles, une valeur approchée de l'aire $\mathcal{A}$. (voir la figure ci-après). 

\begin{center}
\psset{unit=3.6cm}
\begin{pspicture}(-0.2,-0.4)(3.2,1.5)
\psaxes[linewidth=1.5pt]{->}(0,0)(-0.2,-0.4)(3.2,1.5)
\psplot[plotpoints=5000,linewidth=1.25pt,linecolor=blue]{0.001}{2.1}{x x ln mul}
\psframe[fillstyle=hlines](1,0)(1.25,0.279)
\psframe[fillstyle=hlines](1.25,0)(1.5,0.608)
\psframe[fillstyle=hlines](1.5,0)(1.75,0.979)
\psframe[fillstyle=hlines](1.75,0)(2,1.386)
\psframe[fillstyle=vlines](1.25,0)(1.5,0.279)
\psframe[fillstyle=vlines](1.5,0)(1.75,0.608)
\psframe[fillstyle=vlines](1.75,0)(2,0.979)
\uput[dl](0,0){O}
\uput[dr](0.7,-0.2){$\mathcal{C}$}
\end{pspicture}
\end{center}
 
\textbf{Algorithme :}\index{algorithme}

\begin{center}
\begin{tabularx}{0.6\linewidth}{|X|}\hline
\textbf{Variables}\\ 
\hspace{0,8cm}$k$ et $n$ sont des entiers naturels \\
\hspace{0,8cm}$U, V$ sont des nombres réels\\ 
\textbf{Initialisation}\\ 
\hspace{0,8cm} $U$ prend la valeur 0\\
\hspace{0,8cm} $V$ prend la valeur 0\\
\hspace{0,8cm} $n$ prend la valeur 4\\ 
\textbf{Traitement}\\ 
Pour $k$ allant de $0$ à $n - 1$\\ 
\hspace{0,8cm} Affecter à $U$ la valeur $U + \frac{1}{n}f\left(1 + \frac{k}{n}\right)$\\ 
\hspace{0,8cm} Affecter à $V$ la valeur $V + \frac{1}{n}f\left(1 + \frac{k + 1}{n}\right)$\\ 
Fin pour\\ 
\textbf{Affichage}\\ 
\hspace{0,8cm} Afficher $U$\\ 
\hspace{0,8cm} Afficher $V$\\ \hline 
\end{tabularx}
\end{center} 

\begin{enumerate}
\item 
	\begin{enumerate}
		\item Que représentent $U$ et $V$ sur le graphique précédent ? 
		\item Quelles sont les valeurs $U$ et $V$ affichées en sortie de l'algorithme (on donnera une valeur approchée de $U$ par défaut à $10^{-4}$ près et une valeur approchée par excès de $V$ à $10^{-4}$ près) ? 
		\item En déduire un encadrement de $\mathcal{A}$.
	\end{enumerate} 
\item Soient les suites $\left(U_{n}\right)$ et $\left(V_{n}\right)$ définies pour tout entier $n$ non nul par : 

\renewcommand\arraystretch{1.8}
\[\begin{array}{l c l}
U_{n}& =&\dfrac{1}{n}\left[f(1) + f\left(1 + \dfrac{1}{n}\right) + f\left(1 + \dfrac{2}{n}\right) + \cdots  + f\left(1 + \dfrac{n-1}{n}\right)\right]\\
 V_{n}&=&\dfrac{1}{n}\left[f\left(1 + \dfrac{1}{n}\right) + f\left(1 + \dfrac{2}{n}\right) + \cdots + f\left(1 + \dfrac{n-1}{n}\right) + f(2)\right]
\end{array}.\]
\renewcommand\arraystretch{1}
  
On admettra que, pour tout $n$ entier naturel non nul, $U_{n} \leqslant \mathcal{A} \leqslant  V_{n}$. 
	\begin{enumerate}
		\item Trouver le plus petit entier $n$ tel que $V_{n} - U_{n} < 0,1$. 
		\item Comment modifier l'algorithme précédent pour qu'il permette d'obtenir un encadrement de $\mathcal{A}$ d'amplitude inférieure à $0,1$ ?
	\end{enumerate}
\end{enumerate}
	
\bigskip
	 
\textbf{Partie C}

\medskip
 
Soit $F$ la fonction dérivable, définie sur $]0~;~+ \infty[$ par 

\[F(x) = \dfrac{x^2}{2} \ln x -  \dfrac{x^2}{4}.\]

\begin{enumerate}
\item Montrer que $F$ est une primitive de $f$ sur $]0~;~+ \infty[$. 
\item Calculer la valeur exacte de $\mathcal{A}$. 
\end{enumerate}

\vspace{0,5cm}

\textbf{\textsc{Exercice 4}\hfill 5 points}

\textbf{Pour les candidats n'ayant pas suivi l'enseignement de spécialité}

\medskip 
 
Soit ABCDEFGH un parallélépipède rectangle tel que AB = 2, AD = 3 et AE = 1.
 
On appelle respectivement I, J et P les milieux respectifs des segments [CD], [EF] et [AB].
 
On note Q le point défini par $\vect{\text{AQ}}= \dfrac{1}{3}\vect{\text{AD}}$. 

\begin{center}
\psset{unit=0.7cm}
\begin{pspicture}(13,7)
%\psgrid
\psframe(10.3,3.5)%BCGF
\psline(10.3,0)(12,2.1)(12,6.1)(10.3,3.5)%CDHG
\psline(12,6.1)(1.7,6.1)(0,3.5)%HEF
\psline[linestyle=dashed](0,0)(1.7,2.1)(1.7,6.1)%BAE
\psline[linestyle=dashed](1.7,2.1)(12,2.1)%AD
\uput[l](1.7,2.1){A} \uput[dl](0,0){B} \uput[dr](10.3,0){C} 
\uput[r](12,2.1){D} \uput[ul](1.7,6.1){E} \uput[l](0,3.5){F} 
\uput[dr](10.3,3.5){G} \uput[ur](12,6.1){H} \uput[ul](11.15,1.05){I} 
\uput[ul](0.85,4.8){J} \uput[ul](0.85,1.05){P}\uput[u](5.11,2.1){Q}
\psline[linewidth=1.25pt]{->}(1.7,2.1)(0.85,1.05)
\psline[linewidth=1.25pt]{->}(1.7,2.1)(5.11,2.1)
\psline[linewidth=1.25pt]{->}(1.7,2.1)(1.7,6.1)
\psdots[dotstyle=+,dotangle=45,dotscale=1.6](0.85,1.05)(5.11,2.1)(0.85,4.8)(11.15,1.05) 
\end{pspicture}
\end{center}

\bigskip

On appelle \textbf{plan médiateur d'un segment} le plan perpendiculaire à ce segment et passant par son milieu. 

L'objectif de l'exercice est de déterminer les coordonnées du centre d'une sphère circonscrite au tétraèdre ABIJ (c'est-à-dire une sphère qui passe par les quatre points A, B, I, J).
 
L'espace est rapporté au repère orthonormal $\left(\text{A}~;~\vect{\text{AP}},~\vect{\text{AQ}},~\vect{\text{AE}}\right)$.

\medskip
 
\begin{enumerate}
\item Justifier que les quatre points A, B, I et J ne sont pas coplanaires. 
\item Déterminer une équation cartésienne du plan médiateur 
$\left(P_{1}\right)$ du segment [AB]. 
\item Soit $\left(P_{2}\right)$ le plan d'équation cartésienne $3y - z - 4 = 0$. 

Montrer que le plan $\left(P_{2}\right)$ est le plan médiateur du segment [IJ]. 
\item  
	\begin{enumerate}
		\item Démontrer que les plans $\left(P_{1}\right)$ et $\left(P_{2}\right)$ sont sécants. 
		\item Montrer que leur intersection est une droite $(\Delta)$ dont une représentation paramétrique est 
 
\[\left\{\begin{array}{l c l}
x &=& 1\\ 
y &=& t\\
z &=& 3t - 4
\end{array}\right.\: \text{où }\: t\: \text{décrit l'ensemble des nombres réels } \:\R.\]
  
		\item Déterminer les coordonnées du point $\Omega$ de la droite $(\Delta)$ tel que $\Omega$A = $\Omega$I. 
		\item Montrer que le point $\Omega$ est centre de la sphère circonscrite au tétraèdre ABIJ. 
	\end{enumerate}
\end{enumerate}
\printindex
\end{document}