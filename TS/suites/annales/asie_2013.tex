\textbf{Partie\baremeexo{14} A}

\medskip
 
On considère la suite\index{suite} $\left(u_{n}\right)$ définie par : $u_{0} = 2$ et, pour tout entier nature $n$ : 

\[u_{n+1} = \dfrac{1 + 3u_{n}}{3 + u_{n}}.\] 
 
On admet que tous les termes de cette suite sont définis et strictement positifs.

\medskip
 
\begin{enumerate}
\item Démontrer\baremeque{2} par récurrence que, pour tout entier naturel $n$, on a : $u_{n} > 1$. 
\item  
	\begin{enumerate}
		\item Établir\baremeque{1} que, pour tout entier naturel $n$, on a : $u_{n+1}- u_{n} = \dfrac{\left(1 - u_{n} \right)\left(1 + u_{n} \right)}{3+ u_{n}}$.
		\item Déterminer\baremeque{2} le sens de variation de la suite $\left(u_{n}\right)$. 

En déduire que la suite $\left(u_{n}\right)$ converge. 
	\end{enumerate}
\end{enumerate}
	
\bigskip

\textbf{Partie B}

\medskip

On considère la suite\index{suite} $\left(u_{n}\right)$ 	définie par : $u_{0} = 2$ et, pour tout entier nature $n$ :

\[u_{n+1} = \dfrac{1 + 0,5u_{n}}{0,5 + u_{n}}.\]
 
On admet que tous les termes de cette suite sont définis et strictement positifs.

\medskip
 
\begin{enumerate}
\item On considère l'algorithme suivant :
\begin{center}
\begin{tabular}{|c |l|}\hline
 Entrée& Soit un entier naturel non nul $n$\\ \hline 
Initialisation &Affecter à $u$ la valeur 2\\ \hline 
\multirow{4}{1.2cm}{Traitement et sortie }&POUR $i$ allant de 1 à $n$\\ 
&\hspace{1cm}Affecter \`a $u$ la valeur $\dfrac{1 + 0,5u}{0,5 + u}$\\  
&\hspace{1cm}Afficher $u$\\ \hline 
&FIN POUR\\ \hline
\end{tabular}
\end{center}
 
Compléter \baremeque{1}le tableau suivant, en faisant fonctionner cet algorithme pour $n = 3$. Les valeurs de $u$ seront arrondies au millième. 

\begin{center}
\begin{tabularx}{0.6\linewidth}{|*{4}{>{\centering \arraybackslash}X|}}\hline 
$i$&1&2& 3\\ \hline 
$u$&&&\\ \hline 
\end{tabularx}
\end{center} 
\item Pour $n = 12$, on a prolongé le tableau précédent et on a obtenu : 

\begin{center}
\begin{tabularx}{\linewidth}{|c|*{9}{>{\centering \arraybackslash}X|}}\hline 
$i$&4&5&6&7&8&9&10&11&12\\ \hline
$u$&\footnotesize\np{1,0083}&\footnotesize\np{0,9973}&\footnotesize\np{1,0009}&\footnotesize\np{0,9997}&\footnotesize\np{1,0001}&\footnotesize \np{0,99997}&\footnotesize\np{1,00001}&\footnotesize \np{0,999996}&\footnotesize \np{1,000001}\\ \hline
\end{tabularx}
\end{center}

Conjecturer\baremeque{1} le comportement de la suite $\left(u_{n}\right)$ à l'infini. 
\item On considère la suite $\left(v_{n}\right)$ définie, pour tout entier naturel $n$, par : $v_{n} = \dfrac{u_{n} - 1}{u_{n} + 1}$. 
	\begin{enumerate}
		\item Démontrer\baremeque{1} que la suite $\left(v_{n}\right)$ est géométrique de raison $- \dfrac{1}{3}$. 
		\item Calculer\baremeque{2} $v_{0}$ puis écrire $v_{n}$ en fonction de $n$.
	\end{enumerate} 
\item
	\begin{enumerate}
		\item Montrer\baremeque{1} que, pour tout entier naturel $n$, on a : $v_{n} \neq 1$. 
		\item montrer\baremeque{1} que, pour tout entier naturel $n$, on a : $u_{n} = \dfrac{1 + v_{n}}{1 - v_{n}}$.  
		\item En déduire\baremeque{1} une expression de $u_n$ en fonction de $n$ pour tout entier
		naturel $n$.
		\item Déterminer\baremeque{1} la limite de la suite $\left(u_{n}\right)$. 
	\end{enumerate}
\end{enumerate}
