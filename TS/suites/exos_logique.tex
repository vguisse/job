\documentclass[a4paper,11pt,DIV18,BCOR0mm]{scrartcl}
\usepackage[utf8]{inputenc}
\usepackage[francais]{babel}
\usepackage[T1]{fontenc}
\usepackage{amsmath}
\usepackage{amssymb}
\usepackage{color}
\usepackage{textcomp}
\usepackage[official]{eurosym}
\usepackage{enumitem}
\usepackage{numprint}
\usepackage{graphicx}
\usepackage{xifthen}
\usepackage{pifont}
%\usepackage{xlop}
\usepackage{subfig}
\usepackage{cellspace,stmaryrd}
\usepackage[french]{varioref}
\usepackage{pstricks,pst-circ,pstricks-add,pst-plot,pst-math}
%\usepackage{pst-all,pst-tree,pst-3dplot}
\RequirePackage[amsmath,thmmarks,hyperref,framed]{ntheorem}
\usepackage{mathrsfs}

% ==================================================
% Intervalles
% ==================================================
\newcommand{\intervalle}[4]{\mathopen{#1}#2\mathpunct{};#3\mathclose{#4}}
\newcommand{\intff}[2]{\intervalle{[}{#1}{#2}{]}}
\newcommand{\intof}[2]{\intervalle{]}{#1}{#2}{]}}
\newcommand{\intfo}[2]{\intervalle{[}{#1}{#2}{[}}
\newcommand{\intoo}[2]{\intervalle{]}{#1}{#2}{[}}
\newcommand{\intn}[2]{\intervalle{\llbracket}{#1}{#2}{\rrbracket}}
\newcommand{\bigintervalle}[4]{\bigl{#1}#2\mathpunct{};#3\bigr{#4}}
\newcommand{\bigintff}[2]{\bigintervalle{[}{#1}{#2}{]}}
\newcommand{\bigintof}[2]{\bigintervalle{]}{#1}{#2}{]}}
\newcommand{\bigintfo}[2]{\bigintervalle{[}{#1}{#2}{[}}
\newcommand{\bigintoo}[2]{\bigintervalle{]}{#1}{#2}{[}}
\newcommand{\bigintn}[2]{\bigintervalle{\llbracket}{#1}{#2}{\rrbracket}}
\newcommand{\Bigintervalle}[4]{\Bigl{#1}#2\mathpunct{};#3\Bigr{#4}}
\newcommand{\Bigintff}[2]{\Bigintervalle{[}{#1}{#2}{]}}
\newcommand{\Bigintof}[2]{\Bigintervalle{]}{#1}{#2}{]}}
\newcommand{\Bigintfo}[2]{\Bigintervalle{[}{#1}{#2}{[}}
\newcommand{\Bigintoo}[2]{\Bigintervalle{]}{#1}{#2}{[}}
\newcommand{\Bigintn}[2]{\Bigintervalle{\llbracket}{#1}{#2}{\rrbracket}}
\newcommand{\biggintervalle}[4]{\biggl{#1}#2\mathpunct{};#3\biggr{#4}}
\newcommand{\biggintff}[2]{\biggintervalle{[}{#1}{#2}{]}}
\newcommand{\biggintof}[2]{\biggintervalle{]}{#1}{#2}{]}}
\newcommand{\biggintfo}[2]{\biggintervalle{[}{#1}{#2}{[}}
\newcommand{\biggintoo}[2]{\biggintervalle{]}{#1}{#2}{[}}
\newcommand{\biggintn}[2]{\biggintervalle{\llbracket}{#1}{#2}{\rrbracket}}
\newcommand{\Biggintervalle}[4]{\Biggl{#1}#2\mathpunct{};#3\Biggr{#4}}
\newcommand{\Biggintff}[2]{\Biggintervalle{[}{#1}{#2}{]}}
\newcommand{\Biggintof}[2]{\Biggintervalle{]}{#1}{#2}{]}}
\newcommand{\Biggintfo}[2]{\Biggintervalle{[}{#1}{#2}{[}}
\newcommand{\Biggintoo}[2]{\Biggintervalle{]}{#1}{#2}{[}}
\newcommand{\Biggintn}[2]{\Biggintervalle{\llbracket}{#1}{#2}{\rrbracket}}

\newcommand{\pt}{.}
\newcommand{\et}{\text{ et }}
\newcommand{\si}{\text{ si }}
\newcommand{\tq}{\text{ tq }}
\newcommand{\sinon}{\text{ sinon }}
\newcommand{\avec}{\text{ avec }}
\def\pour{\text{~pour~}}

\newenvironment{enumeratecol}[1][2]{\begin{multicols}{#1}\begin{enumerate}}{\end{enumerate}\end{multicols}}


\newcommand{\set}[1]{\left\{#1\right\}}


%\newtheorem{theoreme}{Théorème}
%\newtheorem{axiome}{Axiome}
\newtheorem*{proprieteadmise}{Propriété (admise)}
%\newtheorem*{demonstration}{Démonstration}
\newtheorem*{notation}{Notation}
\newtheorem*{application}{Application}
\newtheorem*{consequence}{Conséquence}
\newtheorem{roc}{Restitution organisée de connaissances.}


\theoremstyle{plain}
\theoremheaderfont{\normalfont\bfseries}
\theorembodyfont{\normalfont}
\theoremseparator{.}
\newtheorem{exercice}{Exercice}
\newtheorem{cours}{Question de cours}
\newtheorem{probleme}{Problème}

\theoremstyle{plain}
\theoremheaderfont{\normalfont\bfseries}
\theorembodyfont{\normalfont}
\theoremseparator{.}
\newtheorem*{rappel}{Rappel}
\newtheorem{exemple}{Exemple}
\newtheorem*{contreexemple}{Contre-exemple}
\newtheorem{remarque}{Remarque}
\newtheorem*{interpretation}{Interprétation}
\newtheorem*{convention}{Convention}
\newtheorem*{vocabulaire}{vocabulaire}

\theoremheaderfont{\sc}\theorembodyfont{\upshape}
\theoremstyle{nonumberplain}
\theoremseparator{ : }
%\theoremsymbol{\rule{1ex}{1ex}}
\theoremsymbol{\ensuremath\square}
\newtheorem{demonstration}{D\'emonstration}



\usepackage[S]{thmbox}
\newtheorem[S]{theoreme}{Théorème}
\newtheorem[S]{propriete}{Propriété}
\newtheorem[S]{axiome}{Axiome}
\newtheorem[S]{methode}{Méthode de résolution}
\newtheorem[L]{definition}{Définition}

\newcommand{\defi}[1]{
\begin{definition}
#1
\end{definition}
}

\newcommand{\theo}[1]{
\begin{theoreme}
#1
\end{theoreme}
}

\newcommand{\demo}[1]{
\begin{demonstration}
#1
\end{demonstration}
}


%-------------------------------------------------
%         TYPOGRAPHIE
%-------------------------------------------------
\newcommand{\celsius}{\,\degres\textrm{C}}


%-------------------------------------------------
%         SCOLAIRE
%-------------------------------------------------
\newcommand{\trou}[1]{\textcolor{white}{#1}}
\newcommand{\prof}[1]{\textcolor{blue}{#1}}
\newcommand{\exercices}[1]{\begin{flushright}\textbf{Exercices : }#1\end{flushright}}

\newcommand{\np}[1]{\numprint{#1}}

\newcommand{\place}[1]{
\vfill\begin{center}
< #1>
\end{center}
\vfill
}

\newif\ifeleve
\def\ac#1{\ifeleve%
\setbox1\hbox{#1}%
\lower2pt\hbox to \wd1{\dotfill}%
\else#1\fi}

\newcommand{\croi}{% croissante
\unitlength=1cm
\begin{minipage} {1cm}%pour centrer verticalement f(x)
\begin{picture}(1,1) % dessin de 2 X 2
\put(0,0){\vector(1,1){1}} % 1 à partir (0,0) direction (1,1)
\end{picture}
\end{minipage}
}
\newcommand{\dec}{%décroissante
\unitlength=1cm
\begin{minipage} {1cm}%pour centrer verticalement f(x)
\begin{picture}(1,1) % dessin de 2 X 2
\put(0,1){\vector(1,-1){1}}
\end{picture}
\end{minipage}
} 

\newcommand{\tauxfxh}[3]{\dfrac{#1(#2+#3)-#1(#2)}{#3}}

\newcounter{QCM}
\setcounter{QCM}{1}
\newcommand{\question}[1]{{\item[Question \arabic{QCM}]\addtocounter{QCM}{1}#1}}
\newcommand{\choix}[3]{
\[
 \begin{tabular}{p{5cm}p{5cm}p{5cm}}
  \textbf{a) }#1.&\textbf{b) }#2.&\textbf{c) }#3.
 \end{tabular}
\]
}

%-------------------------------------------------
%         GEOMETRIE
%-------------------------------------------------
\newcommand{\vect}{\overrightarrow}
\newcommand{\norme}[1]{\|#1\|}
\newcommand{\Norme}[1]{\left\|#1\right\|}
\newcommand{\Oijk}{(O,\vect{i},\vect{j},\vect{k})}
\newcommand{\Oij}{(O,\vect{i},\vect{j})}
\newcommand{\Ouv}{(O,\vect{u},\vect{v})}
\newcommand{\bary}{\mathrm{Bar}}

%-------------------------------------------------
%         FONCTIONS
%-------------------------------------------------
\newcommand{\donne}{\mapsto}
\newcommand{\dx}{\mathrm{d}x}
\newcommand{\dt}{\mathrm{d}t}
\newcommand{\intab}{\int_{a}^{b}}
%-------------------------------------------------
%         PROBA
%-------------------------------------------------
\newcommand{\barre}[1]{\overline{#1}}

%-------------------------------------------------
%         ALGEBRE LINEAIRE
%-------------------------------------------------
\newcommand{\rg}{\text{rg}}
\newcommand{\tr}{\text{tr}}
\renewcommand{\Im}{\text{Im}}
\newcommand{\Vect}{\text{Vect}}
\newcommand{\LL}{\mathcal{L}}
\newcommand{\FF}{\mathcal{F}}
\newcommand{\MM}{\mathcal{M}}
\newcommand{\transpose}{{}^{t}}
%\newcommand{\ker}{\text{ker}}
\newcommand{\coordo}[2]{
\begin{pmatrix}
#1 \\ #2
\end{pmatrix}
}

%-------------------------------------------------
%         ARITHMETIQUE
%-------------------------------------------------
\newcommand{\congru}{\equiv}

%-------------------------------------------------
%         NOMBRES COMPLEXES
%-------------------------------------------------
\newcommand{\e}{\mathrm{e}}
\newcommand{\ii}{\mathrm{i}}
\newcommand{\ei}[1]{\mathrm{e}^{\ii#1}}
\newcommand{\C}{\mathbb{C}}
\newcommand{\U}{\mathbb{U}}
\newcommand{\re}[1]{\mathrm{Re}(#1)}
\newcommand{\im}[1]{\mathrm{Im}(#1)}
\newcommand{\conj}[1]{\overline{#1}}
\newcommand{\abs}[1]{\left\lvert#1\right\rvert}
\renewcommand{\arg}[1]{\mathrm{Arg}(#1)}

\newcommand{\pparmin}[2]{\binom{#2}{#1}}
\newcommand{\Pparmin}[2]{\dbinom{#2}{#1}}

\newcommand{\R}{\mathbb{R}}
\newcommand{\K}{\mathbb{K}}
\newcommand{\N}{\mathbb{N}}
\newcommand{\Z}{\mathbb{Z}}
\newcommand{\D}{\mathbb{D}}
\newcommand{\Q}{\mathbb{Q}}
\newcommand{\rond}{\circ}
\newcommand{\repereoij}{repère orthonormal $(O,\vect{i},\vect{j})$}
\newcommand{\repereoijdirect}{repère orthonormal direct $(O,\vect{i},\vect{j})$}
\newcommand{\accoladedouble}[2]{\left\{\begin{array}{ll}#1\\#2\end{array}\right.}
\newcommand{\soitlasuite}[4]{Soit $(#1)$ la suite définie par $\accoladedouble{#2}{#3\text{ pour tout $#4$}}$}
\newcommand{\coordvect}[2]{\left(\begin{array}{c}#1\\#2\end{array}\right)}
\newcommand{\baremeexo}[1]{\marginpar{\textbf{(#1)}}}
\newcommand{\baremeque}[1]{\marginpar{(#1)}}
\newcommand{\bq}[1]{\marginpar{\textcolor{red}{(#1)}}}


\newcommand{\less}{\leqslant}
\newcommand{\more}{\geqslant}
\newcommand{\equi}{\Longleftrightarrow}
\newcommand{\implique}{\Rightarrow}
\newcommand{\equidef}{\stackrel{def}{\Longleftrightarrow}}
\newcommand{\ssi}{\Longleftrightarrow}
\newcommand{\eqdef}{\stackrel{def}{=}}
\newcommand{\egaldef}{\stackrel{\tiny{def}}{=}}
\newcommand{\egalnot}{\stackrel{\tiny{notation}}{=}}
\newcommand{\latin}[1]{\emph{#1}}
\newcommand{\negl}[1]{\underset{\text{\tiny{$#1$}}}{\ll}}

\newcommand{\limsuite}{\displaystyle\lim_{n\to+\infty}}

% Partie à ommettre en première lecture
\newcommand{\optionnel}[1]{\noindent\hrulefill \rotatebox{180}{\ding{72}\ding{72}\ding{72}} \hrulefill #1 \noindent\hrulefill \ding{72}\ding{72}\ding{72} \hrulefill}
% En tête et pieds custom
\usepackage{fancyhdr}
\pagestyle{fancy}
\lhead{\today}
\rhead{Cours}		
\lfoot{\tiny{vg}}
\cfoot{}
\rfoot{\tiny{Lycée Émile Loubet, Valence}}


% de Dupuy de lome
%\newcommand{\norme}[1]{\left\lVert\ifempty{#1}{\dotpourvariable}{#1}\right\rVert}
\newcommand{\bignorme}[1]{\bigl\lVert#1\bigr\rVert}
\usepackage{hyperref}
\rhead{Pour le lundi 22 septembre 2014}
\chead{}
\lhead{DM : suites, logique, rédaction}
\lfoot{\tiny{vg}}
\cfoot{
%\thepage
}
\rfoot{\tiny{Julie-Victoire Daubié, Argenteuil, TS2}}
\newcommand{\aconnaitre}[1]{{\tiny{À connaître : #1}\par}}
\begin{document}
\subsection*{Propositions}
Un énoncé formel correctement construit est appelé une proposition.
On dira qu'une proposition est vraie si on peut la démontrer, et qu'elle et fausse 
lorsque sa négation (aussi appelée proposition contraire) est vraie.
\begin{exercice}
Soit $u$ la suite définie pour $n\in\N$ par $u_n=n^2-n+41$.
Soit $v$ la suite définie par $v_0=1$ et pour tout $n\in\N$, $v_{n+1}=5v_n-1$,
 et $w$ la suite définie pour $n\in\N$ par $w_n=v_n-\dfrac14$.

 Formuler la négation des propositions suivantes, et dire (sans rédiger de preuve)
laquelle des deux propositions, celle donnée ou sa négation, est vraie.
\begin{description}
 \item[$\mathcal{P}$ :]\og $u_3 = 45.$ \fg
 \item[$\mathcal{Q}$ :]\og $v_3 < 45.$ \fg
 \item[$\mathcal{R}$ :]\og Pour tout $n\in\N$, $u_n\geq40$.\fg
 \item[$\mathcal{S}$ :]\og Pour tout $n\in\N$, $u_n$ est un nombre premier supérieur ou égal à 40.\fg
 \item[$\mathcal{T}$ :]\og La suite $u$ est croissante.\fg
 \item[$\mathcal{U}$ :]\og Il existe $n\in\N$ tel que $u_n>10^6.$ \fg
 \item[$\mathcal{V}$ :]\og La suite $w$ est géométrique de raison 4.\fg
 \item[$\mathcal{W}$ :]\og La suite $w$ est géométrique.\fg
 \item[$\mathcal{X}$ :]\og $\mathcal{S}\implique\mathcal{R}$. \fg
 \item[$\mathcal{Y}$ :]\og $\mathcal{V}\implique\mathcal{W}$. \fg
 \item[$\mathcal{Z}$ :]\og Il existe $n\in\N$ tel que $u_{2n}$ est pair.\fg
\end{description}
\end{exercice}

\aconnaitre{
\begin{itemize}
 
 \item la négation (le contraire) de \og Pour tout $n\in\N$, $\mathcal{P}(n)$\fg{}
se calcule selon : \og Il existe $n\in\N$ tel que NON($\mathcal{P}(n)$)\fg{} ;
 
 \item l'implication vu traditionnellement dans les petites classes sous la forme
\og Si $\mathcal{P}$, alors $\mathcal{Q}$\fg{} se note \og$\mathcal{P}\implique\mathcal{Q}$\fg. Sa négation
se calcule selon : \og$\mathcal{P}$ ET NON($\mathcal{Q}$)\fg. L'implication \og$\mathcal{P}\implique\mathcal{Q}$\fg{}
n'affirme pas $\mathcal{P}$, ni $\mathcal{Q}$, elle ne doit donc pas être confondue avec la déduction
\og $\mathcal{P}$ donc $\mathcal{Q}$\fg, qui est une abréviation pour le fameux et universel\footnote{C'est le \og
On sait que \dots, or \dots, donc \dots\fg{} du collège.} \latin{modus ponens} : \og $\mathcal{P}$, or $\mathcal{P}\implique\mathcal{Q}$, donc $\mathcal{Q}$\fg.

Prouver \og$\mathcal{P}\implique\mathcal{Q}$\fg, c'est produire un discours de la forme : \og Supposons $\mathcal{P}$. Alors on a \dots, donc \dots, \dots, donc $\mathcal{Q}$.


\end{itemize}
}

\subsection*{Preuves}
\begin{exercice}
 Rédiger une preuve de NON($\mathcal{P}$) et de NON($\mathcal{Q}$).
\end{exercice}

\begin{exercice}
 Reformuler $\mathcal{T}$ en utilisant la définition d'une suite croissante.
Rédiger une preuve de $\mathcal{T}$. Rédiger une preuve de $\mathcal{R}$.
\end{exercice}

\aconnaitre{Prouver \og Pour tout $n\in\N,$ $\mathcal{P}(n)$\fg{},
c'est produire un discours de la forme \og Soit $n\in\N$ (sous-entendu 
n'importe lequel), Alors on a \dots, donc \dots, \dots, donc $\mathcal{P}(n)$\fg{}.
}

\begin{exercice}
 Reformuler $\mathcal{V}$ en commençant par \og Pour tout $n\in\N$\fg, puis rédiger 
une preuve de NON($\mathcal{V}$). Rédiger une preuve de $\mathcal{W}$.
\end{exercice}

\begin{exercice}
 Rédiger une preuve de $\mathcal{U}$.
\end{exercice}

\aconnaitre{rédiger une preuve de \og il existe $n\in\N$ tel que $\mathcal{P}(n)$\fg{} consiste
à donner (sans donner d'explication sur comment on le trouve) un entier $t$ comme témoin, et à prouver (souvent une simple vérification
numérique) $\mathcal{P}(t)$.}

\begin{exercice}
 Rédiger une preuve de NON($\mathcal{S}$).
\end{exercice}

\begin{exercice}
 Rédiger une preuve de $\mathcal{Y}$. Est-ce une preuve de $\mathcal{R}$ ?
\end{exercice}

\aconnaitre{pour rédiger une preuve de \og $\mathcal{P}\implique\mathcal{Q}$ \fg,
on commence par \og Supposons $\mathcal{P}$. Alors ...\fg{} et on fait une 
succession de déductions dont la conclusion est $\mathcal{Q}$.

Difficulté : même si la conclusion de la preuve est $\mathcal{Q}$, on n'a pas produit une preuve de $Q$.
C'est une preuve de \og $\mathcal{P}\implique\mathcal{Q}$ \fg, autrement dit
de \og Si $\mathcal{P}$, alors $\mathcal{Q}$\fg. Et on ne s'est jamais prononcé sur $\mathcal{P}$.}


\begin{exercice}[facultatif]
 Rédiger une preuve de NON($\mathcal{Z}$).
\end{exercice}

\pagebreak

\subsection*{Synthèse}

\subsubsection*{Les trois types d'énoncés mathématiques}
\begin{enumerate}
	\item Les définitions, qui seulement des abréviations commodes.
	Exemple : \og Soit $(u_n)_{n\in\N}$ une suite. Lorsque pour tout $n\in\N$,
	$u_{n}\less u_{n+1}$, on dit que $(u_n)_{n\in\N}$ est croissante.\fg
	
	Une fois cette définition énoncée, on peut par exemple abréger
	\og soit $(u_n)_{n\in\N}$ une suite telle que pour tout $n\in\N$,
	$u_n\less u_{n+1}$ \fg{} en \og soit $(u_n)_{n\in\N}$ une suite croissante\fg. 
	
	D'un point de vue logique, les définitions affirment des équivalences vraies
	\og par définition \fg{} :
	\[
	(u_n)_{n\in\N}\text{ est croissante} \ssi \forall n\in\N,\,u_n\less u_{n+1}
	\]
	Il est souvent utile de \og revenir à la définition \fg, c'est à dire de réécrire
	le membre de gauche en le remplaçant par le membre de droite.
	\item Les axiomes, qui sont des propositions dont on admet qu'elles sont vraies.
	par exemple l'axiome de distributivité \og $\forall k,x,y\in\R,\,k(x+y)=kx+ky$ \fg.
	\item Les théorèmes, qui sont des propositions démontrées à partir des axiomes.
	Par exemple
	 
\end{enumerate}

\end{document}
