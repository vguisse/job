\documentclass[a4paper,11pt,DIV12,BCOR0mm]{scrartcl}
\usepackage[utf8]{inputenc}
\usepackage[francais]{babel}
\usepackage[T1]{fontenc}
\usepackage{amsmath}
\usepackage{amssymb}
\usepackage{color}
\usepackage{textcomp}
\usepackage[official]{eurosym}
\usepackage{enumitem}
\usepackage{numprint}
\usepackage{graphicx}
\usepackage{xifthen}
\usepackage{pifont}
%\usepackage{xlop}
\usepackage{subfig}
\usepackage{cellspace,stmaryrd}
\usepackage[french]{varioref}
\usepackage{pstricks,pst-circ,pstricks-add,pst-plot,pst-math}
%\usepackage{pst-all,pst-tree,pst-3dplot}
\RequirePackage[amsmath,thmmarks,hyperref,framed]{ntheorem}
\usepackage{mathrsfs}

% ==================================================
% Intervalles
% ==================================================
\newcommand{\intervalle}[4]{\mathopen{#1}#2\mathpunct{};#3\mathclose{#4}}
\newcommand{\intff}[2]{\intervalle{[}{#1}{#2}{]}}
\newcommand{\intof}[2]{\intervalle{]}{#1}{#2}{]}}
\newcommand{\intfo}[2]{\intervalle{[}{#1}{#2}{[}}
\newcommand{\intoo}[2]{\intervalle{]}{#1}{#2}{[}}
\newcommand{\intn}[2]{\intervalle{\llbracket}{#1}{#2}{\rrbracket}}
\newcommand{\bigintervalle}[4]{\bigl{#1}#2\mathpunct{};#3\bigr{#4}}
\newcommand{\bigintff}[2]{\bigintervalle{[}{#1}{#2}{]}}
\newcommand{\bigintof}[2]{\bigintervalle{]}{#1}{#2}{]}}
\newcommand{\bigintfo}[2]{\bigintervalle{[}{#1}{#2}{[}}
\newcommand{\bigintoo}[2]{\bigintervalle{]}{#1}{#2}{[}}
\newcommand{\bigintn}[2]{\bigintervalle{\llbracket}{#1}{#2}{\rrbracket}}
\newcommand{\Bigintervalle}[4]{\Bigl{#1}#2\mathpunct{};#3\Bigr{#4}}
\newcommand{\Bigintff}[2]{\Bigintervalle{[}{#1}{#2}{]}}
\newcommand{\Bigintof}[2]{\Bigintervalle{]}{#1}{#2}{]}}
\newcommand{\Bigintfo}[2]{\Bigintervalle{[}{#1}{#2}{[}}
\newcommand{\Bigintoo}[2]{\Bigintervalle{]}{#1}{#2}{[}}
\newcommand{\Bigintn}[2]{\Bigintervalle{\llbracket}{#1}{#2}{\rrbracket}}
\newcommand{\biggintervalle}[4]{\biggl{#1}#2\mathpunct{};#3\biggr{#4}}
\newcommand{\biggintff}[2]{\biggintervalle{[}{#1}{#2}{]}}
\newcommand{\biggintof}[2]{\biggintervalle{]}{#1}{#2}{]}}
\newcommand{\biggintfo}[2]{\biggintervalle{[}{#1}{#2}{[}}
\newcommand{\biggintoo}[2]{\biggintervalle{]}{#1}{#2}{[}}
\newcommand{\biggintn}[2]{\biggintervalle{\llbracket}{#1}{#2}{\rrbracket}}
\newcommand{\Biggintervalle}[4]{\Biggl{#1}#2\mathpunct{};#3\Biggr{#4}}
\newcommand{\Biggintff}[2]{\Biggintervalle{[}{#1}{#2}{]}}
\newcommand{\Biggintof}[2]{\Biggintervalle{]}{#1}{#2}{]}}
\newcommand{\Biggintfo}[2]{\Biggintervalle{[}{#1}{#2}{[}}
\newcommand{\Biggintoo}[2]{\Biggintervalle{]}{#1}{#2}{[}}
\newcommand{\Biggintn}[2]{\Biggintervalle{\llbracket}{#1}{#2}{\rrbracket}}

\newcommand{\pt}{.}
\newcommand{\et}{\text{ et }}
\newcommand{\si}{\text{ si }}
\newcommand{\tq}{\text{ tq }}
\newcommand{\sinon}{\text{ sinon }}
\newcommand{\avec}{\text{ avec }}
\def\pour{\text{~pour~}}

\newenvironment{enumeratecol}[1][2]{\begin{multicols}{#1}\begin{enumerate}}{\end{enumerate}\end{multicols}}


\newcommand{\set}[1]{\left\{#1\right\}}


%\newtheorem{theoreme}{Théorème}
%\newtheorem{axiome}{Axiome}
\newtheorem*{proprieteadmise}{Propriété (admise)}
%\newtheorem*{demonstration}{Démonstration}
\newtheorem*{notation}{Notation}
\newtheorem*{application}{Application}
\newtheorem*{consequence}{Conséquence}
\newtheorem{roc}{Restitution organisée de connaissances.}


\theoremstyle{plain}
\theoremheaderfont{\normalfont\bfseries}
\theorembodyfont{\normalfont}
\theoremseparator{.}
\newtheorem{exercice}{Exercice}
\newtheorem{cours}{Question de cours}
\newtheorem{probleme}{Problème}

\theoremstyle{plain}
\theoremheaderfont{\normalfont\bfseries}
\theorembodyfont{\normalfont}
\theoremseparator{.}
\newtheorem*{rappel}{Rappel}
\newtheorem{exemple}{Exemple}
\newtheorem*{contreexemple}{Contre-exemple}
\newtheorem{remarque}{Remarque}
\newtheorem*{interpretation}{Interprétation}
\newtheorem*{convention}{Convention}
\newtheorem*{vocabulaire}{vocabulaire}

\theoremheaderfont{\sc}\theorembodyfont{\upshape}
\theoremstyle{nonumberplain}
\theoremseparator{ : }
%\theoremsymbol{\rule{1ex}{1ex}}
\theoremsymbol{\ensuremath\square}
\newtheorem{demonstration}{D\'emonstration}



\usepackage[S]{thmbox}
\newtheorem[S]{theoreme}{Théorème}
\newtheorem[S]{propriete}{Propriété}
\newtheorem[S]{axiome}{Axiome}
\newtheorem[S]{methode}{Méthode de résolution}
\newtheorem[L]{definition}{Définition}

\newcommand{\defi}[1]{
\begin{definition}
#1
\end{definition}
}

\newcommand{\theo}[1]{
\begin{theoreme}
#1
\end{theoreme}
}

\newcommand{\demo}[1]{
\begin{demonstration}
#1
\end{demonstration}
}


%-------------------------------------------------
%         TYPOGRAPHIE
%-------------------------------------------------
\newcommand{\celsius}{\,\degres\textrm{C}}


%-------------------------------------------------
%         SCOLAIRE
%-------------------------------------------------
\newcommand{\trou}[1]{\textcolor{white}{#1}}
\newcommand{\prof}[1]{\textcolor{blue}{#1}}
\newcommand{\exercices}[1]{\begin{flushright}\textbf{Exercices : }#1\end{flushright}}

\newcommand{\np}[1]{\numprint{#1}}

\newcommand{\place}[1]{
\vfill\begin{center}
< #1>
\end{center}
\vfill
}

\newif\ifeleve
\def\ac#1{\ifeleve%
\setbox1\hbox{#1}%
\lower2pt\hbox to \wd1{\dotfill}%
\else#1\fi}

\newcommand{\croi}{% croissante
\unitlength=1cm
\begin{minipage} {1cm}%pour centrer verticalement f(x)
\begin{picture}(1,1) % dessin de 2 X 2
\put(0,0){\vector(1,1){1}} % 1 à partir (0,0) direction (1,1)
\end{picture}
\end{minipage}
}
\newcommand{\dec}{%décroissante
\unitlength=1cm
\begin{minipage} {1cm}%pour centrer verticalement f(x)
\begin{picture}(1,1) % dessin de 2 X 2
\put(0,1){\vector(1,-1){1}}
\end{picture}
\end{minipage}
} 

\newcommand{\tauxfxh}[3]{\dfrac{#1(#2+#3)-#1(#2)}{#3}}

\newcounter{QCM}
\setcounter{QCM}{1}
\newcommand{\question}[1]{{\item[Question \arabic{QCM}]\addtocounter{QCM}{1}#1}}
\newcommand{\choix}[3]{
\[
 \begin{tabular}{p{5cm}p{5cm}p{5cm}}
  \textbf{a) }#1.&\textbf{b) }#2.&\textbf{c) }#3.
 \end{tabular}
\]
}

%-------------------------------------------------
%         GEOMETRIE
%-------------------------------------------------
\newcommand{\vect}{\overrightarrow}
\newcommand{\norme}[1]{\|#1\|}
\newcommand{\Norme}[1]{\left\|#1\right\|}
\newcommand{\Oijk}{(O,\vect{i},\vect{j},\vect{k})}
\newcommand{\Oij}{(O,\vect{i},\vect{j})}
\newcommand{\Ouv}{(O,\vect{u},\vect{v})}
\newcommand{\bary}{\mathrm{Bar}}

%-------------------------------------------------
%         FONCTIONS
%-------------------------------------------------
\newcommand{\donne}{\mapsto}
\newcommand{\dx}{\mathrm{d}x}
\newcommand{\dt}{\mathrm{d}t}
\newcommand{\intab}{\int_{a}^{b}}
%-------------------------------------------------
%         PROBA
%-------------------------------------------------
\newcommand{\barre}[1]{\overline{#1}}

%-------------------------------------------------
%         ALGEBRE LINEAIRE
%-------------------------------------------------
\newcommand{\rg}{\text{rg}}
\newcommand{\tr}{\text{tr}}
\renewcommand{\Im}{\text{Im}}
\newcommand{\Vect}{\text{Vect}}
\newcommand{\LL}{\mathcal{L}}
\newcommand{\FF}{\mathcal{F}}
\newcommand{\MM}{\mathcal{M}}
\newcommand{\transpose}{{}^{t}}
%\newcommand{\ker}{\text{ker}}
\newcommand{\coordo}[2]{
\begin{pmatrix}
#1 \\ #2
\end{pmatrix}
}

%-------------------------------------------------
%         ARITHMETIQUE
%-------------------------------------------------
\newcommand{\congru}{\equiv}

%-------------------------------------------------
%         NOMBRES COMPLEXES
%-------------------------------------------------
\newcommand{\e}{\mathrm{e}}
\newcommand{\ii}{\mathrm{i}}
\newcommand{\ei}[1]{\mathrm{e}^{\ii#1}}
\newcommand{\C}{\mathbb{C}}
\newcommand{\U}{\mathbb{U}}
\newcommand{\re}[1]{\mathrm{Re}(#1)}
\newcommand{\im}[1]{\mathrm{Im}(#1)}
\newcommand{\conj}[1]{\overline{#1}}
\newcommand{\abs}[1]{\left\lvert#1\right\rvert}
\renewcommand{\arg}[1]{\mathrm{Arg}(#1)}

\newcommand{\pparmin}[2]{\binom{#2}{#1}}
\newcommand{\Pparmin}[2]{\dbinom{#2}{#1}}

\newcommand{\R}{\mathbb{R}}
\newcommand{\K}{\mathbb{K}}
\newcommand{\N}{\mathbb{N}}
\newcommand{\Z}{\mathbb{Z}}
\newcommand{\D}{\mathbb{D}}
\newcommand{\Q}{\mathbb{Q}}
\newcommand{\rond}{\circ}
\newcommand{\repereoij}{repère orthonormal $(O,\vect{i},\vect{j})$}
\newcommand{\repereoijdirect}{repère orthonormal direct $(O,\vect{i},\vect{j})$}
\newcommand{\accoladedouble}[2]{\left\{\begin{array}{ll}#1\\#2\end{array}\right.}
\newcommand{\soitlasuite}[4]{Soit $(#1)$ la suite définie par $\accoladedouble{#2}{#3\text{ pour tout $#4$}}$}
\newcommand{\coordvect}[2]{\left(\begin{array}{c}#1\\#2\end{array}\right)}
\newcommand{\baremeexo}[1]{\marginpar{\textbf{(#1)}}}
\newcommand{\baremeque}[1]{\marginpar{(#1)}}
\newcommand{\bq}[1]{\marginpar{\textcolor{red}{(#1)}}}


\newcommand{\less}{\leqslant}
\newcommand{\more}{\geqslant}
\newcommand{\equi}{\Longleftrightarrow}
\newcommand{\implique}{\Rightarrow}
\newcommand{\equidef}{\stackrel{def}{\Longleftrightarrow}}
\newcommand{\ssi}{\Longleftrightarrow}
\newcommand{\eqdef}{\stackrel{def}{=}}
\newcommand{\egaldef}{\stackrel{\tiny{def}}{=}}
\newcommand{\egalnot}{\stackrel{\tiny{notation}}{=}}
\newcommand{\latin}[1]{\emph{#1}}
\newcommand{\negl}[1]{\underset{\text{\tiny{$#1$}}}{\ll}}

\newcommand{\limsuite}{\displaystyle\lim_{n\to+\infty}}

% Partie à ommettre en première lecture
\newcommand{\optionnel}[1]{\noindent\hrulefill \rotatebox{180}{\ding{72}\ding{72}\ding{72}} \hrulefill #1 \noindent\hrulefill \ding{72}\ding{72}\ding{72} \hrulefill}
% En tête et pieds custom
\usepackage{fancyhdr}
\pagestyle{fancy}
\lhead{\today}
\rhead{Cours}		
\lfoot{\tiny{vg}}
\cfoot{}
\rfoot{\tiny{Lycée Émile Loubet, Valence}}


% de Dupuy de lome
%\newcommand{\norme}[1]{\left\lVert\ifempty{#1}{\dotpourvariable}{#1}\right\rVert}
\newcommand{\bignorme}[1]{\bigl\lVert#1\bigr\rVert}
\rhead{Probabilités}
\begin{document}
\section{Expérience aléatoire, univers, probabilité d'une issue}
Une expérience aléatoire est une expérience qui a plusieurs résultats possibles, appelés
les issues. On ne peut pas savoir à l'avance laquelle de ces issues sera réalisée.

L'univers d'une expérience aléatoire, souvent noté $\Omega$, est l'ensemble
des issues.
\begin{exemple}
 \label{ex:de}On jette un dé à 6 faces numérotées. Il y a 6 résultats différents 
possibles qu'on peut noter 1, 2, 3, 4, 5, et 6. Autrement dit :
\[\Omega=\{1;2;3;4;5;6\}.\]
\end{exemple}
\begin{exemple}
 \label{ex:cartes}On tire au hasard une carte dans un jeu de 32 cartes. Il y a 32 résultats différents possibles qu'on peut noter 
$7_{\spadesuit}$, $8_{\spadesuit}$, ..., $A_{\clubsuit}$. Autrement dit :
\begin{align*}
 \Omega=&\{	7_{\spadesuit};8_{\spadesuit};9_{\spadesuit};10_{\spadesuit};
		V_{\spadesuit};D_{\spadesuit};R_{\spadesuit};A_{\spadesuit}\\
	&	7_{\heartsuit};8_{\heartsuit};9_{\heartsuit};10_{\heartsuit};
		V_{\heartsuit};D_{\heartsuit};R_{\heartsuit};A_{\heartsuit}\\
	&	7_{\diamondsuit};8_{\diamondsuit};9_{\diamondsuit};10_{\diamondsuit};
		V_{\diamondsuit};D_{\diamondsuit};R_{\diamondsuit};A_{\diamondsuit}\\
	&	7_{\clubsuit};8_{\clubsuit};9_{\clubsuit};10_{\clubsuit};
		V_{\clubsuit};D_{\clubsuit};R_{\clubsuit};A_{\clubsuit}\}
\end{align*}
\end{exemple}
\begin{exemple}
 \label{ex:urne}Dans une urne, il y a 6 boule blanches et 4 boules noires. On en tire une au hasard. Il y a 10 résultats différents
possibles que l'on peut noter $b_1$, $b_2$, $b_3$, ..., $b_6$, $n_1$, $n_2$, $n_3$, $n_4$. Autrement dit :
\[
 \Omega=\{b_1;b_2;b_3;b_4;b_5;b_6;b_7;n_1;n_2;n_3\}.
\]


On peut aussi considérer qu'il y a seulement deux résultats différents possibles 
que l'on peut noter $b$ et $n$. Autrement dit :
\[
 \Omega'=\{b;n\}.
\]

\end{exemple}

\begin{definition}
 Soit $\Omega=\{x_1;x_2;...;x_n\}$ l'univers d'une expérience aléatoire. On dit que les 
nombres $p_1$, $p_2$, ..., $p_n$ sont les probabilités des issues $x_1$, $x_2$, ..., $x_n$
lorsqu'ils correspondent aux fréquences d'apparitions de ces issues lorsque l'expérience aléatoire est répétée 
un grand nombre de fois. Ce sont des nombres compris entre $0$ et $1$ et qui vérifient :
\[
 p_1+p_2+\cdots+p_n=1.
\]
Définir un loi de probabilité sur $\Omega$ consiste à donner des nombres $p_1$, $p_2$, $\dots$, $p_n$
convenables.
\end{definition}
\begin{exemple}
 Pour l'exemple du dé, on a la loi de probabilité :
\renewcommand{\arraystretch}{2}
\[
 \begin{array}{|*{7}{c|}}
  \hline
  x_i	&	1	&	2	&	3	&	4	&	5	&	6\\\hline
  p_i	&	\frac16	&	\frac16	&	\frac16	&	\frac16	&	\frac16	&	\frac16\\\hline
 \end{array}
\]
Lorsque que comme dans cet exemple, les probabilités sont toutes égales à un même nombre, on dit
que la loi de probabilité est équirépartie. Dans ce cas, comme la somme des probabilités fait 1
on aura toujours :
\[
 p_i=p=\frac1n.
\]
\end{exemple}
\begin{exemple}
 Pour le tirage dans l'urne, on a la loi de probabilité :
\renewcommand{\arraystretch}{2}
\[
 \begin{array}{|*{7}{c|}}
  \hline
  x_i	&	b		&	n		\\\hline
  p_i	&	\frac{7}{10}	&	\frac{3}{10}	\\\hline
 \end{array}
\]
La loi de probabilité n'est pas équirépartie, mais on a toujours
la somme des probabilités qui fait 1. 
\end{exemple}


\section{\'Evénements, probabilité d'un événement}
\begin{definition}
 Soit $\Omega=\{x_1;x_2;...;x_n\}$ l'univers d'une expérience aléatoire,
un événement est une partie de $\Omega$ (c'est à dire un sous ensemble
de $\Omega$). Il peut être décrit par une phrase (en compréhension) ou par l'ensemble de 
des issues qui le constituent (en extension).
\end{definition}


Ainsi :
\begin{itemize}
 \item dans l'exemple du dé, on peut considérer l'événement $M$ : \og obtenir un multiple de 3\fg.
  On a : $M=\{3;6\}$.
 \item dans l'exemple avec les cartes, on peut considérer l'événement 
$N$ : \og tirer une carte noire\fg, l'événement $V$ : \og Tirer un valet \fg;
\end{itemize}

\begin{definition}
 La probabilité d'un événement $A$ est la somme des probabilités des issues 
qui le constituent. On la note $P(A)$, c'est un nombre compris entre 0 et 1
et qui correspond à la fréquence d'apparition de l'événement $A$.
\end{definition}

Ainsi :
\begin{itemize}
 \item dans l'exemple du dé, on a :
    \[
     P(M)=\frac16+\frac16=\frac26=\frac13.
    \]
  De manière générale, quand la loi est équirépartie, on a toujours :
    \[
     P(A)=\frac{\#A}{\#\Omega}
	 =\frac{\text{Nombre d'issues favorables}}{\text{Nombre d'issues possibles}};
    \]
 \item dans l'exemple avec les cartes, on a :
    \begin{align*}
     P(N)&=\dfrac{16}{32}=\frac12\\
     P(V)&=\dfrac{4}{32}=\frac18.
    \end{align*}
\end{itemize}
\pagebreak
\section{Opérations sur les événements}
\begin{definition}
 L'événement $A\cup B$ (lire $A$ union $B$) est constitué des issues appartenant à l'événement
$A$ ou à l'événement $B$ (ou aux deux à la fois).
\end{definition}
Par exemple l'événement $N\cup V$ : \og Obtenir une carte noire ou obtenir un valet\fg{}
est constitué de 18 issues.
\begin{definition}
 L'événement $A\cap B$ (lire $A$ inter $B$) est constitué des issues appartenant à l'événement
$A$ et à l'événement $B$.
\end{definition}
Par exemple l'événement $N\cap V$ : \og Obtenir une carte noire et obtenir un valet\fg{}
est constitué de 2 issues. 
\begin{definition}[\'Evénement contraire]
 L'événement $\barre{A}$ (lire $A$ barre) est constitué des issues qui n'appartiennent
pas à l'événement $A$.
\end{definition}
Par exemple l'événement $\barre{M}$ : \og Ne pas obtenir un multiple de 3\fg{} est 
constitué de 4 issues.

\end{document}
